\documentclass[12pt]{article}
\usepackage{amsmath, amsthm, amssymb}
\usepackage{fancyhdr, amsmath, amsthm, amssymb, hyperref, paracol, graphicx, setspace}
\usepackage[margin=0.5in]{geometry}
\usepackage[version=3]{mhchem}
\newcommand{\scinot}[2]{#1\times 10^{#2}}
\usepackage[labelfont=bf]{caption}
\begin{document}

\section*{April 3 - Rec}

Two theories of evolution: Lamarckian/Darwinian. First, Lamarckian principles:

\begin{itemize}
	\item use/disuse $\Rightarrow$ acquired traits
	\item acquired traits are passed down
\end{itemize}

This seems fishy, because parents who lose arms have fully-formed children. Darwin comes up with an altrenative set of principles:

\begin{itemize}
	\item genetic variations $\Rightarrow$ traits
	\item fit organisms pass on genes (fitness - the ability to pass genes down)
\end{itemize}

How then do we classify organisms in their evolutionary paths? We can determine this by first establishing necessary criteria (for a good ``chronometer''):

\begin{itemize}
	\item universal; all organisms must have it
	\item conserved function (all organisms must use it to do something)
	\item change slowly over evolutionary time
\end{itemize}

Phylogenetic trees then (which we know the look of it, many Vs and branches), have three terminologies. The root is the beginning of the tree (the globally common ancestor), the leaf is a branch, and a node is a place where the leaves diverge, i.e. a local common ancestor. rRNA is used to build the tree because it changes slowly over evolutionary time (16s or 18s), 16s is for bacteria and archae, 18s is for eukarya. Mutations are the mechanisms for evolution/speciation, or horizontal gene transfer (to be discussed later). 

TA office hours to discuss proposal: 8-9PM Lloyd dining Monday (Harrison), 4-5 Kirchoff 209 Friday (Pei). Recommended website for publications: pubmed.gov

Things to watch out for when evaluating the quality of an article: publication (prestigious?)? print (better)/online? authorship (ugrad/Ph. D)? sources/articles (many sources = quality)? bias/special interest? discipline (correct discipline)?


\section*{April 4 - Evolution}

Aristotle was the first to classify organisms, via unaided vision. Next was Van Leeuwenhoek, who used Light Microscopy, with microbes from the inside of his cheek. Successively, the technologies used for classification included electron microscopy and molecular analysis. The former three techniques were the visual technologies used from 400BC to 1950, and molecular analysis only arose in the last 20-30 years, by the work of scientists such as Woese, Fox, Pace.

During the unaided vision stage, living organisms were classified as plants and animals. Then Van leeuwenhoek classified microbes, which he called animicules (sp?). Electron microscopy gave rise to the five kingdom model: plants, animals, fungi, protists, monera (split into bacteria and archea). Finally, Woese, Fox, and Pace gave the three domains, bacteria, eukarya, and archea.

Woese's seminal paper came in 1990, though his work had been going since 1977, in rRNA sequencing. Sequencing was very slow at first, and only in 1985 with the advent of Polymerase Chain Reaction (PCR) did work really get started.

DNA (Deoxyribonucleic acid) and RNA (Ribonucleic Acid) are comprised of A,C,G, and T (U in RNA). These base pairs are what Woese et. al. worked on. rRNA (ribosomal RNA) is where proteins are made, and because these are very ancient and change slowly, rRNA is perfect for evolutionary classification. 16s/18s RNA were used to classify, where s is the sedimentation (or something); 16s is for non-eukarya, 18s is for eukarya. 

Recent years have completely paradigm shifted bio from thinking that eukarya is king to thinking bacteria/archae are by far God. Viruses are still being debatable as life forms. They are defined as small, infectious, replicating \emph{objects}. They live in host cells and force cells to make new virii, and they have a protein coat plus either DNA or RNA. Big questions now include whether virii are organisms/alive. 

Eukarya use mitochondria because much more efficient in generating ATP from \ce{O2} than \ce{CO2}. Lynn Margulis first proposed endosymbiosis for eukaryotic cells. Additionally, a new view of the mitochondria dictates that the mitochondria do more than generate energy; it tells the cell whether it will live/die and whether it will differentiate.

Evolution is the change in genetic material of a population over generations due to genetic variation, where fitness is then defined as the ability to pass genetic makeup on to offspring. There then must be mechanisms for evolution. One mechanism is a change in genome, such as genetic drift, which creates random genetic novelties. There is also horizontal gene transfer, which is a transfer of large pieces of DNA. Natural selection is then the process whereby advantageous traits grant higher fitness. This then leads most logically to a very gradual evolution process, which is not supported by theory.

Instead, the theory of punctuated equilibrium, proposed by Stephen Gould, stated that evolution occurred in explosions, due to heavy environmental changes. The concept of a biological species was originally defined such that members of the same species produces fertile offspring.

Darwin's theory had some problems though; no mechanism of evolution. Mendel however, came in with epic peas and introduced a genetic mechanism. Mendelian genetics along with Darwinian evolution combined to give the theory termed modern synthesis.

This theory only applies to the Eukarya though! Koonin then proposes that there is not actually an evolutionary tree, a theory called the post modern synthesis. Viruses are also a big hot topic at the moment. 

\section*{April 9 - Cells}

Robert Hooke first discovered cells in 1665, Leevenhoek called them animacules in 1684. Cell theory says that

\begin{itemize}
	\item All life is composed of cells.
	\item The cell is the simplest unit that carries out life.
	\item All cells derive from preexisting cells.
	\item All cells have a membrane.
\end{itemize}

Bacteria and Archae are both 1$\mu$m in diameter, while Eucarya are 10$\mu$m. Otherwise, Bacteria/Archae are so different that ``prokaryote'' is a total misnomer. Eucarya contain bacteria such as chloroplasts and mitochondria.

Bacterial cells have a compact genome called the nucleoid, DNA wrapped around particular proteins. They also have ribosomes, envelope (cell wall + membrane), flagella, and special compartments. The photosynthetic membrane is an example of an adaptive membrane that increases reaction surface area.

An example of current research (AND EXTRA CREDIT) is fonud in certain organisms. A scanning microscope shows that actin is a filament that can maintain magnetite in a line across the length of a bacteria (which I don't know the purpose of, missed it; maybe reacts to magnetism?).

Archae go well out of the range of standard pressure and temperature. Properties include that there is no separation between nucleoid and cell (supercoiling gives stability), abundant ribosomes (machinery closer to Eukarya), cell wall/membrane (different composition from bacteria), and metabolically limited.

Eukaryotic cells have physically separated nuclei (membrane), multiple subcellular compartments (proteins are permenantly separated; organelles), and these organelles sometimes originate from bacteria (endosymbiosis). We list some organelles below:

\begin{enumerate}
	\item Nucleolus - 25\% of the nuclear volume, where genome produces rRNA.
	\item Nucleus - Remainder of DNA.
	\item Ribosomes - Synthesize proteins.
	\item Rough Endoplasmic Reticulum - Shares membrane with nucleus, is studded with ribosomes
	\item Vesicles - Membrane-bound carriers
	\item Golgi body - Modifies proteins and sends them off on their way.
	\item Cytoskeleton - filaments such as actin, tublin (tubules) to direct proteins/organelles around.
	\item Smooth ER - produce lipids, carbs (sugars), detoxification of wastes
	\item Mitochondrion - Creates ATP, other functions that we're not sure of just yet.
	\item Vacuole - Only in plant cells
	\item Cytosol - Substance in cells in which everything is suspended; contains metabolites and concentration gradients.
	\item Lysosome - Digests cell materials and tells cell when to die (apoptosis)
	\item Centrioles - provides coordinates/framework within cell.
\end{enumerate}

Animal cells do not have a cell wall, which plants have. Plant cells exclusively have vacuoles, and plant cells have chloroplasts for photosynthesis, though they still have mitochondria.

There are 200 cell types in humans, including

\begin{itemize}
	\item Neurons - 1m in length, processes neural information
	\item Ostoblasts - Bone formation, helps form bone by moving polymers around
	\item Neutrophils - Defends body against attacking substances.
\end{itemize}

Eukaryotic cell walls are made of lipids, while bacteria cell walls are made of peptidoglycan. Some bacteria have S-layer cell wall, which tests gram positive, while other bacteria have an outer membrane and a cytosol membrane, in between which is the peptidoglycan (composed of NAD and NAG). So gram positive has just one cell membrane, and gram negative has an inner and outer membrane. Note that the two membranes are fundamentally different in composition.

Spores are generated by the inner membrane folding inwards to form a bilayer with a thick coat of peptidoglycan, which is super hardy.

\section*{April 10 - Cells!!}

Life is defined as existing (boundary separating from environment e.g. membrane/phospholipid bilayer) and persist (covered later).

Phospholipid bilayer consists of hydrophilic head, hydrophobic tail. The types of lipids in the bilayer differ between domains, but all are lipids. Cell walls are also part of the boundary, though it does not exist in archae, only in bacteria.

Also, for life, we will need energy. How can we do this? By obtaining food/nourishment, such as by phagocytosis (endocytosis?). Note that while we draw the cell to be somewhat empty, it's actually ridiculously packed.

Moreover, there is some DNA inside the cell, with which we use RNA and ribosomes to create proteins/enzymes to digest the food/nourishment that we absorb. Thus, we currently have a nucleus, nucleoid, ribosomes, and cell wall/membrane in our cells. We also have some ER, Golgi apparatus, vesicles, mitochondria, vacuole+chloroplast (in plant cells only), and others. Then endosymbiosis + mitochondria/chloroplasts, just by some sort of DNA integration after the first endocytosis of a mitochondria. One thing to remember is that eukaryotic ribosomes have 18s rRNA while bacterial/archael ribosomes have 16s rRNA.

Note also that cell walls in bacteria are made of peptidoglycan, but cell walls in archae/eukarya contain different proteins. In gram negative bacteria, the peptidoglycan is on the inside, while gram-positive bacteria the peptidoglycan is on the outside! 

To persist then, life must grow and also produce the next generation. We divide cell growth into four parts: G1, S, G2, M. Cells generally start in the G1 phase. G stands for growth, S is replication, and M is mitosis. Mitosis is where chromosomes (clusters of DNA) are duplicated. Humans have 23 pairs of DNA, so 46 pairs per cell after S phase. We call the cells with 23 pairs of DNA as diploid (2n), also called somatic cells. However, in the germ line cells, there are 23 singlets of DNA, called haploid (n). To create haploid cells, diploid cells undergo meiosis, which produces four haploid cells from a single diploid cell. 

The difference between DNA and RNA is Thymine vs Uracil. Cell needs RNA and ribosomes to make proteins, so the simplest cell comprises RNA, ribosomes, and membrane. The structure of RNA (homework problem) comprises a sugar with a nitrogenous base and a phosphate group attached to the sugar. Scientists have actually managed to synthesize RNA from precursors. 

EXTRA CREDIT!!!!!!! CryoEM tour on monday (4/15) at 4PM in Broad 20. 

\section*{April 11 - Origin of life}

Archae have isopropyl chains, ether linkage (more resistant, tougher bond), L-glycerol, and phosphates; while bacteria/eukaryotic cells have fatty acids, ester linkage, D-glycerol, and phosphates. The phospholipids in the bilayer are drawn with a hydrophilic head and a hydrophobic tail. Archae instead have monolayers, with two hydrophilip heads on either end of a hydrophobic center. This promotes rigidity, helping it better survive higher temperatures and pressures. 

Proteins pretty much account for cells' every function, such as motility via moving the flagella, biosynthetic catalysis, communication, etc.

DNA is the genetic material and exists in only one type, while RNA helps transcribe DNA into proteins and has three types, mRNA, tRNA, and rRNA; messenger, transfer, and ribosomal respectively. In bacterial cells, DNA is just supercoiled as one big strand/loop, while human chromosomes account for DNA. We save the rest of the DNA discussion for next week.

We then would like to disccuss small molecules, which usually catalysze important reactions in cells. HEME, for example, is a cofactor in protein synthesis, part of a protein, specifically hemoglobin, which carries oxygen throughout the body. Two more examples are \ce{NADH2}/NAD+ and ATP/ADP. \ce{NADH2} is the primary reducing agent in the cell. ATP is the main energy carrier in the cell, due to the phosphate bond, which we will discuss later.

We then come to the question of whether virii are alive, a crucial distinction being between viruses and virocells. We will not discuss this further; homework question! Gotta love dat Bi1.

Bacteria reproduce asexually. If we focus on the example of \emph{E. Coli}, we see that the process is macroscopically a simple binary division, after making copies of DNA. However, on another example of \emph{Caulobacter crescentus}, the mother cell grows out a budding sister cell. This is called asymmetric division, while the earlier can be inferred to be called symmetric division.

In eukaryotes, reproductive processes generate gametes/haploids, which have half the necessary DNA, and two fusing gametes create the embryo, at which point mitosis takes over, and the embryo grows! We examine these processes more in depth by noting some key differences (the specific process is pretty well known anyways):
\begin{itemize}
	\item Homologous chromosomes mix in meiosis, called crossing over. 
	\item Meiosis produces four unique haploid cells while mitosis produces two identical diploid cells.
\end{itemize}

That's it for differences!! Just kidding, we will revisit this in detail. Anyhow, we now examine diffusion; can cells transport things by diffusion alone? We note the formula $t = \frac{x^2}{D}$ where $x$ is the length of the of the cell (this make no sense -.- diffusion should depend on volume, which is length cubed, oh well), and $D = 10 \mathrm{\mu m^2/s}$. We can then compute for \emph{E. Coli} to find that $t = 0.1s$, the diffusion timescale. In neurons for length 1m, the computation gives $10^{11}$s, which is obviously infeasible. 

We then recall magnetosomes are kept in a straight line by filaments; if we delete the protein gene for these filaments, these magnetosomes start flying all over the place in the cell. Break time! (I have to go somewhere, will miss the second half of lecture; it's about the origin of life). 

\section*{April 16 - Central Dogma}

Information cannot be transfered out of proteins; DNA can transfer to DNA, RNA, or proteins, but once information hits RNA, it goes straight to proteins and then nowhere else. DNA to DNA is replication, DNA to RNA in transcription, RNA to proteins is translation, and then proteins perform functions by folding. Note that in bacteria, the DNA is in the cytoplasm, while in eukarya the DNA is separated, so ttranscription and translation can be coupled (done simultaneously) in bacteria/archae, but in eukarya they must be distinct processes.

Bacteria and archae have 1 or 2 circular chromosomes while eukarya are in linear chromoomes, and often diploid. A small paradox is that a simple 2-replisome solution to replicating bacterial DNA doesn't reproduce \emph{E. Coli} reproduction speeds. It turns out that multiple forks exist in the replication mechanism, so we have a polyreplisome replication mechanism.

In eukaryotes, nucleosomes formed of 8 histones are coiled and stacked to form chromatin fiber, which is then coiled to form chromatids/chromasomes. Important: chromosomes are only so tightly coiled during replication, otherwise, it is fairly unorganized. The exact replication mechanism is still similar.

How was DNA shown to be hereditary material? A pneumonia bacterium with a virulent strain introduced turns them virulent (rough $\to$ smooth), where the strain is introduced by DNA. The smooth version evades the immune system due to the polysaccharide coating that turns it smooth. Interestingly, if the rough and heat-killed smooth bacteria are both introduced, the rough strain will pick up DNA from the dead bacteria and become resistant! This shows that DNA is the hereditary material. 

All nucleotides have a phosphate group, a 5-C sugar, base. On the 2' position on the sugar, -H is DNA, -OH is RNA. Guanine pairs with Cytosine, Adenine pairs with Thymine/Uracil (G-C, A-T/U). Pairings are hydrogen bonds. Note that G-C has three bonds, A-T/U have 2 bonds, so G-C is a stronger bond. 

DNA is synthesized from 5' to 3' (the two carbons that are exposed). New nucleotides are added to the 3' end. Therefore, when replication forks occur, the DNA polymerase synthesizes one strand towards the fork and one away from the fork (the latter occurs in small segments). Note that the polymerase reads from 3' to 5' but synthesizes 5' to 3'. Helicase separates the strands. The 3' strain is the leading strain, and is continuously synthesized. The 5' strain is the lagging strain, which is synthesized in Okazaki fragments and thus backwards.

Transcription is done by RNA polymerase, which transcribes DNA to mRNA. RNA polymerase takes a leading strand and synthesizes an mRNA strand off of it. mRNA moves to the proteins, but there are three types of RNA: m, t, r, and Regulatory RNA (affects whether DNA is transcribed; exception to central dogma, we won't discuss much for now). Operons are a cluster of genes under a single promoter/regulation factor. There are promoter strains, which is where the RNAP (polymerase) begins transcribing. There are then transcription factors, which can work either proximally or at a distance to activate or repress transcription. This is the bacterial model; the eukarya model is more complex.

Eukarya also have the promoter codon, but there is an untranslated region (UTR) that controls whether proteins are made (5' and 3' UTR on either end). There is a start codon after the 5' UTR whch tells the ribosome to begin and a stop codon right before the 3' UTR which tells the ribosome to stop. In bacteria, a $\sigma$ factor controls situational recognition of promoters (i.e. if it's really hot, then maybe a $\sigma$ factor will tell a certain protein to transcribe). In pre-RNA (right after transcription), a 5' cap and a polyA tail are added (only in eukarya!) to increase longevity. The rest of the RNA has alternating introns and exons, where only exons are ultimately spliced (in many ways, sometimes not involving all exons) in what is called \emph{alternative splicing}. This is most prevelant in eukarya. 

We now discuss translation (RNA to protein), which occurs in ribosomes (three domains have different ribosomes). Ribosomes have a large and small subunit on either side of the mRNA (which is translated from 5' to 3' as always). Small subunit has 16s/18s rRNA (in bacteria/archae vs eukarya respectively). Small subunit selects tRNA, which allows the large subunit to select the correct amino acid based on a sequence of three codons and makes peptide bonds between the amino acids to create a peptide chain/protein (Note 3 base pairs = 1 codon, so 9 base pairs = 1 amino acid). The amino aids are attached to their respective sequences of tRNA (which has anticodons to match codons on the mRNA), and as the tRNA matches the mRNA, the large subunit creates the chain. Three sites: A site, P site, E site. tRNA enters at A site and is tested for codon-anticodon match. If there is a match, the tRNA is shifted to the P site, where the amino acid is added to the chain. The spent tRNA then moves to the E site to be ejected.

There are four main ways to represent proteins. 1) Primary sequence - sequence of amino acids. 2) Secondary structure - alpha helix. 3) Tertiary structure - monomer, complete domain of protein. 4) Quartenary - multimers!

We now can compare the three domains in transcription:

\begin{table}
	\centering
	\begin{tabular}{|c|c|c|c|}
		\hline
		Transcription & cytosolic & nuclear & nuclear\\\hline
		&1RNAP & multiple  & multiple \\\hline
		Initiation & $\sigma$ factor & $\sigma$& other \\\hline
		&few introns & alternative splicing & AS \\\hline
		&no mRNA modicitation (5' head, 3' tail) & yes & yes\\\hline
		Translation & translation / transcription coupled & decoupled & decoupled\\\hline
		&short-lived mRNA & mixture of machinery& long \\\hline
	\end{tabular}
	\caption{Screwed up middle column (archae), please read backwards}
\end{table}



\section*{April 17 - Life and Nucleic Acids, Central Dogma}

Central dogma is DNA $\to$ RNA $\to$ proteins. Nucleotides are the monomer for DNA, made of a nitrogenous base, sugar, and phosphate group. The sugar is called ribose. We prefer DNA to RNA because DNA has 521 year half life. Another difference between DNA vs RNA is that DNA has thymine while RNA has uracil. DNA turns once every 10 base pairs, 34 \AA. A pairs with T, C pairs with G. A and G are purines, T and C are pyramidines. DNa replication has unwinding done by \emph{helicase}. DNA Polymerase attaches to the 3' end of the split and goes up towards the 5'; this is called the leading strand. The lagging strand is synthesized in fragments (Okazaki fragments!!) and DNA ligase attaches the Okazaki fragments.  

We then discuss transcription. First, DNA is unwound by helicase. RNA polymerase then attaches to the strand at a promoter region and stops at the stop signal. The product is mRNA. In bacteria, there is an operator sequence that has an attached repressor that only lets go when the time is right. This is called the operon mechanism. Transcription happens in the nucleus for eukaryotes but happens in the cytosol for bacteria (because no nuclei are present). Also, eukaryotes use alternative splicing to remove introns. A consquence of intron/exon dichotomy is to easily generate diversity, though exactly why introns and exons evolved is difficult to answer. 

We then move onto translation. Twenty amino acids corresponding to 64 possible tricodon sequences. We call one DNA strand the \emph{coding sequence}, and the coding sequence is defined such that the mRNA matches up with the coding sequence (i.e. the mRNA is replicated off the complement of the coding sequence). 

We can then discuss mutations. If we have an insertion (frame shift) mutation, then all subsequent codons are affected. A substitution (point) would only change a single amino acid, so it's not nearly as large an affect as an insertion. Transcription occurs in nucleus of eukarya and in cytosol of bacteria/archae. Translation occurs in the cytosol for all domains. 
\section*{April 18 - Mutations}

How do we test mutations? We use virii! We can use assays to test (assays are petri dishes?? I have no idea. I think that it's basically a petri dish with nutrients already on there). A bacterium would be infected with a phage/virus (same). The process is called lytic, because it kills the host cell. The virus infects the bacterium and creates copies and then the bacteriumbursts. An alternative process of infection is when a virus infects as a lysogen, where the viral DNA infects the DNA of the bacteria such that the bacteria takes up viral DNA.

The way that this tests Darwinian principles is by measuring mutation frequency. We start with a plate seeded with many bacteriophages (virii) such that the plate looks clear. But then at some point, bacteria will mutate a virus-immune gene, and then a colony will emerge. So first the assay is covered with bacteria and then virii. The question is then whether the mutation is a selected genotype or whether it is a adaptive phenotype change (? I don't think I got this right), i.e. Lamarckian vs Darwin. 

The test to answer this question is called the fluctuation test. There are two setups, one with a single flask of bacteria and another setup with ten flasks of bacteria. Then ten plates of of assays were made from both setups, all ten from one flask in the first setup and one from each flask in the second setup. All twenty plates were then saturated with phages, and the resulting immunities were analyzed for resistances. If all plates show identical resistances, then Lamarck is most likely correct, because all bacteria will adapt to the phage equally. However, if Darwin is correct, then there is a likelihood that one of the ten flasks in the second setup already has the phage-resistant mutation, and so one of the plates will show very bacteria-filled.

There are two types of mutations, mismatching a single nucleotide/amino acid (point) or missing a nucleotide/codon/amino acid entirely (frameshift). Two types of frameshift mutations, partial (completely screws up amino acids b/c number of deleted nucleotides is indivisible by 3) and clean (erases a whole amino acid or gene). These can happen at one of three stages: DNA Polymerase, RNAP, or the ribosome. Deletions can occur on the order of one nucleotide or even thousands. Differences in GC content suggest transposons, which arise from horizontal gene transfer (more on this later, thank god .\_.).

An example of a mutation-generated disease is sickle cell anemia, and 2/3 of cases are attributed to the substitution of a single A nucleotide for a T. Another instance is cystic fibrosis, of which cases 2/3 are attributed to the deletion of a single residue (gene?). 

Mutations can arise from vertical inheiritance (mutations from the parent DNA) such as by DNAP, RNAP, ribosomes, or environmental stresses (mutagens). However, a very important category is horizontal gene transfer (rampant in bacteria/archae). One example of this is transfection, such as lysogeny, which is DNA integration. A second example (out of four) is transformation, which is DNA uptake from through the membrane. Transposons are a third example, jumping genes that jump within chromosomes or from an external source. The last class is conjugation by plasmids, exchange of extrachromosomal DNA. 

Conjugation by plasmids occurs through a channel called the pilus. It replicates and goes through to the recipient cell. This mechanism is key to antibiotic resistance. One example, vancomycin resistance, came from the conjugation of a transposon from a different bacterium. Transposons have the ability to hop at random and also to hop selectively. 

Molecular biology enables very dense storage in DNA, but it also allows DNA to be programmed and sequenced, so nanomaterials, diagnosis of genetic diseases, etc. SO MANY THINGS POSSIBLE. DNA sequencing has grown heavily due to the rapid evolution of technology. Automated DNA sequencer and protein sequencing were both developed at Caltech, pivotal in the human genome sequencing project. Costs are exponentially falling. Polymerase Chain Reactions (PCR) allows certain genes to be amplified and recombined, where amplified is meant to create multiple copies.

Initial pioneering of PCR was done using DNAP from \emph{Thermus aquaticus}, which is much more thermally resistant than \emph{E. coli}, a big boon as PCR requires temperatures approaching $100^\circ$. 
\section*{April 23 - Mitochondria/Multicelluarity}

Guest lecturer! The Planctomycete is suspected to be the Last Eukaryotic Common Ancestor (LECA), as it is a bacterium with a rudimentary nucleus. But in any case, an interesting question arises: why don't eukaryotic traits evolve in prokaryotes since the first eukaryote? Eyes evolved independently in many many organisms, why not sex, nuclei, phagocytosis, etc. in prokaryotes?

The LECA had all of these traits, sex, nucleus, phagocytosis, and significantly more. There was a sort of event horizon: as soon as all eukaryotic traits were mutated, eukaryotic diversity exploded. Moreover, there have been no discovered intermediates between bacterial cells and LECA. Why? This is the core of the scandal.

A possible solution (substantiated by preliminary evidence) is that the ``tree of life'' is actually a ``ring of life'', with bacteria and archae merging back to create the LECA, which would explain the rarity. The process is hypothesized to be an endosymbiotic merging. Note that natural selection does not intrinsically give rise to complexity, only to optimization. Commence huge argument number 2 (first one was with the eukaryote/prokaryote distinction). 

Endosymbiosis seems rare in prokaryotes, one notable example being cyanobacteria. Eukaryotes have this very frequently due to phagocytosis. So why were mitochondrion endosymbiosed? Many hypotheses are dispelled, but conclusion is that mitochondrion do still increase metabolic rate per DNA (after examining per mass and per cell), and so endosymbiosis helps the cells grow larger and grow more complex. 

\section*{April 24 - Multicellularity}

Multicellularity has a few criteria to increase fitness: multiple cells of the same species, living in close proximity, differentiated structure/function. The differentiation arises because different DNA is expressed. 

We then wish to seek how genes are expressed. It turns out that certain genes control where and when other genes are expressed, as fruit flies (it's always fruit flies), were tested on with \emph{Hox} mutations. Chemical signals are also a controlling factor. Ectopic expression is another method to express genes where it should not be expressed, such as adding extra eyes to fruit flies (everywhere!).

In multicellular organisms, senses give the multicellular organism the sufficient information to integrate into its future behavior. Chemotaxis is an example of this behavior. 


\section*{April 25 - Multicellulality!}

Multicellular organisms occur in all three domains! They differentiate and exhibit division of labor, and they have the mechanisms cell aggregation and lack of separation upon division. The difference between multicellularity between domains is primarily exhibited in the difference of complexity, where bacteria/archae have 2-3 cell types, fungi have 7-9, while plants/animals have 10-100 types. Animals therefore have extensively developed organ systems that are absent in smaller life forms. 

The consequences of having such developed organ systems is that the functions performed in bacteria that take single molceules will take a whole organ system in more complex organisms such as humans.

The definition of multicellularity, however, is difficult to define. First, we define a population to be a group of interbreeding individuals. However, when we have two populations who live in close symbiosis, such as bacterial biofilms (which comprise many types of bacteria), is this a multicellular organism or just a community? The conclusion differs depending on whether microbiologists are consulted or biologists; the former consider this a multicellular organism (due to differentiation of function, etc.) while the latter would not (trees + humans = symbiotic, but wtf multicellularity). 

We then examine the variation in complexity of animals, before which we have examined very much that has made the note taker terrifically bored and unwilling to take notes. We note that the simplest animals, such as sponges, have two cell layers (germ layers), which then evolve to have three cell layers. There are endogenous and exogenous cell determination mechanisms. One two, skip a few\dots

A fertilized egg will undergo 41 rounds of mitosis to create $\scinot{2.2}{12}$ cells worth of an embryo. We then would like to examine this developmental process. \emph{Caenorhabditis elegans} has exactly 959 cells in all adults! So we can examine the exact process of diferentiation and cell fate. Specifically how is beyond the focus abilities of the present listener; a physics set has called throughout the entirety of lecture! But it is finished now, and professor is having trouble playing a video of an interview.

The Hox genes were discovered in fruit flies that regulate the orientation of the animal. There is a homeodomain of the Hox protein that correlates to a transcription factor. These exact genes were also found in mice! So obviously this is a common evolutionary gene that determines in all vertebrates the front and back. I have no idea what's going on anymore, I don't know where this lecture is going. Maybe more attention should have been paid. Maybe the lecturers should have given a better outline. Maybe they did. In any case, regardless of the \emph{Yin}, the \emph{Guo} is that I stand lost. 

Take home messages! Multicellularity arose many times, but most pronounced in animals/plants. In animals/plants, creates new levels of life. Development is key to multicelluarity, results in shared characteristics among animals/plants and also results in diversity of body plans! What is development? Probably genetic evolution. Look into this if you ever decide to reread these lecture notes.

\section*{April 29 - REVIEW SESSION!!!1}

Note that last year's problems often were pretty open ended; this year will be much more explicit. Keep this in mind when studying!

Format of the test is on ``advice'' under moodle: in-class, closed-note except a single notecard. First type of question is essay question (synthesize multiple parts of course, open ended). Second part looks at components of reserach paper and interpretations thereof. Third type is short answer, not multiple choice. Intended to have plenty of time to check answers within all given 90 minutes. Will have very little super-specific information (tests analysis, not knowledge; no more of ``What did Professor Abulafiakagija the 57 do?''). 

Questions on the required reading will try to check general understanding rather than straight-up memory of details (pffffft yeah right). 

Will not have anything quantitative except interpretation of results. So we now begin review!!!

Central Dogma: in the chain DNA - RNA - protein, information never flows backwards. DNA to RNA is transcription, RNA to protein is translation. DNA to DNA is called replication. RNA can be used for transfer from nucleus (mRNA), translation (tRNA - attached to amino acid, matches with codon, amino acids chain to proteins), and in ribosomes (rRNA - part of the ribosome, 3-D structure for ribosomal complex involves rRNA for catalysis), and more (though we only have these three) that are not involved in the protein coding process. Proteins are used for structure and catalysis (enzymes).

How then do the aspects of the structures of these molecules contribute to their ability to perform the roles listed? DNA is double stranded, so is both more stable and is a second-copy so we can check for mistakes. While DNA is stupidly long, it can be sequenced and programs know how to process this sequence and find genes. RNA doesn't need to be as precise, so net energy-save by not having to continually unzip and rezip. RNA also folds on itself via hydrogen bonds (just like in DNA double helix). Proteins fold and other stuff to contribute to the ability to hold up the cell structure. There are many more amino acids and therefore can pull off a wider variety of functions (observed in proteins). How exactly proteins folds is still totally up for question.

We now discuss cell biology. Let's list some things about cells:

\begin{itemize}
	\item Common features: membrane, DNA, ribosomes
	\item Eukaryotes: nucleus, organelles (mitochondria, chloroplasts, endoplasmic reticulum, Golgi body), linear DNA, cellulose cell walls in plant cells, cytoskeletons (B/A have few/smaller)
	\item Bacteria vs. Archaea: circular bacterial DNA, mostly circular archaeal DNA; peptidoglycan bacterial cell walls in both; Bacteria have ester linkages and Archaea have ether linkages in cell walls; gram-$\pm$ is whether peptidoglycan is on surface: if dye can get to peptidoglycan, then single membrane (positive), else double (negative). 
\end{itemize}

Then, what distinguishing features give rise to abilities to live in what habitats? Ether linkages are stronger so archaea live in more extreme conditions. 

We then discuss evolution! Four major aspects (and then a few):

\begin{itemize}
	\item Darwin - Existing variation \& mutations give natural selection. More fit individuals leave more progeny. Works on population-level.q
	\item Lamarckian - Acquired traits are passed down. Darwin is usually correct, but Lamarck is correct in the case of horizontal gene transfer.
	\item Gradual Change - mutations occur at a constant, gradual rate.
	\item Punctuated Equilibrium - mutations occur in spurts in conjunction with catastrophic events (a lot of the time)
	\item Protocell - Protocells are the initial cells (prototype cells :D), so must need membrane, genetic material (likely RNA b/c can be enzyme). \emph{Abiogenesis} is the genesis of life from nonliving life forms, but is still not determined because the early environment still isn't known for sure. Example: phospholipids in close proximity spontaneously form bilayers. 
	\item Mutations - mechanism for evolution; insertion/deletion (frame shift), transposition (special case, dwai; basically an insertion), point mutation (change at one point). 
	\item Key Innovation - Something drastic that evolves in a group that causes the whole group to radiate out (nucleus, mitochondria, backbone, etc.) as a new group
\end{itemize}

Question time: How could DNA mutation be used to experimentally answer questions/hypotheses. Answers generally fall into ``If we take something out, what goes wrong?'' and ``If we force a new gene to be expressed, what can happen?'' We can also mutate proteins and see how function is affected. We can also tag proteins and track the protein; we tag at the DNA level.

Next discussion is multicellularity. Define multicellularity: differentiation (division of labor), same species cells living in close association/proximity. Increases fitness by specialization treating tasks better. Origin mechanisms: mutation that prevents successful mitosis (doesn't fully split), and aggregation (cells randomly start to give out free hugs). One notable example is choanoflagellates, which make rosettes in the presence of certain bacteria, suggesting at a origin of multicellularity. Note that multicellularity evolved in all three domains of life! In multicellular bacteria/archaea/some eukarya only have 2-3 different cell types, plants/fungi have about 7-9, and animals have 10-hundreds. 

Question time~ Why does multicellularity evolve so much more complex in eukarya than in B/A? Bigger genome length gives more functions that can be encoded in the DNA. Also, alternative splicing gives wider range of functionality that can be encoded from the same genome length, further increasing the range of complexity that eukaryotes can attain. Eukaryotes are also more energy efficient and so can be more complex for the same energy input (efficiency is due to organelle specialization). Also, Eukaryotes have DNA in the nucleus, so more specificity because segments are more easily separated when translating. Lastly, cytoskeleton allows cells to be larger (better transport system) so you can attain greater complexity/size.

In eukarya, transcription: DNAP binds to a transcription factor, which begins to transcribe at the promoter region, then there is a 5' untranslated region, then there is the start codon ATG (? I may have copied that wrong) which opens all mRNA sequences, then there are alternated introns/exons, then there is stop codon and 3' untranslated regions. Note that both untranslated regions are also transcribed, but just not translated; they are the start/stop codons for ribosomes. Then, after the DNA is faithfully translated to a pre-mRNA, the introns are removed in the actual mRNA. This then becomes a protein! Surprise! We also review the replication mechanisms, specifically how the 5' and 3' work: read 3' to 5', construct 5' to 3'. Note that the ``coding strand'' is actually not the one that the ligase reads; the complement is read, so the mRNA is the same as the coding strand. The promoter region defines which strand is to be the coding strand. 

Interesting question: primer vs. promoter? Promoter is in the DNA sequence, while primers are kind of a ``starting'' group upon which to synthesize; it is the initial bit of DNA to start the replication process. It is most often used in labs to do PCR, though it also occurs in nature. 

In prokarya, the transcription begins at the sigma factor, then goes to promoter region, then hits the 5' UTR, then genes, then 3' UTR. The mRNA is then missing the promotor regions/factors, but still has the UTRs (untranslated region), and the proteins then lose this part too. (I may have missed the order on $\sigma$ vs promoter, but I think I'm right). There are no start/stop codons (is that right? it's not drawn). 

Note that during transcription, nucleic acids are worked linearly, but translation can be done in parallel with multiple ribosomes. Note also that this whole thing (from promoter to end) is called an operon.

We lastly discuss transposons (jumping genes), for which the only goal is to jump around in the genome (it will also self-replicate within the genome). They can cause frame shifts and stuff, and is usually detrimental to genetic code. Their function is a ``selfish genetic code'' that has few benefits; they are usually silenced through various mechanisms in replication. They rarely have a good function, just mobile. 

We also need to discuss cell cycle, but that can be discussed later; study this on your own! Ribozyme (sp?) is an RNA that is also an enzyme.

\section*{April 30 - Metabolism/Review lecture}

The two key molecules are ATP, NADH. ATP has high energy bonds, while NADH has free protons and electrons (which are reducing and oxidizing agents; redox chemistry). Energy comes from the digestive system (delivers organic C, i.e. fuel), and the circulatory system distributes oxygen for combustion. These are then processed in mitochondrion via cellular respiration. In the cytosol, sugars (glucose) are broken down into pyrovate via the process called glycolysis, which produces some ATP (2). The pyruvate then goes to the mitochondria, which then joins with oxygen to be combusted. 

Then, bacteria only want to divide and survive, so exponential growth is expected, but of course, it doesn't work that way; it will plateau ta some point, called the stationary phase. Metabolism comprises two parts, catabolism and anabolism. Catabolism brings substrates to products, so it generates energy. This energy is then used to fuel anabolism, which turns monomers into biomolecules, namely biosynthesis.

We then discuss how to grow bacteria. The composition of the medium in which we grow bacteria is directly related to which bacteria grow, so we can select for bacteria like this. Only 1\% of bacteria grow in cultured plates, though; why so low? First, poisoning can occur. Secondly, certain bacteria will eat faster, ousting the slow eaters (dilution/spatial separation solves). Thirdly, a better characterization of the genome can tell what conditions are more favorable. Lastly, symbiosis/interactions can help improve survivability.

We can then discuss what important things would need to be placed into the growth medium. CHNOPS (6 elements of periodic table) in Redfield ratio (C:N:P = 100:16:1, on average; varies by factors of up to 10x). Salts (Na, K, Ca) are required for homeostasis, transport (determines membrane potential). Some cells need trace elements (i.e. HEME (hemoglobin constituent) requires iron). We also need vitamins (such as vitamin B: gut bacteria). Lastly, other growth factors are required as a miscellaneous group: quorum sensing (senses critical density to do something in common), hormones, signalling. We then discuss some examples, which I will not record. 

There are broad classes of metabolisms, anabolism (consumes carbon) and catabolism (produces carbon). Under anabolism, autotrophs use inorganic carbon (\ce{CO2}) to produce energy while heterotrophs use organic carbon. Under catabolism, phototrophs use light to perform chemistry while chemotrophs use chemicals. There is also the dichotomy of organotrophs and lithotrophs (within chemotrophs; are subsets), which use organic and inorganic fuel respectively. We then discuss redox review, which is pretty straightforward; know oxidation numbers and definitions. 

Final exam review! We are posed a data set and required both to write a hypothesis to explain the data and an experiment to test the hypothesis. Key is to articulate assumptions, logic, clarity, specificity of hypothesis. 
\section*{May 1 - Metabolism }

Anabolism builds, Catabolism destroys. ABCD, just like kindergarten. Obviously catabolism requires some sort of substrate/food, which is then catabolized to products in order to produce energy. These products can then be anabolized (requires energy) back to the substrate state (i.e. proteins!). We now discuss the various types of trophs in a freaking awesome table. Ogle:
\begin{table}[!h]
	\centering
	\begin{tabular}{|l|l|l|l|}
		\hline
		Type of -troph & Energy & Carbon source & Example\\\hline
		Photoautotroph & light & \ce{CO2} & Plants, Purple/green sulfur bacteria \\\hline
		Chemoautotroph & breaking chemical bonds & \ce{CO2} & H,S,Fe,N - using bacteria\\\hline
		Chemoheterotroph & chemical bonds & Organic compounds & Meeeeeeee\\\hline
		Photoheterotroph & light & organic compounds & Purple/green non-sulfur bacteria\\\hline
	\end{tabular}
	\caption{Have fun!}
\end{table}

We then note the requirements for a cell-growing medium: carbon source, energy source, cofactors. Metal salts are in the cofactor list. 

\section*{May 7 - Metabolism again!}

There is a wide variety of metabolic pathways. What is critical to know about these is that they are interconnected, that they share a common pathway (ATP, NAD(P)H), and that understanding produces engineering. 

Critically, there are a few components in a metabolic pathway. First, there must be a mechanism of transport across a membrane, there must be energy, and there must be biosynthesis (catabolism/anabolism respectively). The energy generated by the catabolism is then often consumed in the anabolic step. Today, we will examine NADH/NAD+ redox cycle in catabolism and the NADPH/NADP+ in anabolism (the involvement of redox in metabolism). We will also learn today about the Proton-Motive Force (PMF), which is due to the proton gradient $\Delta P$. 

We note that ATP hydrolysis is exergonic (releases energy) via the reaction \ce{ATP + H2O -> ADP + Pi} where Pi is inorganic phosphorous (specifically \ce{PO3^{3-}}). There are two ways to make ATP, which we will discuss in turn.

First is substrate-level phosphorylation (Fermentation) - First, a carbon substrate A is oxidized by NAD+ to B, which is phosphorylated to B-P, which then gives ATP after going through an enzyme that removes the phosphate group, producing some C. C is then rearranged to D, which is reduced to E, regaining the NAD+. So there are three parts, oxidation, rearrangement, reduction. Ex: Lactic Acid fermentation. Glucolysis accomplishes both the rearrangement step and oxidation step (producing 2 ATP), and the pyruvate produced from glucolysis is reduced to lactate in the final step, which does not produce ATP. Another example is ethanolic fermentation:
\begin{itemize}
	\item Glucose + 2NAD+ + 2ADP + 2Pi -> 2 pyruvate + 2NADH + 2H+ + 2ATP
	\item 2 pyruvate -> 2 acetalydehyde + 2CO2
	\item 2 acetaldehyde + 2NADH + 2H+ -> 2 ethanol + 2NAD+
\end{itemize}

which comprise the three parts oxidation, rearrangement, and reduction respectively.

Second is oxidative phosphorylation (OxPhos), which is basically a cellular battery. It comprises two parts, a coupled electron transfer along with proton translocation, though the exact mechanism I managed to miss through all my capacities of paying attention. In any case, this produces a battery of voltage $\Delta P$, the aforementioned Proton-Motive Force. This is defined as:

$$\Delta p = \Delta \phi + \frac{RT\ln\left[ \ce{H+} \right]}{F\ln\left[ H \right]}$$

where $\phi$ is the existing voltage, and $F$ is Faraday's constant $96485$ C. We note that this is pretty straightforward electrochem, yay Ch3x for teaching us this (reviewing this, yay AP Chem for teaching this (reviewing this, yay Yubo for having learned it prior on his own :D)).

We will now discuss electron transport chains. In an ETC, the basic principle is that ATP's dephosphorolyzation pushes protons out of the cell membrane. Newman then proceeds to go crazy drawing electrons flying all over the membrane. We do learn amidst it all that ``oxidoreductases'' are large protein complexes in the membrane that are capable of doing redox reactions. The idea seems to be that an electron donor arrives at the ETC, is oxidized, and reduces an electron acceptor, which then maybe reacts with ATP (??) to vomit protons out the top of the oxidoreductase. I suspect Newman is just going crazy again with details we won't be test with, but only time will tell\dots

I think that I figured out where we are (went distracted mode again). We discuss what the various electron donors are for various organisms. In heterotrophs, NADH is the electron donor (i.e. Krebs/Citric Acid cycle). In lithotrophs, FeS is used, sometimes ferric sometimes ferrous (+3/+2 respectively). 

And then the punch line, it's MORE INTERESTING IN BACTERIA. I don't think anybody would misguess who is lecturing today. Aerobic goes: AH2 -> dehydrogenase -> quinone -> fancy stuff, while anaerobic respiration goes: AH2 -> dehydrogenase -> (mena)quinone -> reductase. Note the difference of the presence of reductase. We can then determine the exact favorableness of these reactions. 

Summary time! Skipping the first half because I understand. Metabolism is highly integrated, catabolism/anabolism is balanced, ATP/redox carriers are key in metabolism, two ways to make ATP (discussed above, used by fermentations and respirations respectively), OxPhos is modular and uses PMF, and thermodynamics underpins these processes.
\section*{May 8 - Metabolic processes}

We first define reduction potential. We then practice some problems regarding the reduction potential, specifically finding the potentials for certain reactions. 

We then recall from lecture that there are two pathways to make ATP, substrate-level phosphorylation and oxidative phosphorylation. We first discuss substrate-level phosphorylation, i.e. fermentation. The steps for glucose are:
\begin{itemize}
	\item 1 Glucose is oxidized to pyruvate via NAD+/NADH couple.
	\item Pyruvate is then reduced to lactate via NADH/NAD+ couple.
\end{itemize}

So the end result is that NAD+/NADH are conserved, while ATP is generated during the oxidation of glucose. 

We now discuss the electron transport chain, in a way that's comprehensible, unlike in lecture! The ETC is coupled with ATP synthase over a membrane. We thus need an electron donor transferring electrons to electron carriers, during which protons are pumped against a concentration gradient. The concentration gradient then fuels ATP generation in the ATP synthase, which phosphorylates ADP to ATP. 

We then ask what the donors and acceptors are. In eucarya, the donor is often NADH. Arsenic (???) is the acceptor in many eucarya.

We then discuss photosynthesis. Sunlight peels an electron off As(III) to to move it into the reaction center, where it is accepted by NADP+ to form NADPH. The NADPH is then moved to the Calvin cycle, which is carbon fixation, and eventually a hydrogen concentration gradient generates ATP. 

Cytochrome is a heme group that allows electron transport by being an elecrton donor for iron (this is on term sheet). Another key term is PMF, the potential energy due to the hydrogen concentration gradient, which can be used for nutrient transport, motion, and others. A typical electric field generated by the PMF is $10^7$ N/C, which is pretty scary! A few other less scary numbers are mentioned which we don't take note.

Finally, we discuss ATP synthase. But before we do, we are going to summarize, because we won't have enough time to discuss ATP synthase anyways. We will draw a handy dandy table!
\begin{table}[!h]
	\centering
	\begin{tabular}{|c|c|c|c|c|c|c|}
		\hline
		Process & Where? & Bacteria & Archaea & Plants & Animals & Other Euk\\\hline
		Respiration & Membrane & Lots! & Various & Yes (Oxygen only) & Yes (Oxygen only) & Various\\\hline
		Photosynthesis & Membrane & Some & No & Duh & No & Some\\\hline
		Fermentation & cytosol & Yes & Yes & Yes & Yes & Yes\\\hline
	\end{tabular}
	\caption{Tables tables}
\end{table}

It seems that everything ferments, while respiration is pretty selective, though current evidence seems to suggest that respiration came first! Which is very counterintuitive.

\section*{May 9 - Photosynthesis}

Half the world's \ce{O2} is produced by photosynthesis. There are two classes of phototrophs, oxygenic and anoxygenic. Oxygenic is defined as ``producing oxygen'', which is namely the process \ce{CO2 + H2O -> C6H12O6 + O2}. Anoxygenic is defined as ``not producing oxygen'', which is the more primative process, and as per Newman is beautiful as always. Examples of the proton donor are \ce{Fe^{2+}, S, H2, As(III), Organic}.

If we look at the structure of chloroplasts, it is simply an ancient cyanobacterium. We then ask what light does for electron transfer? The answer is that the light excites the electrons such that the reduction potential is lowered and the electron is more easily donated. It is also easy to note that the photosynthetic reaction centers in bacteria are very similar to the chloroplasts, specifically PSII evolves from purple bacteria, and PSI evolved from green sulfer [sic - Newman] bacteria.  

The general photosynthesis reaction is then \ce{AH2 + CO2 -> CH2O + A_{ox}}. The \ce{CH2O} glucose is the electron acceptor, the \ce{AH2} is the electron donor, and the transformation of the carbon is called carbon fixation. The specific pathways are very diverse due to the freedom in A. We recall that the energy gain of an electrochemical reaction is $\Delta U = q\Delta V$. This is called the Gibbs free energy, where a negative Gibbs free energy gives favorable reactions and a positive gives unfavorable. 

Oops. Found an interesting link on Facebook. Anyways, we are discussing the Anaerobic Oxidation of Methane (AOM). This has applications to cleaning up the oil spill, and we will discuss the specific process after this momentary break!! :D

We write out the net equation for AOM: \ce{CH4 + SO4^{2-} + H+ -> HCO3- + H2S + H2O}. We then note that the actual equation comprises two parts, ANME-1 and SRB (no idea what they stand for). The donor/acceptor couple for ANME is \ce{CH4/H2O} while for SRB the couple is \ce{H2/SO4^2-}. We then examine $\Delta G$ for this reaction:

$$\Delta G = \Delta G_0 + 2.3RT\log\left( \frac{\ce{[HCO3-][H+][H2]^4}}{\ce{[CH4]}} \right)$$

We note that this is the ANME (ANaerobic MEthane oxidizer) half reaction, and though $\Delta G_0 > 0$, the concentration of \ce{H+} is so low that the net $\Delta G < 0$. Then, since SRB has $\Delta G_0, \Delta G < 0$, we find the net AOM reaction to be spontaneous/exergonic. 

We can also link $\Delta G$ to $\Delta p$, the proton-motive force. We ask how much $\Delta p$ is recquired to synthesize ATP. Let's say that $\Delta G$ to make ATP is $-50 kJ$. Note that $y = 3$ is the number of protons required per ATP, and so we can simply work with $\Delta G = yF\Delta p$, giving $\Delta p = -174$ mV. We then ask the minimum potential difference required, which is $\Delta G = -nF\Delta E$ (Not quite sure what the difference between y,n is), which gives us the equation $-nF\Delta E = yF\Delta p$.

We then ask what are limiting factors for metabolism: \ce{H2O} (or whatever medium), availability of redoxidants (kinetics/toxicity), physical conditions, chemical conditions. 

Real world applications a.k.a. Diane Newman's Desperate attempt at keeping our attention. Forget it I'm refusing to write this down.

\section*{May 14 - Bacterial Populations}

Bioluminescence helps accomplish many things: avoid predators (distraction, counterillumination [belly-lit squid]), attract prey, interspecies communication. Bioluminescence sometimes arises from autogenic causes (within genome) or symbiotic causes. In fact, some bacteria invest up to 10\% of their total energy into making light. This shows that light production must be controlled, otherwise it will drain energy faster than Ubuntu.

Experiments show that some cells indeed only luminesce at a certain number of cells. Crap. That was a little too long without jotting things down. Moving on\dots

There is a difference between signal vs. queue. The former is directed while the latter is just a general cry-out. Quorum signaling is used to control specific genes, particularly studied in Gram-proteobacteria (including luminescence! and biofilms). Biofilm development requires quorum sensing; experiments show just like luminescence that if quorum sensing is disrupted so is biofilm formation. 

Quorum sensing is embedded in complex sensory pathways. Some signals are specific while others are shared. In conclusion, quorum sensing is the commuinty communication mechanism.

Bacteria have the same core genes among species, but have very different dispensable genes, resulting in different kinds and numbers of genes in its genetic code even among species. This leads to intra-species conflict over resources, as pathogenic strands are selected for while benign strands are selected against, which is a poor situation for the host. The host then attempts to solve this by separating the bacteria strands to decrease intra-species competition. This can be done with an anatomical bottleneck, which is actually increased in efficiency after bacteria first leak through, which decreases intra-species competition.
\section*{May 15 - Biodiversity}

Populations are the number of organisms in the same species in an environment. Carrying capacity is the maximum population that an environment can support indefinitely. Population density is straightforward; high densities favor organisms that can survive with few resources, while low densities favor organisms that reproduce rapidly.

There are density dependent and density independent factors that will contribute to the population size. Density dependent factors include resource abundance, overcrowding, resting sites. Density independent factors include natural disasters, predation, disease.

Communities are groups of populations in a given area such that the populations interact. The diversity is then defined as the relative abundance of different organisms in a community. Diversity factors include resources, disease, predation.

Three types of population distributions, clumped, uniform, random. No pictures will be necessary to depict, because you're not dumb.

There are three primary ways that populations in a community interact: competition, predation, and symbiosis. Results of interaction: Competitive exclusion, resource partitioning, character displacement.

\section*{May 16 - Macroscopic organisms???}

Saltwater dominates earth while freshwater is a patina. At the same time, the biomass ratio of water:land is 900:1. This is supposed to be awe-inspiring, so let's all ``wow'' on three! Oxygen hasn't been constant in the past either, and it is suspected that an increase in oxygen corresponded to the transition to land, though it is a nebulous theory at best. It is known for sure that plants made it to land before animals, and it is hypothesized that animals moved to land at around 18\% atmospheric oxygen.

The initial plants made it on land by a symbiosis between plants and root fungus (called mycorrhizae) which enabled the water-land transition by mobilizing phosphorous (making it available to the plant). The Proterozoic era denotes the part of history where everything is still in the water, the Paleozoic denotes the first part of the transition, the Mesozoic denotes the era of the O2 dip (explained below) and the Cenozoic is the most recent era since the last O2 peak. After animals made it onto and, there were two large spikes in atmospheric O2, sandwiching a steep dip in O2. One might expect gigantism to come up during these large spikes, and indeed, insect gigantism was obsered during the first spike, while dinos were observed during the latter.

Some obstacles faced by terrestrial organisms are support/dessication (structural/form adaptations) and temperature control/respiration/reproduction (physiological/function adaptations). We first examine the structural stresses that biolifeforms experience: compression, tension, bending, shear, torsion. There thus result large differences in locomotion and feeding.

After a break where many a student didn't pee fast enough to answer the clicker quiz, we discuss species interactions. We begin with competition: first principle is competitve exclusion, the belief that two competitive organisms that occupy the same niche cannot coexist.

Abiotic features are crucial in determining diversity, such as temperature, etc. But equally important are biotic features, such as \textbf{keystone species}, species that have a disproportionately large effect on the environment relative to its abundance.Another factor that is key is \textbf{Allelopathy}, the production of biochemicals that influence the growth/development of other organisms. 

There is then a delicate balancing process between growth and virulence, and the various pairings contribute to diversity. The larger the number of pairings, the better the organism can adapt, which introduces yet another balancing act that ensures diversity and optimal characteristics. (something like that, I was totally physicsing too hard)



\section*{May 21 - Guest lecturer Nealson, Prokaryotes/Electron flow/life}

Electron flow is essential to the generation of energy for all life, the generation of which arises from membrane potential (key to ATP, motility, nutrient transport).

We first discuss prokaryotes (Newman hates this dude xD). They have been around for much longer than eukaryotes, not to mention humans. Prokaryotes do much less than eukaryotes though, functions such as respiration, catabolism, photosynthesis, etc. Biosynthesis/0growth are critical parts, both of which use ATP/NADH. Motility is also important, coming from ATP/PMF. Active transport is a third function that they accopmlish, which is again done by PMF/ATP.

Key metabolic inventions that allowed eukaryotes to surpass these basic functions include photosynthesis, respiration, lithotrophy, autotrophy, heterotrophy, and predation (very few prokaryotic predators).

There are two major sources of carbon flow in Ancient Earth, burial and photosynthesis; burial is very important due to inorganics.  The Mitchell hypothesis is how everything generates energy; details can be found in lecture slides, though it is basically the electron transport chain. A critical part of the Mitchell hypothesis is the distinction between electron and hydrogen carriers. Thus, having electron and hydrogen carriers on either side of a selectively permeable membrane, which intuitively uses an electron transferred down the chain to generate a proton gradient. Iron turns out to be a great metal for this process, and as life evolved to less iron-rich environments life became iron-limited. 

The various classes of metabolizers include methanogens, SRB, denitrifier, and aerobe, which correspond to using \ce{CO2}, \ce{SO4^{2-}}, \ce{MnO2}, and \ce{O2} as oxidants respecitvely. 

There are two important prokaryotic characteristics, small and rigid cell walls. 

We now discuss electron transport in living organisms. God, I don't even know whether this is important because I haven't been paying attention\dots So I think that the idea is just that electrons moving across membranes and generating energy is endlessly fascinating to biochemists, particularly the exact proteins. Though that is my hypothesis which seems to be proven true, I find exactly zero theoretical ground upon which to base this observation; my primary objection is that this field is endlessly boring and therefore by definition is incapable of fascinating. Alas, perhaps I am not enlightened.

Basically, the idea is that glucose is oxidized anaerobically and oxygen is reduced aerobically. Protons diffuse from the anaerobic anode to the aerobic cathode. I'm not sure whether this is the actual mechanism or whether this is just an application, though the lecturer thinks that this happens on biofilms everywhere. This process is extracellular electron transfer, which maybe possibly occurs in microbes as well? OTL I'm bad at takings notes and working.

SUMMARY TIME. MFCs have many applications, such as cleaning up dirty water (produces pure water on cathode side), though it can't be a major source of energy due to the small amounts of energy generated. Exciting new fields of research are electrosynthesis, growth of unculturable microbes on anodes/cathodes, and microbal sense of surface charge.
\section*{May 22 - Other inter-organism interactions}

Symbiosis is the rule and not the exception! Holobiont is a group of symbiotic organisms. How is transmission of symbiosis acheived then? Horizontal and vertical transfer! How is symbiois determined (specificity)? Pretty complicated, but it centers around no interaction between ``candidates.'' Another important factor is development, how simbionts influence development. Key message is unknown because emails were typed. We then discuss how symbiotic stability is acheived/maintained. Microbiota have short-term stability and long-term plasticity. Excatly what microbiota are remains to be discovered\dots The last aspect of symbiosis we discuss is pathogenesis and mutualism.
\section*{May 23 - Nested Ecosystems}

Classical definition of symbiosis is ``Organisms of different species living together,'' but we require interaction for a full definition. Thus, a refined definition would have to include descriptors of intimate, coevolved.

We thus examine the three types of interactions:
\begin{table}
	\centering
	\begin{tabular}{|c|c|c|c|}
		\hline
		&Parasitism & Commensalism & mutualism\\\hline
		Partner 1 & + & + & +\\\hline
		Partner 2 & - & 0 & +\\\hline
	\end{tabular}
	\caption{Whee (Ammesalism is -/0, but is very rare)}
\end{table}

Note that this is a spectrum rather than three distinct classes, and different organisms can exhibit different interorganism interactions depending on its environment. 

A \emph{consortium} of organisms is a group or community of a single species. Consortia (sp?) produce the symbiotic relatioships. A study demonstrates that proximity is not always related to partner interactions; horizontal gene transfer can occur between bacterial species farther away and not in closer bacteria. 

We can then examine vertical transmission of symbiosis, i.e. across generations. Despite the fact that only sperm cells (in the sexual reproduction cycle) do not experience symbiotic relationships, symbiotic interactions persist throughout everywhere else in the growth process, even in an embryo.

We again return to the discussion of pathenogenic vs. mutualistic associations. Pathenogenisis is commonly defined as imbalance in the ecosystem, while mutualism is discussed above.

After a long delay, we are discussing human microbiomes, specifically gut bacteria in C-section and vaginal children. As vaginally delivered children experience better immunity, it is hypothesized that the longer time spent in womb allows the baby to better inheirit the mother's gut bacteria, thus producing better immunity. A similar trend follows for breast feeding.

\section*{May 28 - ``The TV crew is filming your classmate, Yubo Su'' a.k.a. Ecosystem biology}

Ocean is key to ecosystem biology, both large surface area and volume. Yay oceans! Let's talk for the next half hour about how oceans are great. Tropical waters are clear with fewer nutrients and animals/diversity, though the northern/southern oceans are the exact opposite! Then the question is obvious; why do coral reefs work, if the tropics are practically deserts?

I totally missed the answer. Good thing the film crew is gone, so my distractions may go undocumented :D. 

\section*{May 29 - Global issues today/Climate Change}

Coral was discssed in the lecture, so we will discuss the formation of corals. Larva grow to polyps with calcium carbonate shells. Polyps sexually reproduce via spores, which then asexually reproduces to a new polyp larva! The energy required for growth of shells comes from a symbiotic relationship with the photosynthesizing zooxanthellae. The zooxanthellae are passed vertically, i.e. inheirited over generations.

\section*{May 30 - Coevolution of Life and Earth}

How do we recognize life? ``Layers'' are byproducts of life, layering is a key factor (I think?? I missed this whole discussion).

Water is very good for life because it is a great solvent, it maintains its temperature, and it is polar (giving hydrophobic, hydrophilic). Chlorophyll, water, and ozone are key to life on Earth.

Cycles are very important to life, such as the photosynthetic/cellular respiration cycle. Screw this, this is uninteresting. Consult Cody and Cassidy for notes.
\end{document}

\documentclass[12pt]{article}
\usepackage{fancyhdr, amsmath, amsthm, amssymb, hyperref, paracol, graphicx, setspace}
\usepackage[margin=1in]{geometry}
\usepackage[version=3]{mhchem}
\newcommand{\scinot}[2]{#1\times 10^{#2}}
\usepackage[labelfont=bf]{caption}

\begin{document}

\doublespace
\pagestyle{fancy}
\lhead{Yubo Su - Ph12bNotes}
\setlength{\headheight}{15pt}

\section{April 4 - Entropy/Irreversibility}

For a closed system, all accessible microscopic states are equally likely. Then, given the multiplicity $g$ of a state $s$, the probability of thate state is $P(s) \propto \frac{1}{g}$, such that $\sum{P} = 1$. If we then have for example, a binomial distribution of multiplicities $g(N,s) =\frac{N!}{N_{1}!N_{2}!}$ where $N$ is the total number of particles and $1,2$ are the respective states, the Stirling approximation gives that $g(N,s) \approx \sqrt{\frac{2}{\pi N}}2^Ne^{-2s^2/N}$, a Gaussian, as we expect. We note that the width of this Gaussian is $\propto \sqrt{N}$. This was discussed last week.

We then investigate a system of temperature flow. We construct a pair of systems of certain particles $N_1, N_2$ of energies $U_1, U_2$ yielding multiplicities $g_1, g_2$. The total multiplicity of the combined system is then $g = g_1g_2$. We then consider the transfer of energy when $U = U_1 + U_2$ energy is conserved; note that we will not allow particles to transfer. We then note that the new $g'$ after transfer is given by $g' = \sum{g_1(N_1,U_1)g_2(N_2,U-U_1)}$, where we sum over all values of $U_1$. In general, $g' > g$. Note that $g' = g_1g_2$ is a Gaussian centered at some energy $\hat{U}_1$, which is denoted the most probable configuration or the equilibrium configuration. We can then solve for this peak. We differentiate $g_1g_2$:

\begin{align*}
	0 &= dg\\
	&= \left(\left.\frac{\partial g_1}{\partial U_1}\right|_{N_1}\right)g_2dU_1 + g_1\left(\left.\frac{\partial g_2}{\partial U_2}\right|_{N_2}\right)dU_2\\
	&= \left(\left.\frac{\partial g_1}{\partial U_1}\right|_{N_1}\right)g_2 - g_1\left(\left.\frac{\partial g_2}{\partial U_2}\right|_{N_2}\right)\\
	\frac{\partial}{\partial U_1}\left(\left.\ln g_1\right|_{N_1}\right) &= \frac{\partial}{\partial U_2}\left(\left.\ln q_2\right|_{N_2}\right)
\end{align*}

where we take advantage of the fact that $dU_1 = -dU_2$. We then define entropy $\sigma(N,U) = \ln g(N_U)$. Note that entropies are additive: $g = g_1g_2\cdots g_n$, $\sigma = \sigma_1 + \sigma_2 +\cdots + \sigma_n$. We then note that the equilibrium condition from the last line then becomes

\[ \left. \frac{\partial \sigma_1}{\partial U_1} \right|_{N_1} = \left.\frac{\partial\sigma_2}{\partial U_2}\right|_{N_2}\]

This then shows that when the two terms are not equal, energy flows. If the left term is larger, then energy flows from system 1 to system 2, and if the right term is larger, then vice versa. We then define temperature $\tau = \left. \frac{\partial \sigma}{\partial U}\right|_{N}$. This then gives the following expression: $d(\sigma_1 + \sigma_2) = \left(\frac{1}{\tau_1} - \frac{1}{\tau_2}\right)dU_1$. At equilibrium then, it is clear that $\tau_1 = \tau_2$. But if $\tau_2 \neq \tau_1$, then heat will flow from the higher $\tau$ to the lower.

We then note that $\tau = k_B T$ where $T$ is the conventional temperature in $^\circ K$ and $k_B = \scinot{1.4}{-23}\mathrm{J/K}$ is the Boltzmann constant.

We then note that we can write $g(N,s) = g(N,0)e^{-2s^2/N}$. We can also find that $\sigma = \ln(g_1g_2) = C - \frac{2s_1^2}{N_1} - \frac{2s_2^2}{N^2}$ and so substituting $s_2 = s - s_1$ and forcing our constaint $\frac{\partial \sigma}{\partial s_1} = -\frac{4s_1}{N_1} + \frac{4(s-s_1)}{N_2}$, we find that the equilibria $\hat{s}$ give $\frac{\hat{s}_1}{N_1} = \frac{\hat{s}_2}{N_2} = \frac{s}{N}$. 

We then investigate the likelihood of fluctuations $s_1 = \hat{s}_1 + \delta$. This gives $\sigma = C - \frac{2}{N_1}(\hat{s}_1 + \delta)^2 - \frac{2}{N_2}(s-\hat{s}_1 - \delta)^2$, which if we group the $\hat{s}_1$ terms into the $C$ gives $\sigma = C + 0\delta - \frac{2}{N_1}\delta^2 - \frac{2}{N_2}\delta^2$, where we know that the coefficient is $0$ because we have minimized $\sigma$. Then $g_1g_2 = e^{\sigma_1 + \sigma_2} = Ce^{-2\left(\frac{1}{N_1} + \frac{1}{N_2}\right)\delta^2}$. Then when $N_1 = N_2$, we find $g_1g_2 = Ce^{8\delta^2/N}$. We finally find that $\delta \sim \sqrt{N}$. 

If we perform a back-of-the-envelope calculation, giving $N \sim 10^{22}$, we find that $\delta \sim 10^{11}$. We then ask the probability that $\delta$ varies as high as $10^{12}$? It can easily be calculated that given $\delta = 10^{12} \Rightarrow g_1g_2 \sim e^{-400} = 10^{-174}$. In comparison, the age of the universe is $10^{61}$ planck times.

The second law of thermodynamics can then be seen here, because entropy never decreases when two systems are brought into contact. This means that almost all processes are irreversible.

We then discuss open systems. We first examine a quantum system with a spectrum of eigenstates. We then would like to compute, for two eigenstates, $\frac{P(\varepsilon_1)}{P(\varepsilon_2)} = \frac{g_R(U_0 - \varepsilon_1)}{g_R(U_0 - \varepsilon_2)}$, where the equality is because the difference in energies is drawn from the reservoir, so we calculate degeneracies of the reservoir. This then gives:

\begin{align*}
	\text{ratio} &= e^{\sigma_R(U_0-\varepsilon_1) - \sigma_R(U_0-\varepsilon_2)}\\
	&= \mathrm{Exp}(\sigma(U_0) - \varepsilon_1\left. \frac{d\sigma}{dU}\right|_{U = U_0} +\dots + \sigma(U_0) - \varepsilon_2\left.\frac{d\sigma}{dU}\right|_{U=U_0} +\dots)
\end{align*}

	We note that the smaller order terms are negligible, and the constant terms cancel, so we find that the ratio is $= \mathrm{Exp}(-\frac{\varepsilon_1 - \varepsilon_2}{T})$ where $T$ is the temperature of the reservoir. This then gives that $P(\varepsilon) \propto e^{-\varepsilon/T}$. This formula is called the Boltzman factor, the probability of being in a state of energy $\varepsilon$ when in reservoir at temperature $T$. We then normalize this by requiring that these probabilities sum to $1$. 

\end{document}


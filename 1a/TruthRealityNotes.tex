\documentclass{report}
\usepackage{amsmath, amsthm, amssymb, fullpage}
\begin{document}

\title{Hum009 Notes - Truth and Reality}
\author{Yubo Su}
\date{ }

\maketitle

\tableofcontents

\chapter{Hum009}

\section{Oct. 2 - What will we be learning in class?}

Philosophers tend to be interested in roughly five sorts of issues:

\begin{itemize}
\item What is the nature of world?
\item What is the nature of the divine?
\item What is human nature?
\item How do these things interact and how do they overlap?
\item What implications do these give on how to live life and the definitions of right/wrong?
\end{itemize}

We will not be touching on the last of these five.

We will study how we know something is real and how minds and modies interact. We can also decide whether God exists and what we can know. We will look into free will. There is no consensus on this subject.

We will be interested in our reasons for belief and the assumptions we make in believing things.

\section{Oct. 4 - Berlin, Pryor}

Discussion:

\begin{itemize}
\item What puts a question in Berlin's "`intermediate basket"'?

\begin{itemize}
\item A question that is neither observable or calculable (the other two baskets).
\item Intermediate questions have no pointers towards their answer.
\item There must be no means to answer the question (no rigorous calculable framework).
\item "`Unintelligible questions"' are according to Berlin anything with no set framework.
\item Maybe there are multiple classes of unintelligible, such as "`what is color seven?"' which gives incorrect pointer, and absence of pointers, which is philosophy.
\item This basket of questions is a set of unevolved disciplines. 
\item Philosophy is humans trying to figure out a place in the world, an emotional response (Pg. 10).
\end{itemize}

\item If philosophy is a disappearing field, why does Berlin believe that philosophy is endless?

\begin{itemize}
\item Berlin says that newer disciplines still spawn philosophical questions, keeping philosophy alive.
\end{itemize}

\item How does Berlin handle the opinion that philosophy is irrelevant to our lives?

\begin{itemize}
\item Berlin shows that philosophy manifests itself in many fields.
\item Berlin categorizes philosophy as all logically consistent unintelligble questions.
\item Philosophy is relevant for philosophical reasons: rational order and just appreciation of things, according to Berlin. 
\item Philosophy is meant to lend understanding and rules. 
\end{itemize}

\item Berlin's model is that everybody philosophizes and is capable of doing so, even if not on a professional scale.

\end{itemize}

Note: As an amateur philosopher, if one finds in established texts trivial mistakes, the mistake is one's own! Reread, meditate, and figure out why you're wrong.

Prior's piece is probably interesting for the following things:

\begin{itemize}
\item Use arguments, not assertions.
\item Outlines will be required for the first paper (1 page for a 5 page paper).
\item Find important arguments and then connect dots: is probably best way to outline.
\item Write as if the writer is lazy, stupid, and mean.
\end{itemize}

\section{October 9 - Descartes P.1}

My Notes (oops, wrong section):
\begin{itemize}
\item Reason is what makes us strong, and what distinguishes people is not an ability to reason but the ability to reason along correct paths.
\item Descartes believes that his search for truth is noblest of all "`vain and useless"' professions.
\item Mathematics is a language of rigor but an undervalued subject in all but mechanical arts.
\item All other sciences have unstable foundations, which Descartes presumes can be overthrown. Notice the careful rhetoric about religion.
\item Descartes then decides that he will travel the world and find his own knowledge/truth.
\item Practical judgement > theoretical, because practical impacts the lives of the thinkers.
\item "`Thus the greatest profit I derived from this was that, on seeing many things that, although they seem to us very extravagant and ridiculous, do not cease to be commonly accepted and approved among other great peoples, I learned not to believe anything too firmly of which I had been persuaded only by example and custom; and thus I little by little freed myself from many errors that can darken our natural light and render us less able to listen to reason."' - Does this mean that he grew to distrust people's words? Is he accepting hypocrisy? I'm not quite sure what the general meaning is?
\end{itemize}

Questions:
\begin{itemize}
\item What is Descartes's greatest goal through this reading? - To introduce doubt and skepticism? To believe things only through good reason! 
\item Is Descartes's view of knowledge as a sort of math correct, with axioms? Maybe the house metaphor, the foundational mindset, is incorrect; think webs rather than foundations.
\item How can we establish a new set of beliefs? Descartes says that you don't have to have a belief disproven before razing it, you can just start over immediately!
\item Descartes argues that anything with even remote amounts of doubt is to be disregarded as "`truth"'.
\item He first overthrows premise that senses are universally trustworthy then overthrows any trust in the senses.
\item Can you know you're not dreaming? Descartes argues not, but is that true? Figure it out!
\item Must one not be dreaming to know something about reality? Could you discover things even while dreaming?
\item The dream argument seems to depend on that anything inconsistent can debunk an initial belief. Is that true though? Must everything be run through?
\item Note that there is a weird argument shift from "`can't tell whether dreaming while dreaming"' to "`can't tell whether dreaming whie awake"'. For example, blind people can't tell whether lights are on or off, but sighted people definitely can...
\item Painter analogy establishes that there must be certain truths. 
\item God argument says that God has control over truths, and were there the worst case scenario that God were constantly deceiving, then he knows nothing. Also, since we do make mistakes, God cannot have made us in a way that is perfect and thus we cannot rule out the worst case scenario. But then, is the "`evil genius"' considered part of the God argument or is it a fresh argument?
\item Were God to have made us perfectly, we're still being deceived. If we come about in any less perfect process, we're even more screwed, and we can still conclude only general features.
\item Always keep in mind that the goal is absolute certainty, not things that are applicable in the real world. 
\end{itemize}

\section{October 11 - Descartes P.2}

Questions:
\begin{itemize}
\item The Archimedes reference shows that Descartes wants to rebuild knowledge from a fulcrum.
\item "`Cogito Ergo Sum"' - I think therefore I am: broken down - 
\begin{itemize}
\item I think
\item Whatever thinks exists - Thoughts exist $\rightarrow$ Thoughts are properties $\rightarrow$ Properties exist in things $\rightarrow$ Thoughts exist in things (call it "`I"')
\item I exist
\end{itemize}
\item Cogito can also be called "`non-inferential knowledge"', something immediately seen and not thought over (for example, during role call, anything somebody does proves that he/he is here, so non-inferential b/c no dependence). In other words, thinking and existence are permanently tied together. 
\item Note that the clothing and wax argument show that we only sense what the mind interpolates, not the things we get from senses. 
\item Maybe "`thinking"' only needs an actor because of grammatical reasons! Language necessicates an actor for verbs. The counterargument is then that "`thought"' is a property of things, so call what "`thought"' is a property of I.
\end{itemize}

\section{October 16 - Descartes Medidation Six}

Meditations 2(closing) and a bit of Meditation 3:
\begin{itemize}
\item Note that all of Descartes critereon for "`What am I?"' includes imagining and sensing, which aren't so much reliant on external stimuli, in the same list as the reactive senses. This is justified as being a list of things for which one must be conscious.
\item It is easy to restrain claims to "`it seems to me"' rather than a universal claim, which makes the Medidation Two seem much more plausible. Descartes tries to merge self-knowledge (restricted) and external knowledge (universally shared).
\item Every thought and sensory input once fed into Cogito then yields the conclusion that "`I exist and I am a thinking thing"', so this must be certainty.
\item Truth Rule: Trust those things you clearly and distinctly perceive. \begin{itemize}
\item I only know this because I clearly and distinctly perceive it.
\item If my clear and distinct perceptions were ever wrong, I could no longer be sure of my existence.
\end{itemize}
\end{itemize}

Meditation Six at last!
\begin{itemize}
\item Schematize the paragraph that starts with "`First, I know that all the things that I..."' (premises and conclusions)
\end{itemize}

\section{October 18 - Kripke FML}

Notes:
\begin{itemize}
\item \emph{a priori} $\neq$ necessary
\item necessary can be evaluated without discussing sufficiency.
\item Rigid designators are such that in every world the same thing is designated, while strong designators designate necessarily existent objects.
\item Proper names are rigid, but not strongly rigid because the person might not exist.
\item Examining ``S is 1m long at $t_0$'' shows that there is no conflict because we define 1m. However, it is not a necessary truth because $S$ isn't always 1m long. This arises because 1m is rigid while length of S is not. 
\item But then should 1m be a rigid designation, the statement then becomes contingent on a lot of things, even as it is \emph{a priori}. 
\item Using a sentence like ``Aristotle is the greatest Plato pupil'' could be bad as definition, b/c it is nonrigid (could have other pupil). On the other hand, we can use the sentence to fix the referent to always mean Aristotle, where Aristotle refers to the man rather than the pupil.
\item Note that the evilness in names like Hitler are nonrigid because in other worlds he would have different properties! And not be Hitler.
\item Defining meterstick = contingent on what the meterstick was like! But also, we define a length but not necesarily that the meterstick is always 1m, so the meterstick is nonrigid reference.
\item There are contingently true identities!
\item Mill says all singular/general = connotative, proper = non-connotative. Kripke disagrees, saying that general nouns aren't connatative because they still describe something proper rather than an amalgam of properties; if something not H2O is discovered to be water-like, is it water? no. Ergo, water is a general yet proper noun.
\item Note that defining light as ``stream of photons'' is fixing a reference, beacuse in some alternative world should sound be the means of vision then	their ``light'' would be different from ours but still cause the same sensations.
\item There are also a posteriori truths, which are truths that must have been regardless of outcome; such as mathematics, though things like the elemental nature of Au can be disputed.
\item Any necessary apriori or aposteriori truth could not have turned out otherwise, though aposteriori truths aren't unique going back through time; maybe other situtaions could have produced the same judgement.
\item Note that it is important in asserting equality/inequality to note whether a non-rigid designator is contingent but fixed by reference to a rigid designator, which could lead to a confusion on the rigid designator.
\item Descartes's separation of mind/body holds true because both mind and body are rigid designators.
\item It is clear that the heat example is a posteriori contingent, because other beings could have arrived at a different truth from epistomologically identical conditions.
\item The pain analogy is different because there is no intermediate step of sensing the heat, ergo no aposteriori contingency is to be found. This is identified because we label heat based on something we feel, so it is in a sense a reference to how we feel rather than a completely rigid designator.
\end{itemize}

\section{October 23 - More Kripke}

\begin{itemize}
\item Before Kripke, a priori = necessary, a posteriori = contingent. Kripke argues that a posteriori/priori is the means of knowing knowledge while necessary/contingent is the character of the knowledge.
\item An example of a necessary, aposteriori truth would be math proofs, but Kripke suggests other examples. Others include water = H2O, light, etc., truths thta must have been even before discovery.
\item Kripke will argue the difference between seeming to conveive something and really conceiving it (not in the scientific case), EXCEPT when discussing mental/physical phenomena.
\item Kripke is careful to differentiate between metaphysical (the way the world is) and epistomological (how you know things) claims.
\item Define possibility of $P$ to be if and only if not necessarily not $P$. Define conceivability of $P$ to be if and only if you can't know independent of experience that it is untrue, or if and only if it is not a priori that not $P$.
\item Famous example: one knows that bachelors are not married independent of experience. One knows this without going through any experiences, so it must be a priori. 
\item The connection between conceivability and possbiility is: if I can conceive $P$, then $P$ is knowable a priori. If $P$ is a priori, it cannot have a logical contradiction and thus necessarily is not a contradiction, which makes $P$ possible. 
\item Frege (philosopher) argues that labels are what we make of them. He then asks how to affirm ``Hespherous is Phospherous'' (morning star is evening star, Venus is Venus). If we believe that the words mean what they reference, then Venus is Venus. 
\item Frege then points out that we had to go out to discover that these terms correspond to Venus, and thus it must have been aposteriori knowledge. So Frege argues that this truth must have been contingent. So then obviously the statement isn't a synonyms argument or an argument about the nature of the words, otherwise it should be known a priori. 
\item Frege concludes that the meaning of a term isn't necessarily what it it denotes, that reference $\neq$ meaning. The sense of the term is the meaning, and only upon discovering the senses of both Hespherous and Phospherous can we determine that they sense the same object.
\item Ex. If we need Chris, the guy in our class, we need a sense of which Chris we're talking about, essential/unique properties of this Chris. 
\item Kripke argues that all identities are necessary and cannot be contingent. 
\begin{enumerate}
\item X = Z.
\item Leibniz's law cites that for two things to be identical they must share all properties. 
\item X is necessarily X.
\item X is necessarily Z.
\end{enumerate}
\item Kripke thus argues that they are necessarily identical in all possible worlds regardless of our state of our knowledge.
\item Kripke then asks that, given the statement ``Gideon is the instructor of Hum009'', how do we refer to Gideon in worlds where he does not teach Hum009? One can:
\begin{itemize}
\item Reject alternative realities (claim that it is an essential property), which then means that all properties become essential, OR
\item Argue that alternative worlds don't reflect the ``same'' Gideon. 
\end{itemize}
\item However, it is clear that when we regret or when we look ahead, we stay the same person, so counterfactual claims are possible. This then concludes that some things have nonessential properties that we use to ``sense'' things, which means that the sense is not the properties, but the reference! QED
\end{itemize}

\section{October 25 - Even more Kripke}

\begin{itemize}
\item There has got to be something beyond the reference of a word when analyzing it, and we call that the sense, the unique properties of the world that we use to identify the term.
\item Nixon's example: even if Nixon weren't president, he wolud still be Nixon. Therefore, the reference/sense of the word isn't enough. 
\item Cannot generate worlds that claim ``Nixon is not Nixon'', but can change almost all other modifiers. 
\item Names or ``natural kinds'' are rigid designators by Kripke, at least for now.
\item Note that rigid/nonrigid designators have no difference in the actual world.
\item We must ask ourselves ``How does counterfactual discourse develop?''
\begin{itemize}
\item To fix rules for designators, they must be universal and always apply. 
\item Is there a way for us to find descriptions that help us define rigidly objects? 
\item Descriptions are meant to fix references, a sort of stipulation. For example, let some rigid designator pick out in all worlds what it picks out in this world. This will then be the refeence of the rigid designator in all possible worlds.  
\end{itemize}
\item Consider though, the rigid designator ``heat''. How did people talk about heat before we knew it was the motion of molecules? It was referred to as the sensation of warmth. However, note this is a nonrigid designator. It is not a definition because the sensation is not an essential property of heat.
\item We then return to the question of contingent, a priori truths. The above designation of heat is probably a priori, but it cannot be necessary due to the nonrigidity of one of the designators. It must therefore be a priori contingent!
\item Remember that Hespherous is Phospherous must have been discovered, but is necessary because both designators are rigid! So it then becomes clear that a priori/necessary and a posteriori/contingent are no longer always hand in hand.
\item The strategies available to sort through good/bad claims cannot be used to sort through physical/mental connections; the appearance of contingency actually implies contingency here. Because pain doesn't have to imply c-fiber stimulation in all worlds, it must be that mind/body are distinct.
\end{itemize}

\section{October 30 - Galileo/Locke}

Read a bit of Meditation Six in addition to Barkeley reading, and be ready to schematize paragraph $4$ in the Barkeley.

\begin{itemize}
\item Galileo classifies properties into non-existent perceived and existent inherent properties. 
\item tooo lazy to take notes....
\item Locke separates qualities into primary and secondary qualities. Our ideas of primary qualities resemble the primary qualities themselves (size), while our ideas of secondary qualities do not resemble the secondary qualities (ticklishness). 
\item Locke specificies resemblance to differentiate primary vs. secondary, which is what does a lot of heavy lifting for him.
\item Locke also says that qualities have ``powers'' to generate ideas, but what power is possessed when there is nothing in which to generate ideas? Would then these properties really exist? Or does Galileo hold weight in calling these properties facetious?
\end{itemize}

\section{November 6 - Berkeley}

\begin{itemize}
\item Two significant points: there exists an inner world of ideas of certain truths and an external world that we only can claim directly to perceive rather than to know; we then have to reason through our ideas based on something external to justify something outside of the world. 
\item There is a distnction: Locke explained objects as not having the qualities the way we think they have them (having qualities they don't have), but Berkeley claims that these qualities cannot exist (having qualities they cannot have). 
\item Berkeley says there are three ways to acquire knowledge: actual sense experience, introspection, imagination. (Note that Descartes considers innate truths such as math to be a fourth category of knowledge) 
\item A single object/name refers to a whole collection of ideas, including internal responses such as the second way to acquire knowledge above. 
\item Ideas must be perceived to exist. We know that there are at least two things that must exist - thinking things and ideas. We know that thinking things exist outside of our three categories of ideas. 
\item Berkeley's ideas are along the lines of the appearance-reality distinction we've been studying so far.
\item Paragraph 3 starts the Ordinary Usage Argument. The immediate objects of perception cannot exist unperceived is not the exact point Berkeley is making, but more so that the qualities we use to characterize objects cannot exist unperceived. This contradicts Locke and is a much more powerful, much more interesting point. \emph{Essi is percepi}. 
\item Berkeley then concludes that objects cannot exist absolutely outside of perception, because everything known about objects is perception thereof, so matter is unintelligble.
\item No sensible object can exist to be unperceived, says Paragraph Four:
\begin{enumerate}
\item Sensible things are perceived by sense.
\item All we perceive by sense are our own ideas/sensations.
\item Thus, sensible things are just our ideas/sensations
\item Ideas or sensations cannot exist unperceived by some understanding.
\item Sensible things can't exits unperceived by some understanding
\end{enumerate}
\item Philosophical argument is paragraph 4!
\end{itemize}

\section{November 13 - Berkeley/Malcolm}

\begin{itemize}
\item Just remember that Berkeley is trying to prove that all science is knowing God's ideas rather than to disprove the existence of anything in particular. He only wishes to reframe ``matter'' as God's ideas. 
\item The Master Argument: Ballsy Berkely claims that the mere possibility of conceiving an object/idea to exist outside of a mind. In other words, one can conceive the possibility of a sensible object existing unperceived if and only if \emph{esse is percipi} is false.
\item One can conceive the possibility of a sensible object existing unperceived if and only if one can conceive a sensible object existing unperceived. 
\item Berkeley then claims that one cannot conceive a sensible object existing unperceived. Thus, one cannot conceive the possibility of a sensible object existing unperceived. This proves that \emph{esse is percipi} is true.
\end{itemize}

Malcolm

\begin{itemize}
\item Mill's argument is one by analogy, that when I exhibit behavior $X$ I am enjoying mental state $Y$, and thus when I observe behavior $X$ in others they must be in the same mental state $Y$. 
\item Malcolm argues that since there exist no criterion for mental states, saying the ``same'' mental state actually has zero meaning. Were there a criterion, Mill must cite evidence that fulfill the criterion. 
\end{itemize}

\section{November 20 - Descartes Meditations 3,5}

Note that the God arguments in Aquinas are called cosmological arguments, which are basically First Cause arguments.

Paley is called the teleological argument, or proof by design. Watchmaker's analogy, means that appearance of complexity must imply actual design, because chance has too low of probability. Note that highly improbable does not imply logically impossible. Coordination is key to implying design, because things like rocks have no coordination between parts. However, if we examine life, we see that there is a coordination in self-preservation, which is why this doe still imply design. 

Descartes:

\begin{itemize}
\item Remember that ideas come from sensation, imagination, and innate ideas (pre-loaded bloatware). We then ask which of these categories can support something external to us?
\item Descartes searches for an idea that has a cause aside from himself. What ideas concern are what differentiates ideas, while all ideas are equal in that they are modes of thinking.
\item Descartes then differentiates between formal and objective reality. Descartes argues that ideas about God must have more objective reality than finite substances and ``modes and accidents.''
\item Objective reality is defined as the fact that the objects ideas are about exist. In other words, the idea of the object of a thought exists if it has objective reality.
\item The heirarchy exists because a mode/accident must have a finite substance as a base (i.e. a coat, a finite substance, is necessary to claim that a coat is red, a mode/accident). Because God can be thought of independently of the rest, God must be at the top of the heirarchy.
\item Every objective reality then needs to be based in a formal reality tbat is at the same level of the heirarchy. For example, I exist, and since I am a finite substance, I can cause anything finite or modes/accident. But I cannot cause God. Thus, since only God can cause God, God must exist. 
\item If one argues that one doesn't have an idea of God, then it is like saying ``I have no concept of infinity in mathematics'' which is a deficiency of the thinker not the thought of; the idea of God exists in Descartes' mind, which is sufficient. 
\end{itemize}

\section{November 29 - Ressurection and rudimentary notes on Ayer/Strawson}

Determinism means everything is causally necessicated and thus nobody could have done otherwise. Because free will relies on being able to do otherwise, nobody can have free will and thus no moral responsibility. Ayer intervenes by claiming that we deliberate about our desires and because our desires are causes for our actions our deliberations still have an impact on what we end up doing. Thus, determinism and free will can be reconciliated, and Ayer is a compatibilist.

\end{document}
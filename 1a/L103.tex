\documentclass{report}
\usepackage[cm]{fullpage}
\usepackage{amssymb, amsmath}
\usepackage[utf8]{inputenc}
\usepackage[T1]{fontenc}
\begin{document}

\title{L103 - Le fran\c{c}ais - Professeur Tr\'evor Merrill}
\author{Monsieur Yubo Su}
\date{ }

\maketitle

\tableofcontents

\chapter{1 - Les notes pour L103a}

\section{October 3 - Intro (few notes)}
Son bureau est dans Dabney 115 \`{a} Vendredi pendant 4h-5h ou faire un e-mail (merrill@hss.caltech.edu ou tcmerrill@gmail.com (? on doit confirmer)). 

Quelque questions qu'on peut utiliser pour s'introduire:

\begin{itemize}
\item Comment t'appelles tu?
\item Quel \^age as-tu?
\item o\`u est-ce que tu habites?
\item D'o\'u viens-tu?
\item Qu'est-ce que tu aimes faire?
\item Qu'est-ce que tu \'etudies?
\end{itemize}

Pour les instruments musiques, utiliser "`jouer de"' et "`faire de."' Pour les sports, utiliser "`jouer au"' et "`faire de."'

\section{October 5 - Conjuger les verbes...	}

-er verbs (90\% de toutes les verbes):

\begin{itemize}
\item Je respire
\item Tu respires
\item Il/elle/on respire
\item Nous respirons
\item Vous respirez
\item Ils/elles respirent
\end{itemize}

Exc\'eptions: aller, nettoyer, manger

\emph{Scapegoat} est "`bouc \'emissaire"'.

Nous avons aussi conjuger les verbes \emph{-re} et \emph{-ir}...

\section{October 8 - Une petite dict\'ee}

Dict\'ee: L'hypocrisie est un hommage que le vice rend \`a la vertu. 

Les devoirs: SAM: IV - p. 6, I - p. 7, II - p. 8.
EBF: Lire 42-44, etudier 45-47.

\section{October 10 - L'imperatif}

Finir: \begin{itemize} \item Finis!
\item Finissez vos devoirs!
\item Finissons!
\end{itemize}

La derni\`ere option est plus comme une suggestion qu'une command. C'est comme "`Let's finish"' en anglais.

Notez les exceptions pour l'imperatif:

\begin{center}
\begin{tabular}{|l || l | l | l|}
\hline
&tu&vous&nous\\
\hline
\^etre & sois&soyez&soyons\\
\hline
savoir&sache&sachez&sachons\\
\hline
aller& va & allez & allons\\
\hline
\end{tabular}
\end{center}

Note que les verbs \emph{-er} n'utilise le \emph{-s} pour le forme \emph{tu} sauf quand il y a le pronom \emph{y} comme "`Vas-y"'. Sauf \c{c}a, on n'utilise pas le \emph{-s}: "`Va \`a la chambre!"'

\section{October 12 - COMPOSITION 1!!}

Sujet: Ecrivez une lettre de rupture (imaginaire) dans laquelle vous expliquez \`a votre ami(e) pourquoi vous rompez. Il devra \^etre 1-2 pages. Utiliser "`Cher [Pierre]"', la voix informel. Les devoirs: SAM I - p. 15, II - p. 16, VII - p. 18 (IV - p. 6, I - p. 7, II - p. 8.)

La vocabulaire:

\begin{itemize}
\setlength{\itemsep}{0pt}
\setlength{\parskip}{0pt}
\setlength{\parsep}{0pt}
\item l'amour
\item tomber amoureux de-
\item aimer\item la nuit
\item rompre/la rupture/caisser
\item d\'etester
\item l'engagement
\item divorcer/la s\'eparation
\item le grand amour/le coup de foudre
\item faire la cour \`a-/s\'eduire
\item embrasser
\item un baiser (faire de l'amour)
\end{itemize}

Une lettre pour les amoureux (practique en classe):

La semaine derni\`ere, moi et Pi\`erre, nous sommes all\'es en Italie pendant trois jours. La premi\'ere jour, nous nous sommes promen\'es sur la rue Appian. Apr\`es \c{c}a, nous avons din\'e au restaurant au bord de la rue qui a la fromage tr\`es delicieux. Nous avons mang\'e beaucoup, mais Pierre a pay\'e l'addition. Il le fait toujours \c{c}a! Il ne me laisse jamais l'opportunit\'e de payer pour moi-m\^eme. Apr\'es avoir mang\'e le d\^iner, nous sommes all\'es \`a la fontaine de la Villa Medici. C'\'etait vraiment assez jolie que j'ai entendu! La jour prochaine, nous avons assist\'e \`a l'op\`era Don Carlos. Pierre sait bien cette op\`era, et il m'a dit beaucoup ce que les chanteurs ont bien fait et ce qu'ils ont mal fait. J'ai appris beaucoup! Cette nuit, nous avons fait une promenade en bateau sur la rivi\`ere Rubicon. La matin prochaine, nous avons retourn\'e aux \'Etats-unis. C'\'etait vraiment une journ\'ee tr\`es romantique. C'\'etait notre premi\`ere journee avec chaqu'un d'autre, et si nos journ\'ees en futur sont toutes commes \c{c}a je suis tr\'es excit\'ee!

C'est tout. Je t'embrasse!

Selina

\section{October 15 - Dict\'ee}

Dict\'ee: Il y a des gens qui n'auraient jamais \'et\'e amoureux, s'ils n'avaient jamais entendu parler de l'amour. 

``Entendre parler'' et ``to hear of'' en anglais. 

Je suis confus $\Rightarrow$ je suis embarass\'e.

\section{October 17 - Les f\^etes en indochine p. 64 EBF}

Les devoirs: Etudier pp. 70-73 dans EBF. Composition \#1 \`a rendre vendredi. L'interro \#1 est Mercredi prochain! Le 24. Le revue pour l'interro va se rendre la classe prochaine.

\section{October 19 - }

Lire pp. 82-89. Les adjectifs BAGS (Beauty, Age, Goodness, Size) sont devant le nom, mais les autre suivent le nom.

Le Quiz ($\approx$ trente minutes) 	sera:

\begin{enumerate}
\item Remplir les blancs
\item R\'eponses courtes
\item Court essai
\item Une dict\'ee
\end{enumerate}

Study:

\begin{itemize}
\item Le pr\'esent
\begin{itemize}
\item Les verbes -er, -re, -ir, et irr\'eguliers (changes orthographique)
\item Depuis quand/depuis combien de temps + le pr\'esent
\end{itemize}
\item Le pass\'e compos\'e
\begin{itemize}
\item Le choix d'auxiliare
\item L'accord du participe pass\'e (avec les objets directs ou le sujet)
\item Les adverbes (bien, beaucoup, mal, souvent, etc.) et leurs positions + les changements du participe pass\'e (comme au-dessus).
\end{itemize}
\item L'imparfait
\begin{itemize}
\item Les \'emotions/les \'etats monteaux 
\item Pour faire la description
\item Les habitudes
\item Les actions progressives (e.g. Je regardais la t\'el\'e pendant que je mangeais. 
\end{itemize}
\end{itemize}

Le sommeil = le nom pour ``dormir.''

\section{October 22 - Rien}

\textbf{Dict\'ee: } Maintenant, que l'hiver de notre m\'econtentement s'est chang\'e en \'et\'e glorieux par ce soleil d'York. Et toute la nu\'ee (cloud), pesant sur ma maison, engloutie dans le sein (bosom) profond de l'oc\'ean. 

En fran\c{c}ais, on utilise les Alexandrins (douze syllables) pour les poems. 

\section{October 24 - Reviser le composition}

Reviser le jet de la premi\`ere composition et le rendre Vendredi prochain. 

\section{October 26 - Lire un romain!}

Les devoirs: p. 40 - III (dict\'ee), p. 41 - I.

\section{October 29 - Pass\'e Simple}

Autobus = prendre par les citoyens, autocar = prendre par les tourists.

d\'evaler un escalier = le descendre tr\`es rapidement 

Au bord de la Seine, un homme d'affaires se promenait apr\`es avoir fini son travail. La lune brillait et le vent soufflait doucement. Les feuilles d'un seule arbre sussuraient dans le vent. Une autobus et une fille \`a v\'elo passaient l'homme. Mais, soudainement, deux hommes et un petit gar\c{c}on l'ont approch\'e. Ils lui ont demand\'e beaucoup d'argent, et l'homme d'affaires avait trop peur de refuser.

Les Devoirs: Lisez \emph{Les Malendtendus} et notez les mots et les phrases qu'on ne conna\^t pas.

\section{October 30 - Le subjonctif/conditionnel}

Le subjonctif et utilis\'e pour exprimer le doubt ou les \'emotions, comme ``Il faut que'', ''J'\'exige que'', ``Il est essentiel que'', ``Je ne pense pas que'', ``Je suis tr\`es heureux que'', etc.

Le conditionnel et utilis\'e apres ``Si + imparfait...conditionnel''. 

Il faut reviser le pass\'e simple! C'est difficile. 

\section{October 31 - Plus de romain}

Le Composition doit \^etre rendu Vendredi, EBF pp. 198-203. wordreference.com est meilleur que Google Translate. 

Un tueur = quelqu'un qui tue les autres personnes, et un tueur \`a gages ou un tueur \`a s\'erie est un tueur qui tue pour l'argent.

\section{November 2 - Des notes sur ma composition}
tel = ``such'' en anglais, comme ``la telle d\'ecision'' est ``such a decision''
\'ego\"iste = selfish. C`est different que ``egotist'' en anglais! ``egoist vs. egotist.''
On peut dire ``Je pars pour Paris dans deux semaines'', ou ``Je pars \`a Paris dans deux semaines.''
``To realize'' est ``m'\^etre rendu compte''
``Nulle part'' est le contraire de ``partout'' (everywhere)
``Laquelle est-ce que tu aime, l'opera ou le ballet?'' --> ``J'aime \emph{et} l'opera et le ballet''

\section{November 12 - L'utilisation de ``de''}

C'est quelqu'un \emph{de} mechant. C'est quelque chose \emph{d'}amusant. L'adjectif est toujours masculin quand on parle de ``quelque \_\_\_''.

Je n'ai rencontr\'e personne \emph{de} gentil \`a Paris. Nous n'avons vu rien \emph{d'}interessant. 

\section{November 13 - Comment poser une question}

Les devoirs: \'edutier EBF p. 220-230

L'inversion: Avez-vous l'argent? Les \'etudiant sont-ils en col\`ere? 

Les mots interrogatifs: Quel, quand, qui, combien, comment, pourquoi, quoi, que, qu'est-ce que, etc.

\section{November 14 - Quiz 2/Composition 2!}

Les devoirs: SAM p. 82 IV, p. 83-4 I-III. 

La deuxieme Quiz (le 21 Novembre):
\begin{itemize}
\item Le pass\'e simple
\item la n\'egation
\item Quelque chose de/quelqu'un de/ personne/rien de, etc.
\item ne...que
\item l'interrogation
\end{itemize}

Composition 2 (le 21 November aussi): Un dialogue ``L'interrogation''. 200-250 mots. Employer toutes les formes de l'interrogation. Un e-mail va suivre.

Les forms d'interrogation:
\begin{itemize}
\item Intonation: Vous \^etes content, Monsieur?
\item Est-ce que vous \^etes content, Monsieur?
\item \^Etes-vous content, Monsieur?
\begin{itemize}
\item Une autre form: Le Monsieur, est-il content?
\item ``t'' euphonique: Aime-t-on le fromage?
\item Le negatif: N'\^etes-vous pas content?
\end{itemize}
\end{itemize}

Note: Quel pronom utilise-on? \begin{itemize}
\item J'ai vu un cheval - Qu'est-ce que tu as vu?
\item J'ai vu Paul - Qui est-ce que tu as vu?
\item J'y vais - Qui est-ce qui va?
\end{itemize}

\section{November 19 - Pronons}

Les quatres groups de pronons
\begin{itemize}
\item Sujet - Je, tu, il, on, nous, vous, elles
\item Objet direct - Me, te, le/la, nous, vous, les
\item Objet indirect - Me, te, lui, nous, vous, leur, [y et en]
\item Pronons Disjoints - Moi, toi, lui/elle, nous, vous, eux/elles
\end{itemize}

L'ordre (chanson \`a la melodie de Blue Danube): Je ne me le lui y en verbe pas. 

\section{November 26 - Plus de pratique sur les pronons}

Tousser = ``to cough.''

\section{November 28 - Refaire Composition numero deux}

Les Devoirs: p. 90, I-III. Composition numero deux \`a rendre le 5 d\'ecembre.

\`A l'heure actuelle, quel est le plus grand probl\`eme auquel homme/l'\^etre humain doit faire face? 

Mener = ``to lead to''

\section{December 3 - Un peu de devoirs...}

\'Etudier pp. 314 - 322 (EBF). Deuxi\`eme jet de composition \`a rendre mercredi. 

\section{December 7 - Un petit jeu}

\`A l'avenir, vous reussirez. Vous gagnerez plus d'argent que vous pouvez imaginer maintenant. Le success vous suivra quoi que vous fassiez. Vous recevrez la Prix Nobel pour la mathematique. Mais, je suis d\'esol\'ee de vous dire, votre vie personelle ne seras pas aussi confortable. Votre marriage seras difficile au d'abord, et vous deux auriez beaucoup de dispute, particuli\'erement sur les \^ages de l'autres personnes. Mais, si vous restez content et travaillez avec d\'etermination, tous sera bien. Vos enfants seront en bonne sant\'e, intelligentes, et gentille. 

\end{document}
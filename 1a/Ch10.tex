\documentclass{report}

\usepackage{amssymb}

\begin{document}

\title{Frontiers in Chem}
\author{Yubo Su}
\date{ }

\maketitle

\tableofcontents

\chapter{October 4 - Organizational Meeting}

Lab tours will be at:

\begin{itemize}
\item Monday 1600-1700
\item Tuesday 1300-1400
\item Wednesday 1300-1400
\end{itemize}

\chapter{October 11 - Sarah Reisman}

There is a lot of chemical synthesis that I am missing out on...but they can help provide access to unnatural alternatives to natural drugs. For example, by tweaking some groups on certain molecules you can either increase effectiveness or circumvent antibiotic resistance. Also, some antibiotics are too rare, so synthesis can significantly drop the prices of these drugs. Designing new reactions can also help synthesize interesting things. Of course this must also be made to be as industrially feasible as possible. For example, we can target an ideal molecule, split it into parts that are commercially available and easy, and then devise a reaction to accomplish that synthesis.

Keep in mind though, that the reaction isn't a simple matter of getting particles next to particles, we also have to consider orientation. We have to devise a reaction path and then figure out which orientations will actually generate the desired reaction. This is complicated because some partcles are chiral, called enantiomers (or stereoisomers!). This makes life much more difficult...And not only will this mean sometimes reactions don't work, sometimes they will produce undesirable compounds! Also, Caroway flavor (Bright Red gum) and spearmint flavor correspond to stereoisomeric molecules. Another example is a drug that was a sedative and morning sickness relief that had an isomer that was teratogenic. Worse, the enantiomers would convert in the body, meaning you can't avoid giving the (S)-thalidomide enantiomer when giving the (R)-thalidomide enantiomer.

To succeed in this, we need chiral catalysts, which are catalysts that work only in specific orientations. They are rated by their "`enantiomeric excess"' (which is the success rate, right? roughly?...idk). This is called enantioselectivity, which allows us to create the specific enantiomers that we want. 

\chapter{October 18 - Harry Gray}

Forgot my battery :( check iPad for notes

\chapter{October 25 - Greg Fu}
\chapter{November 1 - David Tirrell}


\chapter{November 8 - Jack Beauchamp}
\chapter{November 15 - Bil Clemons}
\chapter{November 29 - Thomas Miller}
\chapter{December 6 - Rudy Marcus}

\end{document}
\documentclass[10pt]{report}
\usepackage{amsmath, amsthm, amssymb, tikz, mathtools, hyperref,enumerate}
\usepackage[margin=0.5in]{geometry}
\newcommand{\scinot}[2]{#1\times 10^{#2}}
\newcommand{\bra}[1]{\left<#1\right|}
\newcommand{\ket}[1]{\left|#1\right>}
\newcommand{\dotp}[2]{\left<#1\left.\right|#2\right>}
\newcommand{\rd}[2]{\frac{d#1}{d#2}}
\newcommand{\pd}[2]{\frac{\partial #1}{\partial#2}}
\newcommand{\norm}[1]{\left|\left|#1\right|\right|}
\newcommand{\abs}[1]{\left|#1\right|}
\newcommand{\expvalue}[1]{\left<#1\right>}
\newcommand{\rtd}[2]{\frac{d^2#1}{d#2^2}}
\newcommand{\pvec}[1]{\vec{#1}^{\,\prime}}
\let\Re\undefined
\let\Im\undefined
\DeclareMathOperator{\Re}{Re}
\DeclareMathOperator{\Tr}{Tr}
\DeclareMathOperator{\Im}{Im}
\newcommand{\ptd}[2]{\frac{\partial^2 #1}{\partial#2^2}}
\usepackage[labelfont=bf, font=scriptsize]{caption}
\everymath{\displaystyle}

\begin{document}

\title{Groupes de Sym\'etrie en Physique\\ Salle 66 Tu 1330 --- 1800\\ Denis Bernard et David Renard}
\author{Yubo Su}
\date{}

\maketitle
\tableofcontents

\chapter{Les notes du livre}

\section{Petites d\'emonstrations}

\begin{itemize}
    \item \textbf{Prop:} Soit $(\rho,V)$ une repr\'esentation finie de $G$. On peut munir $V$ d'un produit hermitien qui rend $(\rho,V)$ unitaire.

        \textbf{D\'em:} Soit $(x|y)_0$ un produit hermitien quelconque. Alors le produit
        \begin{align}
            (x|y)_1 &= \sum\limits_{g \in G}^{}\frac{1}{\abs{G}}\left( \rho(g)\cdot x | \rho(g) \cdot y\right)_0
        \end{align}
        rend $(\rho,V)$ unitaire.

    \item \textbf{Prop:} Une repr\'esentation irr\'eductible $(\rho,V)$ d'une groupe finie $G$ est de dimension finie.

        \textbf{D\'em:} Soit $x \in V$ et $W = \left\{ \rho(g) \cdot x, g \in G \right\}$. $W$ est stable par $\rho$ car $\rho(g_1) \cdot \left( \rho(g_2) \cdot x \right) = \rho(g_1g_2)\cdot x \in W$. Alors $W = V$ et $\dim V = \dim W = \dim G$.

    \item \textbf{Prop:} Soit $(\rho,V)$ une repr\'esentation finie de $G$. Soit $W \in V, \rho(g) \cdot W \in W$. Alors $V = W + W'$.

        \textbf{D\'em:} On peut munir $V$ d'un produit hermitien qui rend $(\rho,V)$ unitaire. Alors $W^\perp$ reste orthogonal \`a $W$ par
        \begin{align}
            \rho\left[ \rho \cdot \left( W^{\perp} \right) \perp \rho \cdot W \right]
        \end{align}
        alors $W^{\perp}$ est stable.

    \item \textbf{Prop:} Soit $T$ une op\'erateur d'entrelacement $(\rho,V)$ et $(\tau,W)$. Si $(\rho,V), (\tau,W)$ ne sont pas \'equivalentes $T = 0$. Si oui, $\dim (\mathrm{Hom}_G(V,W)) = 1$ o\`u $\mathrm{Hom}_G(V,W) = \left\{ c\mathrm{Id}_v, c \in \mathbb{C} \right\}$. C'est la \emph{lemme de Schur}.

        \textbf{D\'em:} Si ce sont pas \'equivalentes, $T$ n'est pas bijectif, $\ker(T) \neq \emptyset$ mais car il est irr\'eductible $\ker(T) = V, T = 0$. 
        
        S'il sont \'equivalentes, on note $T(\rho(g) \cdot v) = \rho(g)\cdot T(v) = \lambda \rho(g) \cdot v$ et alors $T(\forall v \in V) \sim \lambda v, T = \lambda \mathrm{Id}_v$. Alors, car $V$ est \'equivalente \`a $W$, $\mathrm{Hom}(V,V) \to \mathrm{Hom}(V,W)$ leurs dimensions sont \'egales aussi.

    \item \textbf{Ex. 2.2.4:} Un morphisme de groupe continue $\phi: \mathbb{R} \to GL(n,K)$ est differentiable et est de la forme $\phi(t) = \exp(tX)$ soit $X \in M_n(K)$.

        \textbf{D\'em:} $\phi'(t) = \frac{\phi(t + \delta) - \phi(t)}{\delta} = \phi(t)\left[ \frac{\phi(\delta) - \phi(0)}{\delta} \right]$ doit exister parce que $\phi$ est continu. L'appelons $X$, alors $\phi(t) = X\phi(t), \phi(t) = \exp(tX)$.

    \item \textbf{Prop:} Soit $h,e,f$ une ase pour $\mathfrak{sl}(2,\mathbb{C})$. Soit $\phi: \mathfrak{sl}(2,\mathbb{C} \to \mathfrak{gl}(V)$. Soit $\lambda$ la valeur propre d'un $v \in V$ sous $\phi(h)$. Alors $\phi(e[f]) \cdot v \sim \lambda + 2 [-4]$

        \textbf{D\'em:} C'est la m\^eme d\'emonstration pour les deux, alors
        \begin{align}
            \phi(h)\phi(e)\cdot v &= \phi(e)\phi(h) \cdot v + \phi\left( [h,e] \right) \cdot v\\
            &= \phi(e)\lambda \cdot v + 2\phi(e) \cdot v\\
            &= \left( \lambda + 2 \right)\phi(e) \cdot v
        \end{align}
\end{itemize}

\section{Groupes, Actions de groupe, repr\'esentations}

\subsection{Les groupes et les actions de groupe}

Une action de groupe est une morphisme $A: G \to \mathrm{Aut}(X)$, est une repr\'esentation est une morphisme $A: G \to GL(V)$ ou $X$ est une ensemble et $V$ un espace vectoriel. Quelques exemples des groupes:
\begin{itemize}
    \item $GL(V) = \mathrm{Aut}(V)$ --- Les bijections qui pr\'eservent la structure tel que $f(gh) = f(g)f(h)$.
    \item $U(V)$ --- Les bijections qui pr\'eservent le produit hermitien sur $\mathbb{C}^n = V$.
    \item $O(V)$ --- Les bijections q i pr\'eservent le produit scalaire $\expvalue{x,y} = x_iy_i$. Une sous-classe est les bijections $O(p,q)$ qui pr\'eservent la forme bilin\'eaire sym\'etrique $\expvalue{x,y} = x_1y_1 +\dots + x_py_p - x_{p+1}y_{p+1} -\dots -x_{p+q}y_{p+q}$.
    \item $SL(V)$ --- Le sous-espace de $GL(V)$ tel que $\det g = 1$. Aussi, $SO(V) = SL(V) \cap O(V)$.
    \item $Sp(V,\omega)$ --- Le produit $\omega(x,y)$ anti-sym\'etrique, tel que $\omega(x,y) = -\omega(y,x)$.
\end{itemize}

Le \emph{groupe de Heisenberg} est une group $H(V,\omega) = V \oplus \mathbb{R}$ ou $V = \mathbb{R}^{2n} = \begin{pmatrix} \vec{x}\\ \vec{y} \end{pmatrix} , \vec{x}, \vec{y} \in \mathbb{R}^n$. Une repr\'esentation matricielle de ce groupe est donn\'e par
\begin{align}
    M &= \begin{pmatrix} 1 & \vec{x} & \vec{y} & z\\0 & 1 & 0 & \vec{y}\\0 & 0 & 1 & -\vec{x}\\0 & 0 & 0 & 1 \end{pmatrix} 
\end{align}
ou $z \in \mathbb{R}$. 

\subsection{Les repr\'esentations}

Une \emph{repr\'esentation} est une $\rho$ tel que $\rho: G \to GL(V)$. Si $\rho: G \to U(V)$ on dit que $\rho$ est unitaire. La dimension d'une repr\'esentation est la dimension de l'espace vectoriel, $\dim \rho = \dim V$.

Si une $W \in V$ \'existe tel que $\forall g \in G: \rho(g) \cdot W \in W$, $W$ est invariant par $\rho$ et on appelle $(\rho_\omega, w)$ une \emph{sous-repr\'esentation}. On dit que $(\rho,V)$ est \emph{irr\'eductible} si les seules sous-repr\'esentations de $(\rho,V)$ sont $V$ et $\emptyset$. On dit que $(\rho,V)$ est compl\`etement r\'eductible si $(\rho,V) = \oplus_i (\rho_i,w_i)$ avec $w_i$ irr\'eductible.

Soit $(\rho,V), (\tau,W)$ deux repr\'esentations du groupe $G$. On appelle $T: V\to W$ un \emph{op\'erateur d'entrelacement} si $T(\rho(g)\cdot v) = \tau(g) \cdot T(V)$. Un op\'erateur d'entrelacement est aux repr\'esentations ce qu'une changement de base est aux espaces vectoriel. Si $T$ \'existe, on dit que $(\rho,V)$ est \emph{isomorphe} \`a $(\tau,W)$, et $\tau(g) = T \cdot \rho(g) \cdot T^{-1}$. 

Aussi, il faut noter que $\ker(T)$ est invariant par $\rho$ et que $\mathrm{Im}(T)$ est invariant par $\tau$. Une d\'emonstration de ces deux point suit. Notons que $T(\rho(g) \cdot v) = \tau(g) \cdot T(v) = 0$. Aussi, appelons $v'$ l'\'el\'ement $\rho(g) \cdot v \in V$, alors $\tau(g) \cdot T(v) = T(v') \in \mathrm{Im}(T)$.

On appelle une \emph{repr\'esentation contragr\'ediente} $\tilde{\pi}$ tel que $(\tilde{\pi}(g) \cdot \lambda)(v) = \lambda(\pi(g)^{-1} \cdot v)$ ou $\lambda \in V^*, v \in V$ ou $V^*$ est le dual de $V$.

Car la r\'eduction compl\`ete n'est pas unique, on introduit la \emph{d\'ecomposition canonique}. Soit $\delta$ une ``classe'' de repr\'esentations irr\'eductible de $V$. Alors $V_\delta = \oplus \left\{ V_i \in \delta \right\}, V = \oplus V_\delta$. On note aussi que $\dim \mathrm{Hom} (V_\delta, V)$ est la multiplicit\'e des $i$ dans $\delta$.

On peut d\'emonstrer \c{c}a par cet argument. Clairement $\mathrm{Hom}(V_\delta, \oplus_i V_i) = \mathrm{Hom}(V_\delta,\oplus_{i\in\delta}V_i = \oplus_{i\in \delta}\mathrm{Hom}(V_\delta,V_i)$. Alors car $\mathrm{Hom}(V_\delta, V_i) = 1$ si $i \in \delta$, $\dim \mathrm{Hom}(V_\delta,V)$ est \'egale \`a la multiplicit\'e des $i$ dans $\delta$. 

\subsection{Les groupes topologiques}

Une \emph{groupe topologique} est un groupe tel que les op\'erations de la multiplication et de l'inverse sont continues. On appelle un groupe topologique \emph{localement compact} s'il est approximativement dense et compact sur une aire finie.

Pour ces groupes, il faut introduire la mesure de Haar, qui est invariant par translation \`a gauche ou \`a droite. Seulement si $G$ est compl\`etement compact, gauche $=$ droite. Par exemple, si $G = U(1) = z \in \mathbb{C}$ tel que $\abs{z} = 1$, la mesure de Haar est $\frac{dz}{2\pi iz}$. 

\section{Groupes lin\'eaire et leurs alg\`ebre de Lie}

Soit $M_n(\mathbb{K})$ ou $\mathbb{K} = \mathbb{R}$ ou $\mathbb{C}$ les matrices de dimension $n$ sur $\mathbb{K}$. D\'efinit $\abs{M_n} = \mathrm{sup} \frac{\abs{M_n x}}{\abs{x}}$ ou $x \in \mathbb{K}^n$ et aussi $\exp M_n = \sum\limits_{n=0}^{\infty}\frac{(M_n)^n}{n!}$. On appelle un groupe lin\'eaire un sous-groupe de $GL(n,\mathbb{K}$.

\subsection{Alg\`ebre de Lie}

On commence avec l'id\'ee d'un espace tangent. $g$ est tangent \`a $G$ en l'identit\'e si $\forall x \in g$, il existe $a(t): \mathbb{K} \to M_n(\mathbb{K})$ (la co-domaine d'$a(t)$ est $G$) tel que $a(0) = \mathrm{Id}$ et $a'(0) = x$.

Alors si on a un crochet $[X,Y]$, le sous-algebre de Lie de $L$ est stable par ce crochet, $[L,L] \to [L]$ et satisfait $[X,[Y,Z]] +[Y,[Z,X]] +[Z,[X,Y]] = 0$ l'identit\'e de Jacobi. L'alg\`ebre de lie de $GL(\mathbb{K}^n)$ est $gl(n,\mathbb{K})$. L'espace tangent en l\'identit\'e d'un groupe $G$ est un alg\`ebre de Lie, et en faite c'est l'alg\`ebre de Lie unique.

On appelle cet alg\`ebre de Lie les g\'en\'erateurs infinit\'esimaux car $\exp Cg \in G \forall C\in\mathbb{R}$, ou autrement dit $g$ engendre $G$ par l'exponentiel. On dit que $G$est connexe si $G$ est engendr\'e par toute voisinage $U$ des \'el\'ements dans $g$.

Soit $f: G \to H$ une morphisme de groupes, alors $\phi:h\to h$ est aussi une morphisme d'alg\`ebres de Lie tel que $\phi = df_{Id}$. De plus, $f$ est localement bijectif si $\phi$ est isomorphe.

\subsection{Rev\^etements et repr\'esentations projectives}

On appelle $\rho: \tilde{G} \to G$ qui est localement bijectif une \emph{rev\^etement}. Et si une morphisme d'alg\`ebre de Lie $\phi: g\to h$ existe, il existe une rev\^etement $\rho: \tilde{G} \to G$ tel que $f:\tilde{G} \to H$ est une morphisme de groupes.

On appelle la complexification $E_c = E \otimes \mathbb{C}$ d'un espace r\'eel $E$ tel que $\dim E = \dim_{\mathbb{C}} E_{\mathbb{C}}$. 

On appelle une repr\'esentation adoint la repr\'esentation de $X$ tel que $X \cdot x = xXx^{-1}$ ou $x \in g, X \in G$. Le differentiel de $\mathrm{Ad}$ est $\mathrm{ad}$ tel que $\mathrm{ad}(X) \cdot Y = \left[ X,Y \right]$, alors $\mathrm{ad}$ est une repr\'esentation de $g$ dans elle-m\^eme.

Finalement, les repr\'esentation projectives sont les fonctions $\rho: G \to GL(V)$ tel que $\rho(g) \rho(h) = c(g,h) \rho(gh)$. Notons que si $c(g,h) = 1 \forall g,h$ qu'il s'agit d'une repr\'esentation. C'est facile de montre que $c(g_1, g_2)c(g_1g_2,g_3) = c(g_2,g_3)c(g_1,g_2g_3)$ est satisfait par $c(g,h)$, la \emph{relation de cocycle}. En plus, soit $\tilde{G}= G \times A: (g,z)(g',z') = (gg', zz' c(g,g')$, alors $\tilde{\rho}: \tilde{G} \to GL(V)$, ou $\tilde{\rho}(g,z) = z\rho(g)$, est une repr\'sentation.

\section{$\mathfrak{sl}(2,\mathbb{C}),\mathbf{SU}(2), \mathbf{SO}(3)$ }

\subsection{$\mathbf{SU}(2)$ comme rev\^etement de $\mathbf{SO}(3)$}

Notons que $\mathbf{SU}(2) = \left\{ a \in \mathbf{GL}(2,\mathbb{C}), a^\dagger a = 1, \det a = 1 \right\}$ et $\mathbf{SO}(3) = \left\{ a \in \mathbf{GL}(3,\mathbb{R}), a^\dagger a = 1, \det a = 1 \right\}$. Alors leurs alg\`ebres de lie sont $\mathfrak{su}(2) = \left\{ x \in \mathfrak{gl}(2, \mathbb{C}), x^\dagger + x = 0, \mathrm{Tr} x = 0 \right\}$ et $\mathfrak{so}(3) = \left\{ x \in \mathfrak{gl}(3,\mathbb{R}), x^\dagger + x = 0, \mathrm{Tr} x = 0 \right\}$. Plus explicitement, 
\begin{align}
    \mathbf{SU}(2) &= \left\{ \begin{vmatrix}\alpha & -\beta^*\\`b & \alpha^*\end{vmatrix}, \abs{\alpha}^2 + \abs{\beta}^2 = 1 \right\}\\
    \mathfrak{su}(2) &= \left\{ \begin{vmatrix}iz & -y + ix \\ y + ix & -iz \end{vmatrix}, x,y,z \in \mathbb{R}\right\}\\
    \mathfrak{so}(3) &= \left\{ \begin{vmatrix}0& -z & y\\z & 0 & x\\-y & -x & 0\end{vmatrix},x,y,z \in \mathbb{R} \right\}
\end{align}

Alors on trouve une base pour $\mathfrak{su}(2)$, \`a savoir
\begin{align}
    \left\{ \begin{bmatrix} 0 & i\\i & 0 \end{bmatrix}, \begin{bmatrix} 0 & -1\\1 & 0 \end{bmatrix} , \begin{bmatrix} i & 0\\0 & -i \end{bmatrix}  \right\}
\end{align}

On peut trouver un produit tel que $\omega(I,I) = \omega(J,J)= \omega(K,K) = 1$, \`a savoir $\omega(X,Y) = -\frac{1}{8}\mathrm{Tr}\left( \mathrm{ad}(X)\mathrm{ad}(Y) \right)$. Alors $\left( \mathfrak{su}(2), -\frac{1}{8}\omega \right)$ est isomorphe \`a $\mathrm{R}^3$.

Donc on a la repr\'esentation adjointe $\mathbf{SU}(2) \to O(\mathfrak{su}(2), \omega)$, et car $O(\mathfrak{su}(2),\omega) \simeq \mathbf{SO}(3)$ on trouve la repr\'esentation adjointe $\mathrm{ad}: \mathfrak{su}(2) \to \mathfrak{so}(3)$ et alors $\mathbf{SU}(2)$ est une rev\^etement de $\mathbf{SO}(3)$ ($\mathfrak{su}(2) \sim \mathbb{R}^3 \sim \mathfrak{so}(3)$. 

\subsection{Repr\'esentation de $\mathfrak{sl}(2,\mathbb{C})$}

Rappelons que $\mathfrak{su}(2)$ \'etait param\'etris\'e par $(x,y,z) \in \mathbb{R}$, alors la complexification $\mathfrak{su}(2) \otimes \mathbb{C}$ est appel\'ee $\mathfrak{sl}(2,\mathbb{C})$. Alors c'est \'equivalent d'\'etudier $\mathfrak{su}(2)$ sur l'espace $\mathbb{C}$ d'une repr\'esenttation $\mathrm{R}$-lin\'eaires ou $\mathfrak{sl}(2,\mathbb{C})$. 

On introduit une base pour $\mathfrak{sl}(2,\mathbb{C})$ \`a savoir $\begin{pmatrix} 1 & 0\\0 & -1 \end{pmatrix} , \begin{pmatrix} 0 & 1\\0 & 0 \end{pmatrix} , \begin{pmatrix} 0 & 0\\1 & 0 \end{pmatrix} $ sur $\mathbb{C}$ (\c{c}a veut dire les coefficients complexes). Les notons $h,e,f$. On calcule $\left[ h,e \right] = 2e, \left[ h,f \right] = -2f, \left[ e,f \right] = h$.

Soit $\phi: \mathfrak{sl}(2,\mathbb{C}) \to \mathfrak{gl}(V)$. On trouve que $\phi(e), \phi(f)$ agit comme les op\'erateurs d'\'echelle. Notons $V_\lambda$ le sous-espace de $V$ \`a valeur propre $\lambda$. Car $\phi$ est de dimension finie, il y a une nombre finie de $V_\lambda$ non nuls. On peut calculer simplement que $\phi(e) \cdot v_k = k(\lambda_0 - k + 1)v_{k - 1}$. Donc $V_{\lambda_0 + 1} = \emptyset$ dans cette \'echelle. Alors, si $V$ est irr\'eductible on trouve que $\dim_{\mathbb{C}} V = \lambda_0 + 1$. Les $V_i$ engendrent $V$ qui est stable par $\mathfrak{sl}(2,\mathbb{C})$. Appelons cette repr\'esentation $(\phi_n, \mathbb{C}^{n+1})$.

Une exemple de ce groupe est les polyn\^omes en deux variables $\mathbb{C}[z_1, z_2]$. Definissons $D_e, D_f, D_h = -z_2\partial_1, -z_1\partial_2, z_2\partial_2 - z_1\partial_1$; c'est la repr\'esentation de $\mathfrak{sl}(2, \mathbb{C})$ dans $\mathbb{C}[z_1, z_2]$. C'est isomorphe \`a $(\phi_n, \mathbb{C}^{n+1})$ dans la base $z_1^k, z_2^{n-k}$.

Aussi, $\mathbf{SL}(2,\mathbb{C})$ agit naturellement aussi, $\rho(g): (z_1, z_2) \to g^{-1}(z_1, z_2)$. Si on aux repr\'esentations de $\mathbf{SU}(2) \in \mathbf{SL}(2, \mathbb{C})$, et alors $\mathbf{SU}(2)$ est isomorphe \`a $(\phi, \mathbb{C})_n$ qu'on appele $(\pi_n, V_n)$ (seulement la repr\'esentation du sous-groupe $\mathbf{SU}(2)$). 

\subsection{Les harmoniques sph\'eriques, l'op\'erateur de Casimir}

On appelle les \emph{harmonique sph\'eriques} les solutions de l'\'equation $\nabla^2 f = 0, f \in \mathcal{C}^2(\mathbb{R^3},\mathbb{C})$.

Notons $\Omega_\rho = \frac{1}{4}\left( \rho(I)^2 + \rho(J)^2 + \rho(K)^2 \right)$ ($I,J,K$ sont les \'el\'ements de la ase) l'\emph{op\'erateur de Casimir}. On trouve que $\nabla^2 = \partial_r^2 + \frac{2}{r}\partial_r + \frac{1}{r^2}\Omega_\rho$, ou $\Omega_\rho$ dans les coordon\'ees sph\'eriques est \'egal \`a $\partial_\theta^2 + \cos\theta\partial_\theta + \frac{1}{\sin^2\theta} \partial_\phi^2$.

Notons $P_l(X,Y,Z) = X^\alpha Y^\beta Z^\gamma, \alpha + \beta + \gamma = l$ les monomes de puissance total $l$, on voit que la dimension de $P_l$ est $\sum\limits_{k=0}^{l}(k+1) = \frac{(l+1)(l+2)}{2}$. Notons aussi $H_l$ le sous-espace de $P_l$ harmoniques. 

On examine au debut les valeurs propres de $H_l$; appelons $E = r\partial_r$ l'op\'erateur d'Euler, alors $\Omega_\rho = r^2\nabla^2 - E^2 - E$. C'est clair que les \'el\'ements de $P_l$ sont des fonctions propres de $E$ \`a valeurs propres $l$. Pour les \'el\'ements de $H_l$, $\nabla^2$ n'agit pas \`a eux et donc les $H_l$ ont les valeurs propres $-l^2 - l = -l(l+1)$.

On voit aussi que $H_l$ est une repr\'esentation irr\'eductible de $\mathbf{SO}(3)$ car $P_l = H_l \oplus r^2 P_{l-2}$. On voit la derni\`ere \'egalit\'e en examinant $\nabla^2 P_l$, parce que si ce n'est pas \'egal \`a $0$ c'est envoy\'e vers $P_{l-2}$, et si c'est \'egal \`a $0$ c'est une \'el\'ement de $H_l$. Ca montre que $P_l = H_l \oplus r^2 H_{l-2} \oplus \dots$, une relation qui montre le lien \`a $\mathbf{SO}(3)$. 

\section{Sym\'etries en mecanique quantique}

\subsection{Les translations et rotations}

Soit $T_{\alpha}$ l'op\'erateur de translation par $\vec{a}$, donc $T_{\vec{a}} = \exp\left( -i \vec{a} \cdot \vec{P} \right), \left[ P_i, P_j \right] = 0$. Aussi, pour les rotations, $R_{\vec{n};\alpha} = \exp\left( -i \alpha \vec{n} \cdot \vec{J} \right), \left[ J_i, J_j \right] = i\epsilon_{ijk}J_k$.

On note que l'espace $L^2(\mathrm{R}^3) = L^2(\mathrm{R}_+ \otimes L^2(S^2)$, o\`u $S^2$ est le surface du sph\`ere du rayon $1$. De plus, $Y_l^m$ les harmoniques sph\'eriques forment une base de $S^2$. Alors chaque $Y_l^m$ supporte une repr\'esentation irr\'eductible $R_l^{SO(3)}$ de dimension $2l+1$, enfin
\begin{align}
    L^2(\mathbb{R}^3) = \bigotimes_{l=0}^\infty L^2(\mathbb{R}_+) \otimes R_l^{SO(3)}
\end{align}

Les harmoniques sph\'eriques sont des polyn\^omes de degr\'es $l$. La dimension de l'espace $\dim P_l = \frac{(l+1)(l+2)}{2}$, mais car $r^2Y_{l-2} \in Y_l$ forme une sous-repr\'esentation donc chaque $Y^m_l$ soi-m\^eme supporte une dimension $2l+1$. 

\subsection{Repr\'esentations proj\'ectives}

Si on a une $g$ qui agit sur $\mathcal{H}$ par $U(g)$, on a deux possibilit\'es pour $U(g_1g_2)$. Si $U(g_1g_2) = U(g_1)U(g_2)$, c'est une repr\'esentation toute simple, on peut examiner l'alg\`ebre de Lie.

Mais on peut aussi avoir $U(g_1g_2) = U(g_1)U(g_2)\Omega(g_1, g_2)$. Si $\Omega(g_1, g_2) = \exp\left( i\left( \eta(g_1) + \eta(g_2) - \eta(g_1g_2) \right) \right)$ on trouve un coycle trivial, et on peut simplement red\'efinir $U \to \hat{U}(g) = U(g)e^{i\eta(g)}$ qui devient une repr\'esentation. Sinon, c'est ce qu'on appelle une \emph{repr\'esentation projective}.

Une exemple est $SO(2) \simeq U(1)$. Posons $D_j(e^{i\pi x} = e^{2\pi ijx}$. Pour $j = \frac{1}{2}$ ce n'est pas une application. Choissons $\hat{D}_j(e^{i 2\pi x} = e^{2\pi ij \overline{x}}$ o\`u $\overline{x} = x - E(x)$. Comme \c{c}a on trouve que
\begin{align}
    \hat{D}_j(x)\hat{D}_j(y) = \underbrace{e^{i2\pi\left( E(x+y) - E(x) - E(y) \right)j}}_{\Omega(x,y)}\hat{D}_j(x+y)
\end{align}

Car $E(x)$ n'est pas une application sur $U(1)$ on a une repr\'esentations projective, $\Omega(x,y)$.  n'est pas une application sur $U(1)$ on a une repr\'esentations projective, $\Omega(x,y)$. 

En g\'en\'eral, pour $j = \frac{p}{q}$ on trouve que $\Omega(x,y)$ est une $q$-i\`eme racine de l'unit\'e. Donc si on extend $U(1) \to U(1) + \mathbb{Z}(q) \simeq \mathbb{R}$ c'est une rev\^etement et une repr\'esentation.

Une autre exemple est $SO(3) \simeq SU(2)$. Les \'el\'ement de $SO(3)$ sont d\'etermin\'ees par $(\vec{n},\alpha)$ o\`u $\abs{\vec{n}}^2 = 1, \alpha \in [0,\pi]$ (parce que $(\vec{n},\alpha) = (-\vec{n}, -\alpha)$). Aussi, $SU(2)$ agit sur les matrices $X$ par conjugaison $X \to UXU^\dagger$, qui est une rotation ($\det(UXU^\dagger = \det X$, la d\'efinition d'une rotation). 

Alors si on \'ecrit $X = \vec{x} \cdot \vec{\sigma}$ (rappelons que $X$ est une matrice $2 \times 2$), on trouve que $UXU^\dagger = (R_u \cdot \vec{x}) \cdot \vec{\sigma}$. Donc il y a une correspondence entre $U, R_u$ jusqu'\`a une signe. ou \'egalement il y a une morphisme entre $U \in SU(2), R_u \in SO(3)$. Le noyau de cette morphisme est $\pm 1$ car $\pm U$ sont envoy\'es au m\^eme $R_u$. Donc $\frac{SU(2)}{SO(3)} \simeq \mathbb{Z}(2)$. Il faut noter que $SU(2)$ est une repr\'esentation du spin $\frac{1}{2}$ est que $SO(3)$ est la repr\'esentation projective de ceci.

\subsection{Les sym\'etries dans les dynamiques}

Si un groupe est un groupe de sym\'etrie des dynamiques, $\left[ H, U(g) \right] = 0 \forall g \in G$. Le th\'eor\`eme de Ehrenfest dit donc qu'une telle sym\'etrie correspond \`a une loi de conservation. Autrement dit, les d\'eg\'eneracies correspond aux sym\'etries.

Par cons\'equence de le lemme de Schur, on voit que si une repr\'esentation $D$ est irr\'eductible, alors toutes endomorphismes $S$ qui commutent avec $D$ sont une multiple de l'identit\'e.

Soit alors $H = \bigoplus_\lambda \bigoplus_{i=1}^{m_\lambda}R_\lambda^{(i)}$ avec $\lambda$ qui fait ref\'erence aux repr\'esentations differentes de $G$ et $m_\lambda$ la multiplicit\'e de chaque repr\'esentation. Alors $U\left( g\right)$ est bloc-diagonal dans chaque espace $R_\lambda^{(i)}$.

On peut chosir la base $W_\lambda = \left\{ e_\lambda^{i} \right\}, i \in [0, m_\lambda]$. Pour chaque $\lambda$, $e_\lambda^i \otimes R_\lambda \simeq R_\lambda^{(i)}$ et alors on peut aussi \'ecrire $H = \bigoplus_\lambda W_\lambda \otimes R_\lambda$, la d\'ecomposition canonique.

On voit que $H$ est bloc-diagonal sur les blocs $W_\lambda$, C'est naturel suivant le lemme de Schur, car $\left[ U,H \right] = 0$, et il n'y a ue des \'el\'ements liant les $W_\lambda$, les repr\'esentations \'equivalentes. 

On sait aussi que les \'etats de $\lambda$ different ne passe aux autres $\lambda$, ce qu'on appelle les r\`egles de s\'election. Apr\`es \c{c}a, on peut finalement \'ecrire l'hamiltonien comme $H = h_\lambda W_\lambda \otimes R_\lambda$, $h_\lambda$ qui agit sur le sous-espace $W_\lambda$. 
\section{Les groupes de Lorentz, de Poincar\`e, et l'\'equation de Dirac}

\subsection{Le groupe et la transformation de Lorentz}

Le \emph{groupe de Lorentz} est $\mathbf{O}(1,3)$. Notons $\eta = \mathrm{diag}(-,+,+,+)$ la \emph{m\'etriuqe Minkowskienne}, et soit $X,Y$ les \emph{quadri-vecteurs} tel que $X \cdot Y = \eta(X,Y)$.

La transformation de Lorentz est $\Lambda$ tel que $\eta(\Lambda X, \Lambda Y) = \eta(X, Y)$ o\`u $(\Lambda X)^\mu = \Lambda^\mu_\nu X^\nu$. Par l'unitarit\'e $\det \Lambda = \pm 1$ et donc $(\Lambda_0^0)^2 \geq 1$. Si $\det\Lambda = 1$ la transformation de Lorentz pr\'eserve l'orientation; si $\Lambda_0^0 \geq 1$ aussi la transformation pr\'eserve la fleche du temps. Chaque signe de $\det \Lambda, \Lambda_0^0$ correspond \`a un groupe simplement connexe. $\Lambda_>^+$ contient la parit\'e et $\Lambda_<^-$ contient l'op\'erateur du renversement de temps.

Les rotations sont comme $\mathrm{diag}(1, R_z)$ autour l'axe $z$ et les boost sont $\begin{pmatrix} \cosh \alpha & 0 & -\sinh \alpha \\0 & \mathbb{I}_2 & 0\\\sinh \alpha & 0 & \cosh \alpha \end{pmatrix}$; les deux sont $\in \Lambda_>^+$.

On peut les d\'evelopper comme $\Lambda_\nu^\mu = \delta_\nu^\mu + \epsilon \omega_\nu^\mu + O(\epsilon^2)$ et on trouve que
\begin{align}
    \omega_\rho^\mu \eta_{\mu\sigma} + \eta_{\rho\nu} \omega_\sigma^\nu = 0
\end{align}
o\`u $\omega_\mu\nu = \eta_{\mu\sigma}\omega_\nu^\sigma$. Donc $\omega_{\mu\nu}$ est antisym\'etrique et engendre les transformation de Lorentz. Alors la dimension du groupe de Lorentz et $6$ car la dimension des matrices antisym\'etrique $4 \times 4$ est $6$. 

Notons $J_{x,y,z}, K_{x,y,z}$ les g\'en\'erateurs des rotations et des boosts. Les nouvelles commutateurs (on connait d\'ej\`a les commutateurs des $J_{x,y,z}$) sont
\begin{align}
    \left[ J_x, K_y \right] &= iK_z\\
    \left[ K_x, K_y \right] &= iJ_z
\end{align}

On peut \'ecrire aussi $J^{[\sigma\rho]}_\mu \nu = i\left( \delta_\mu^\sigma \delta_\nu^\rho  -\delta^\rho_\mu \delta^\sigma_\nu\right)$ (car $J^{\sigma \rho} = i\left( X^\sigma\partial^\rho - X^\rho\partial^\sigma \right)$). La base antisym\'etrique est contenue dans cette expression.

\subsection{$\mathbf{SL}(2,\mathbb{C})$ comme rev\^etement de $\mathbf{SO}(1,3)$}

Soit la base $\sigma_0 = 1, \sigma_i$ les matrices de Pauli, et $\sigma = (\sigma_0, \vec{\sigma})$ la quadri-vecteur des matrices de Pauli. On associe \`a chaque quadri-vecteur $X \to s_\mu X^\mu$ une matrice $2 \times 2$ qui est hermitienne $\det (\sigma_\mu X^\mu) = X \cdot X = 1$.

$\mathbf{SL}(2,\mathbb{C})$ agit sur $X$ par $X \to UXU^{-1}$. Chaque $\pm U\in \mathbf{SL}(2,\mathbb{C})$ est associ\'e \`a une transformation $\Lambda_\mu$, et alors les transformations de Lorentz ($\mathbf{SO}(1,3)$) correspond \`a une matrice de $\mathbf{Sl}(2,\mathbb{C})$ jusqu'\`a une signe. Autrement dit, $\mathbf{SL}(2,\mathbb{C})/\mathbf{SO}(1,3) \simeq \mathbb{Z}_2$. 

\subsection{Le groupe de Poincar\'e}

Le groupe de Poincar\'e est simplement $\mathbf{O}(1,3) \otimes \mathbb{R}^4$ avec $\vec{a} \in \mathbb{R}^4$. Soit $T_{\vec{a}}$ un op\'erateur de translation. On trouve que $\Lambda T_{\vec{a}}\Lambda^{-1} = T_{\lambda \cdot \vec{a}}$, et alors ils ne commutent pas.

Soit $M(\Lambda, \vec{a}), M(\Lambda', \pvec{a})$ des \'el\'ements du groupe de Poincar\'e, on a $M(\Lambda, a)M(\Lambda', \pvec{a}) = M(\Lambda \Lambda', \vec{a} + \Lambda \cdot \pvec{a})$ repr\'esent\'e par $M(\Lambda a) = \begin{bmatrix} \Lambda & \vec{a}\\0 & 1 \end{bmatrix} $. 

Les relations de commutation entre $P_\mu$ les g\'en\'erateurs de translation et $J^{[\rho \sigma]}$ sont
\begin{align}
    \left[ P_\mu, P_\nu \right] &= 0\\
    \left[ J^{[\rho \sigma]}, P_\mu \right] &= -i \left( \delta^{\sigma}_\nu P^{\rho} - \delta^{\rho}_\nu P^{\sigma} \right)
\end{align}

\subsection{Les repr\'esentations du groupe de Poincar\'e}

On construit les repr\'esentations du groupe Lorentz par complexification $\mathbf{SO}(1,3)_{\mathbb{C}} = \mathbf{SL}(2)_{\mathbb{C}} \oplus \mathbf{SL}(2)_\mathbb{C}$. Si on \'ecrit $M_i, N_i = \frac{1}{2}\left( J_i \pm iK_i \right)$ on trouve les relations de commutation
\begin{align}
    \left[ M_j, M_k \right] &= i\epsilon_{jkl}M_l\\
    \left[ N_j, N_k \right] &= i\epsilon_{jkl}N_l\\
    \left[ M_j, N_k \right] &= 0
\end{align}
et donc $M_j, N_k$ engendre les deux sous-espaces de $\mathfrak{sl}(2)_{\mathbb{C}}$. 

Rappelons que si $\rho_v, \rho_w$ sont deux repr\'esentations de $\mathfrak{sl}(2)_{\mathbb{C}}$, une repr\'esentation de $\mathfrak{so}(1,3)_{\mathbb{C}}$ est tel que
\begin{align}
    \rho_{v \otimes w}\begin{pmatrix} M_i\\N_j \end{pmatrix} &= \begin{pmatrix} \rho_v(M_i) \otimes 1\\ 1 \otimes \rho_w (N_j) \end{pmatrix} 
\end{align}

Les $\rho_v, \rho_w$ sont index\'ees par $n$ qui est $2\times$ le spin (rappelons que $\phi(e,f,g)$ changent la valeur propre par $2$ et donc les valeurs propres paires et impaires restent paires et impaires respectivement). Alors les repr\'esentations du spin sont $\left( \frac{m}{2}, \frac{n}{2} \right)$ sur l'espace $\mathbb{C}^{m+1} \otimes \mathbb{C}^{n+1}$ les deux spins.

\subsection{L'\'equation de Dirac}

Dans les traitements traditionels de l'\'equation de Dirac on trouve qu'elle d\'ecrit ``\'etrangement'' une particule de spin une-demie, et on commence avec les axiomes de la m\'ecanique quantique. Ici, on part de l'autre direction, et on montre que selon les r\`egles de la th\'eorie des groupes l'\'equation de Dirac est tout naturelle, une cons\'equence de la covariance de $\chi$ par les boosts.

On consid\`ere un spineur de Dirac $\chi = \begin{pmatrix} \chi_R\\ \chi_L \end{pmatrix} $ qui \'existe dans l'espace $\left( \frac{1}{2},0 \right)\otimes \left( 0, \frac{1}{2} \right)$. Ce spineur d\'epend sur le r\'ef\'erentiel et l'impulsion, donc on le note $\chi(P)$ un fonction de l'impulsion. Notons $P^*$ l'impulsion dans le r\'ef\'erentiel du centre de masse, alors $\chi_L(P^*) = \chi_R(P^*)$ par la parit\'e dans le r\'ef\'erentiel du CM.

On change le r\'ef\'erentiel comme
\begin{align}
    \chi(P) &= \exp\left( \pm \frac{\alpha}{2}\vec{n} \cdot \vec{\sigma} \right)\chi_{R,L}(P^*)
\end{align}

Si on combine les deux \'equations, on trouve (rappelons que $\chi_R(P^*) = \chi_L(P^*)$) que
\begin{align}
    \exp\left( -\alpha \vec{\nu} \cdot \vec{\sigma} \right)\chi_R(P) &= \chi_L(P)\\
    \exp\left( \alpha \vec{\nu} \cdot \vec{\sigma} \right)\chi_R(P) &= \chi_R(P)\\
    \exp\left( \alpha \vec{\nu} \cdot \vec{\sigma} \right) &= \cosh \alpha + (\sinh\alpha) \vec{n} \cdot \vec{\sigma}\\
    \begin{pmatrix} -m & p^0 + \vec{p} \cdot \vec{\sigma}\\
        p^0 - \vec{p} \cdot \vec{\sigma} & -m\end{pmatrix} \begin{pmatrix} \chi_R(P)\\ \chi_L(P) \end{pmatrix}  &= 0
\end{align}

Cette \'equation est dans la base de Weyl; les matrices de Dirac se retrouvent comme $\gamma_0 = \begin{pmatrix} 0&1\\1&0 \end{pmatrix} , \gamma_j = \begin{pmatrix} 0 & \sigma_j\\ - \sigma_j & 0 \end{pmatrix} $ et finalement 
\begin{align}
    (\gamma_\mu P^\mu - m) \chi(P) &= 0
\end{align}
l'\'equation de Dirac.

\subsection{Les repr\'esentations du groupe de Poincar\'e}

C'est un peu trop difficile de considerer directement les repr\'esentations du groupe de Poincar\'e comme avant, alors on consid\`ere d'abord les op\'erateurs de Casimir. Rappelons que tous sous-espaces irr\'eductibles ont la m\^eme valeur propre sous l'action de l'op\'erateur de Casimir.

Le premier tel op\'erateur est $P^2 = P_\mu P^\mu$ avec valeur propre $m^2$. Les deux classes de l'espace sont lesquelles avec $m^2 = 0$ et $m^2 \neq 0$.

Le deuxi\`eme tel op\'erateur est un peu plus difficile \`a trouver, c'est $W_\mu W^\nu, W^\nu = \frac{1}{2}\epsilon^{\mu \nu \rho \sigma}P_\nu J_{[\rho \sigma]}$, ou \'egalement $W_0 = -\vec{P} \cdot \vec{J}, \vec{W} = -P_0\vec{J} + \vec{P} \times \vec{K}$.

Quelles sont les valeurs propres de cet op\'erateur? Si $m^2 \neq 0$ on sait que chaque op\'erateur de la m\^eme masse $m$ est li\'ee par une transformation de Lorentz. Alors on peut toujours trouver le r\'ef\'erentiel o\`u l'impulsion est nulle et consider que ce r\'ef\'erentiel. Dans ce r\'eferentiel, $W_0 = 0, \vec{W} = -m\vec{J}$ et alors car les valeurs propres sont ind\'ependantes du r\'ef\'erentiel on trouve que les valeurs propres de ce deuxi\`eme op\'erateur sont $s(s+1)$. Donc pour les particules massives on voit qu'il y a une base $\ket{m; l, s}$ de la masse et du spin.

Pour les particules non-massives on trouve que le deuxi\`eme op\'erateur est de valeur propres $0$ par les \'experiences, et dans ce cas on peut trouver un r\'ef\'erentiel o\`u $W = \lambda P$ ou $\lambda$ est le helicit\'e. La base pour ces particules est donc $\ket{p, \lambda}$ avec $p$ l'impulsion.

\section{Invariance Conforme en Physique}

\subsection{Les transformations conformes en physique classique en 2D}

On repr\'esente $(x,y) \to z \in \mathbb{C}$. Une trannfsormation conforme est un fonctoion holomorph qui ob\'eit la pr\'eservation des argle tel que $f(x,y) = f(z)$ et pas $f(z,\overline{z})$. Autrement dit, soit $z_0 + \delta z, z_2 + \delta z_2 = z_1, z_2$ respectivement, alors l'angle entre $\delta z_1, \delta z_2$ est eg\`ale \`a laquelle entre $\delta w_1, \delta w_2$. Au prerier ordre donc, $f(z)$ agit comme les rotation et les dialation qui varient de point \`a point.

Les solutions de $\nabla^2 \phi(x,y) = 0$ sont les fonctions harmoniques. Notons $\nabla^2 = \partial_x^2 + \partial_y^2 = 4\partial_z \partial_{\overline{z}}$. Toutes fonctions harmonique se d\'ecomposent comme $\phi(z,\overline{z}) = \phi(z) + \phi(\overline{z})$, deux parties conforme et anti-conforme. Dans la m\'ecanique des fluids, si la vorticit\'e $\vec{\nabla} \times \vec{u} = 0$ et la fluide est incompressible $\vec{\nabla} \cdot \vec{u} = 0$ alors $\nabla^2 u = 0$ et le champs de v\'elocit\'e est harmonique.

\subsection{En $d$ dimension}

Appelons $g_{\mu \nu}$ la m\'etriue tel que $ds^2 = g_{\mu\nu}dx^\mu dx^\nu$. Sous un diff\'eomorphisme la metrique se transforme comme $\hat{g}_{\mu\nu} = \pd{x^\sigma}{y^\mu}\pd{x^\rho}{y^\nu}g_{\sigma\rho}$. Si ce diff\'eomorphisme est conforme on trouve $g^{\mu\nu} = e^{\phi(x)}g_{\sigma\rho}$ o\`u $\phi(x)$ est le facteur de dialation local.

On condi\`ere les transformations infinit\'esmmales $x^\mu \to x^\mu + \epsilon \xi^\mu(x)$. Alors l'invariance dit apr\`es quelques calculs que
\begin{align}
    \partial_\mu \xi_\nu + \partial_\nu \xi_\mu &= (\delta \phi)\delta_{\mu \nu}\\
    d\left( \delta \phi \right) &= 2\left( \partial \cdot \xi \right)
\end{align}
aver $\delta \phi$ la variation infinit\'esmale de $\phi$ apr\`es avoir pris la somme. On peut continuer de calculer $\xi$ et on trouve pour $d > 2$
\begin{align}
    \xi_\nu(x) &= a_\nu + k x_{\nu} + \theta_{\nu\sigma}x^\sigma + \left[ (b \cdot x)x_\nu - \frac{1}{2}(x \cdot x)b_\nu \right]
\end{align}

Les quatres termes correspond \`a la translation, la dialation, la rotation, et les transformations ``speciales'' respectivement. Les param\`etres ont en somme $d + 1 + d + \binom{d}{2} = \binom{d+2}{2}$. Si $d = 2$ on trouve la dimension est infinie avec $z \to z + \epsilon v(z)$ o\`u $v(z)$ est holomorph, alors il y a un infini de choix pour $v(z)$.

\subsection{Diff\'eomorphisme $\mathbb{S}_1$ et son extension centrale}

En $2D$, $f$ est une transformation qui agit sur $\phi(z) \to \phi(f^{-1}(z))$. Si $f$ est proche de l'identit\'e $f(z) = z + \epsilon v(z)$ et on peut \'ecrire
\begin{align}
    (f \cdot \phi)(z) &= \phi(z) - v(z) \partial_z \phi(z)\\
    &= \delta(v)\phi(z)
\end{align}

En particulier, si $v(z) = z^{n+1}$ alors $\delta(v) = l_n \equiv -z^{n+1}\partial_z$ sont les g\'en\'erateurs des transformations avec la relation de commutation $\left[ l_n, l_m \right] = (n-m)l_{n+m}$. Ces relations de commutation sont appel\'ees \emph{l'alg\`ebre de Witt}.

L'alg\`ebre de Lie de $\mathbb{S}_1$ est les diff\'eomorphismes infinit\'esmales $x \to x + \epsilon v(x)$ o\`u $v(x)$ est p\'eriodique sur $x \in [0, \pi]$. On trouve la base $l_n = -e^{-inx}\partial_x$ qui ob\'eit l'alg\`ebre de Witt.

L'alg\`ebre de Virasoro ajoute \`a \'alg\`ebre de Witt comme $\oplus c\mathbb{R}$ avec $c$ un \'el\'ement centrale qui commutent avec toutes autres \'el\'ementes. Les nouvelles relations de commutations sont $[\delta (v_1), \delta(v_2)] = \delta([v_1, v_2]) + c\eta(v_1, v_2)$ avec $\eta$ satisfaisant la cocycle.

\section{L'espace de Fock, les champs quantiques}

\subsection{L'espace de Fock}

On commence avec l'oscillateur harmonique, o\`u $\left[ q,p \right] = 1$ et $h = \frac{1}{2}\left( p^2 + \omega^2 q^2 \right)$. Il y a deux op\'erateurs d'\'echelle $a, a^\dagger: \left[ a, a^\dagger \right] = 1$. On peut \'ecrire $h = \omega\left(a a^\dagger + \frac{1}{2}\right)$, avec $\epsilon_n = \omega\left( n + \frac{1}{2} \right)$. On trouve une base comme $(a^\dagger)^n\ket{0}$.

Soit $V$ l'espace vectoriel de dimension $D$ avec les op\'erateurs d'\'echelles $a(u),a^\dagger(v), u,v \in V$. Ces op\'erateurs agissent sur un espace de Fock $F_v$. Notons que
\begin{align}
    F_v &= \bigoplus F_v^{(n)}\\
    F_v^{(n)} &= \mathrm{Span}\left\{ a^\dagger(u_1)\dots a^\dagger(u_n) \ket{0}, u_i \in V \right\}
\end{align}

Ces $F_v^{(n)}$ sont les sous-espace avec $n$ paricules. Car $\left[ a^\dagger(u), a^\dagger(v) \right] = 0$ les $u_i$ sontt sym\'etriques, alors $F_v^{(n)}$ est isomorphe \`a $\mathrm{Sym} V^{\otimes n}$. Aussi, on peut exprimer $a_i^\dagger = z_i, a_i = \pd{}{z_i}$ et on trouve une repr\'esentation sur les polynomes.

Les op\'erateurs $N_i = a_i^\dagger a_i$ compte le nombre des bosons dans un \'etat $[N_i, N_j] = 0$. Le fonction partition est
\begin{align}
    Z_v \equiv \mathrm{Tr}_{F_v} \left( q^{\sum\limits_{i}^{}u_i N_i} \right) = \prod_i^b \frac{1}{1 - q^{u_i}}
\end{align}

Plus g\'en\'eralement, consid\'erons les op\'erateurs $\left\{ a_ia_j, a_i^\dagger a_j, a_i^\dagger a_j^\dagger \right\}$. Ils preservent la parit\'e de $N$ et forment une alg\`ebre de Lie $\mathrm{sp}(2D)$. $F_v$ se d\'ecompose en $N$ paires et impaires.

\subsection{Les Fermions}

Les fermions sont exactement pareil avec $[a, a^\dagger] \to \left\{ b, b^\dagger \right\}$, et on trouve la principe de Pauli comme \c{c}a aussi.

La seule difference c'est que $\gamma_i = \frac{1}{\sqrt{2}}\left( b_j + b_j^\dagger \right), \gamma_{n+j} = \frac{1}{i\sqrt{2}}\left( b_j - b_j^\dagger \right)$ formenht l'alg\`ebre de Clifford tel que les relations de commutations
\begin{align}
    \left\{ \gamma_j, \gamma_i \right\} &= \delta_{i,j} &
    \left\{ \gamma_j, \gamma_{n+j} \right\} &= 0
\end{align}

\subsection{Les champs Quantiques}

Commen\c{c}ons avec $\box^2 \phi(x,t) = 0$. Imposons les conditions aux limites $\phi(0) = \phi(x = L) = 0$. C'est clair alors qu'il faut d\'ecomposer en s\'erie de Fourier, $\phi(x,t) = \sqrt{2}Q_n(t) \sin(nx)$. Donc l'\'equation d'onde devient
\begin{align}
    \partial_t^2 Q_n + n^2Q_n &= 0
\end{align}

C'est un oscillateur harmonique de pulsation $n$. Par quantification, $Q_n \equiv \frac{1}{\sqrt{2n}}\left( a_n + a_n^\dagger \right)$ et
\begin{align}
    \phi(x,t) &= \frac{1}{\sqrt{n}}\left( a_n e^{int} + a_n^\dagger e^{-int} \right)\sin(nx)
\end{align}

Avec $\ket{0}$, aucun mode n'est excit\'ee; on agit avec $a_n^\dagger$ pour cr\'eer des particules. Ce $\phi(x,t)$ agit sur un espace de Fock engendr\'e par $\left\{ a_n, a_n^\dagger \right\}$. 

\chapter{16/09/14 --- Introduction, la commence de ma mort}

La page web est \url{http://www.math.polytechnique.fr/~renard/GrSym.html}. Les emails des profs sont \url{renard@math.polytechnique.fr} et \url{denis.bernard@ens.fr}.

On doit choisir un sujet et envoyer sa choix au prof. On devra faire un projet sur \c{c}a; on peut choisir au plus une partenaire\dots 

La symetrie est tr\`es importante pour la th\'eorie de notre \^age. Par exemple, il y a des th\'eories (r\'elativit\'e g\'enerale) qui sont develop\'ees en commen\c{c}ant par leurs symetries. \emph{La th\`eor\`eme de Noether} guarantie qu'il y a des quantit\'es conserv\'e s'il y a une symetrie continuelle.

Quand on parle des structures math\'ematiques comme un espace, une syst\'eme, un ensemble on utilise $X$ et quand on parle des groupes, bijections ou quelques choses comme \c{c}a on utilise $G$. On appelle $e$ l'\'el\'ement neutre. La notion de base, c'est une action de groupe $a$ qui est comme $(G \times X) \overset{a}{\to} X$, ou $a(g,x) = g \cdot x$. Les actions de groupe sont \'equivilant \`a donner un morphisme de groups $G \overset{A}{\to} \mathrm{Bij}(X)$.

Soit $X=\left\{ 1\dots n \right\}$ un ensemble et $G = \mathrm{Bij}(X)= \sigma_n$ le groupe symetrique; c'est la structures des bosons. Aussi, soit $X = V$ un espace vectoriel sur $\mathbb{R}$ ou $\mathbb{C}$. Soit aussi $G = \mathrm{GL}(V)$ les applications lin\'eaires bijectives. 

Si on a le morphisme de groups $G \overset{A}{\to} \mathrm{GL}(V)$ on appelle cette action un \emph{repr\'esentation}. Et donc si on a des repr\'esentations $F,G$ sur $X$, on sait que $\left(g \circ f\right)(x) = f\left(g^{-1}\circ x\right)$ (si on regarde l'axiome $f\circ\left( g\circ x \right) = \left( fg \right)\circ x$ on voit le $-1$)

Des autre examples des groups
\begin{itemize}
    \item L'espace de Hilbert $\mathcal{H}$ du produit $\left<\;,\;\right>$. On se demande quel est la structure de ce groupe? $\mathcal{U}(\mathcal{H})$ est groupe unitaire, ou $g \in \mathrm{GL}(\mathcal{H}), \expvalue{g \cdot v, g \cdot w} = \expvalue{v,w}$.
    \item Le produit scalaire sur $V$ est aussi un groupe. Et donc $\mathcal{O}\left(V,\expvalue{\;,\;}\right)$ est un groupe orthogonal, ou $g \in \mathrm{GL}(V), \expvalue{g\cdot v, g \cdot w} = \expvalue{v,w}$.
    \item $V = \mathrm{R}^n$ avec un produit $\expvalue{\;,\;}$ qui est une forme bilin\'eaire sym\'etrique de signature $p,q, p+q=n$ d\'efinir comme $\expvalue{\mathbf{x},\mathbf{y}} = x_1y_1 + \dots + x_py_p - x_{p+1}y_{p+1} - \dots - x_ny_n$. Le groupe $O(1,3)$ est le groupe de Lorentz, en relativit\'e.
\end{itemize}

On parle aussi de groupe $SL(V) = \left\{ g \in GL(V), \det g = 1 \right\}$, le groupe \emph{sp\'ecial lin\'eaire}. Enfin nous pouvons \'ecrire $SO(p,q) = O(p,q) \cap SL(V)$ et $SU(p,q) = U(p,q) \cap SL(V)$.

Donc si on a une groupe $(V,w)$ avec $w$ une forme bilin\'eaire antisymmetrique ($w(x,y) = -w(y,x)$) et non-degen\'er\'ee sur $V$, alors $\dim V$ est paire. On appelle \c{c}a une forme \emph{symplectique}. Et donc $Sp(V,w)$ est une groupe symplectique.

Un exemple de groupe symplectique c'est le groupe de Heisenberg. $(V,w)$ et symplectique. On \'ecrit ce groupe $H = H_{(V,w)} = V \oplus \mathbb{R}$. Ensuite on d\'efine $(v,x) \cdot (v',x') = (v+v', x+x' + w(v,v'))$. Si on choisit $V = \mathbb{R}^2,w$ on obtient $H = \mathbb{R}^2 \oplus \mathbb{R} = \mathbb{R}^3$ et donc
\begin{align}
    H &= \begin{bmatrix} 1 & x & y & z \\
        0 & 1 & 0 & y \\
        0 & 0 & 1 & -x \\
        0 & 0 & 0 & 1 \end{bmatrix} 
\end{align}

C'est bien sur anti-symetrique, et si on compose deux matrices comme \c{c}a on obtient encore une matrice comme \c{c}a, et donc on a la meme loi de groupe que plus t\^ot. Tous ces groupes s'appellent le groupe de Heisenberg.

On a aussi $X$ un variet\'e diff\`erentiable, et donc $G = \mathrm{Diff}(X)$ groupe de diff\`eromorphisme d'$X$. $G$ est de dimension infinie.

Le but de cette classe est d'\'examiner la th\'eorie des repr\'esentations, au plupart des groups finis. Donc qu'est-ce que c'est une repr\'esentations? \emph{Une repr\'esentation de $G$ est la donn\'ee d'un espace vectoriel complexe et d'un morphism de groups $G \overset{\rho}{\to} \mathrm{GL}(V)$}. Si $V$ est un espace hermitien et si $\rho: G \to U(V, \expvalue{\;,\;})$ on dit que la repr\'esentation $(\rho,V)$ est unitaire. Si $V, \rho: g \in G \mapsto \mathrm{Id}_V$ on dit que $(\rho,V)$ est une repr\'esentation triviale. Si $V \in \mathbb{C}$ on parle de \emph{la} repr\'esentation triviale. On parle de la dimension de la repr\'esentation $\dim(\rho,V) = \dim V = \dim \rho$. 

Maintenant on suppose que $G$ est fini et que $(\rho,V)$ est une repr\'esentation de $G$. On propose qu'il existe un produit hermitien qui rend $(\rho,V)$ unitaire. L'id\'ee c'est de prendre un nouveaux produit $\expvalue{\;,\;}_0$ quelconque $\expvalue{v,w}_0 = \frac{1}{\abs{G}}\sum\limits_{g\in G}^{}\expvalue{\rho(g)\cdot v, \rho(g) \cdot w}$. 

Une sous-repr\'esentation d'une repr\'esentation $(\rho,V)$ est un sous-espace $W$ de $V$ stable par l'action de $G$. Une repr\'esentation est irr\'eductible si ces seules sous-repr\'esentation sont $\left\{ 0 \right\}$ et $V$. Un lemme dit que si $G$ est fini, $(\rho,V)$ est irreductible alors elle est de dimension finie.

On a maintenant deux th\`eor\`eme importantes de la th\'eorie de repr\'esentation des groups finis. On dit que $(\rho,V)$ est compl\`etement reductible (ou semisimple) si $V$ s'\'ecrit comme somme directe $V = \oplus W_i$ ou $W_i$ sont des repr\'esentations irreductible. La th\`eor\`eme est que toute representation de dimension finie d'un groupe fini est compl\`etement reductible.

On a des op\'eratures  d'entrelacement (en anglais ``intertwining operators''). Soit $G, (\rho_1, V_1), (\rho_2,V_2)$ et $T: V_1 \to V_2$ application lin\`eaire tell que $T\left( \rho_1(g) \cdot v_1 \right) = \rho_2(g) \cdot \left( T \cdot v_1 \right)$. On dit que les deux repr\'esentations sont equivalentes si $T:V_1 \to V_2$, l'op\'erateur d'entrelancement, est inversible.

On a une lemme. Soit $T: V_1 \to V_2$ un op\'erateur d'entrelacemcement, alors $\ker T, \mathrm{Im} T$ sont des sous-representations de $V_1, V_2$ respectivement. Soit aussi $V_\lambda \subset V$ sous-espace propre de $T$ avec une valeur propre $\lambda$, alors $V_\lambda$ est une sous-representation. 

On a maintenant la lemme de Schor. Si $(\rho_1, V_1), (\rho_2, V_2)$ sont irreducibles, alors si $\rho_1 \sim \rho_2$, $\dim \mathrm{Hom}_G(V_1, V_2) = 1$ et $\rho_1 \times \rho_2$ alors $\dim \mathrm{Hom}_G = 0$ ($\mathrm{Hom}$ c'est homoromphism entre $V_1, V_2$).

La contragredi\`ente: $G, (\rho,V)$, si on appelle $V^*$ les formes lineaire sur $V$, $\left(\rho^*(g) \cdot \lambda\right)(v) = \lambda\left( \rho^{-1}(g) \cdot v \right)$. Si on a $(\rho_1, V_1), (\rho_2, V_2)$, on peut mettre $V_1 \oplus V_2$ sous l'action de $G$, on le fait comme $\left( \rho_1 \oplus \rho_2 \right)(g)(v_1, v_2) = \left( \rho_1(g)\cdot v_1, \rho_2(g) \cdot v_2 \right)$.

On rapelle la difference entre $V_1 \oplus V_2, V_1 \otimes V_2$ parce que la dimension de le premier c'est la somme des deux dimensions et de le dernier c'est le produit. Autrement dit, si $v_i, w_j$ sont des bases de $V_1, V_2$ alors $v_i \otimes w_j$ est une base de $V_1 \otimes V_2$.

On a encore un th\`eor\`eme, c'est que si $G$ est un group fini et $(\rho,V)$ sont de dimension finie, $(\rho,V)$ est compl\`etement r\'eductible.

Il y en reste plus mais je suis compl\`etement perdu\dots Je suis fini pour le moment.

Il y a quelques categories: groupes finis $\subset$ groupes compacts $\subset$ groupes topologiques, et tous ces groupes contienent les groupes lineaires.

\chapter{07/10/14 --- Repr\'esentations des groupes en MQ}

On commence avec les action sur les \'etats, par exemple rotation et translation. En g\'en\'eral, il y a des repr\'esentation pour les groupes des actions, ou les repr\'esentation projective comme $\mathbf{SO}(2)$. Les symmetries sont li\'ees des conservations et des r\`eges de selection; on peut discuter la structure de l'espace de Hilbert. On parle maintenant des exemples des groupes on physiques.

Au d'abord, il faut parler du \emph{th\'eor\`eme de Wigner}: Soit $S$ une transformation bijective sur l'espace des \'etats qui preserve le produit scalaire, alors $S$ est lin\'eaire ou anti-lin\'eaire et unitaire. Donc, si $G$ est un groupe agissant sur les \'etats, on l'appelle unitaire si $g \in G, g:\ket{\psi} \to \ket{g\psi} = U(g) \ket{\psi}$ avec $U(g)$ unitaire.

Il y a donc deux possibilit\'es pour l'action d'un produit de deux \'el\'ements $g_{1,2} \in G$: soit $g_1(g_2)\psi = (g_1g_2) \psi$, et donc $U(g_1)U(g_2) = U(g_1g_2)$ avec $U$ une repr\'esentation unitaire; soit $g_1(g_2)\psi = e^{i\omega(g_1,g_2)}(g_1g_2)\psi$ une repr\'esentation projective $U(g_1)U(g_2) = e^{i\omega(g_1, g_2)}U(g_1g_2)$.

Pour un repr\'esentation on trouve le r\'esultat suivant: La repr\'esentation $U$ de $G$ sur $H$ c'est une repr\'esentation de Lie $G$ sur $H$ not\'ee $U$: $[U(X),U(Y)] = U([X,Y])$. 

Pour un repr\'esentation projectives on a quelque propriet\'es/contraintes (notons $e^{i\omega(g_1, g_2)} = \Omega(g_1,g_2)$)
\begin{itemize}
    \item Associativit\'e sur produit $(g_1g_2)g_3 = g_1(g_2g_3)$ --- On trouve que $\Omega(g_1,g_2)\Omega(g_1g_2,g_3) = \Omega(g_1,g_2g_3)\Omega(g_2,g_3)$, la \emph{relation de cocycle}.
    \item Certaines solutions de la relation de cocycle sont triviales --- Si on arrive \`a \'ecrire $\omega(g_1,g_2) = \eta(g_1g_2) - \eta(g_1) - \eta(g_2)$, en notant que $\eta:G \to \mathbb{R}, \omega: G \times G \to \mathbb{R}$, alors on trouve la solution triviale par trouver $\hat{U}(g) = e^{i\eta(g)}U(g)$ et $\hat{U}$ est une repr\'esentation.
    \item Une repr\'esentation projective c'est simplement un repr\'esentation d'une [plus grosse] groupe. Par exemple, on d\'efinit $\hat{G} = G \otimes \mathbf{U}(1)$ avec $(g,\lambda) \cdot (h,\mu) = (gh, \lambda \mu \Omega(g,h))$.
\end{itemize}

\section{$\mathbf{SO}(2)$ et $\mathbf{U}(1)$}

On pose un exemple $\mathbf{SO}(2)$ les rotations de 2D. On note que $\mathbf{SO}(2) \simeq \mathbf{U}(1)$ parce que $\mathbf{U}(1) = e^{i\theta}$ et les rotations sont aussi d\'ecrites par un angle. On prend maintenant une classes des repr\'esentations (ou non?) $D: \mathbf{U}(1) \to \mathbb{C}, D_j(e^{i2\pi\theta}) = e^{ij2\pi x}$; c'est une repr\'esentation pour les spins. C'est clairment une repr\'esentation seulement quand $j \in \mathbb{Z}$, ou $D_j$ c'est pas une application; si deux angles des deux elements dans $\mathbf{U}(1)$ diff\`erent par $2\pi i$ ils devraient \^etre envoyer \`a la m\^eme nombre par $D$.

Mais donc, qu'est-ce qui se passe quand $j$ est de $\frac{1}{2}$-entier, par exemple $j = \frac{1}{2}$? On aurait $D_j(e^{i2\pi x})D_j(e^{i2\pi y}) = D_j\left( e^{i2\pi x}e^{i2\pi y} \right)$. Ca marchera si on red\'efinit $\hat{D}(e^{2\pi i x}) = D(e^{i2\pi \overline{x}}), \overline{x} \in \left[ 0,1 \right]$, et comme \c{c}a on trouve $D_j(e^{i2\pi \overline{x}})D_j(e^{i2\pi \overline{y}}) = D_j(e^{i2\pi \overline{x + y}})\Omega_j(\overline{x}, \overline{y}), \Omega_j = \lfloor x + y\rfloor - \lfloor x \rfloor - \lfloor y\rfloor$. Comme \c{c}a on trouve un rel\^evement qui change la repr\'esentation projective \`a une repr\'esentation.

\section{$\mathbf{SO}(3) \to \mathbf{SU}(2)$}

On parle maintenant de $\mathbf{SO}(3) \to \mathbf{SU}(2)$. On note que la repr\'esentation de spin $j = \frac{1}{2}$ n'est pas une repr\'esentation de $\mathbf{SO}(3)$ mais une repr\'esentation de $\mathbf{SU}(2)$, et donc une repr\'esentation projective de $\mathbf{SO}(3)$. Rappelons les matrices de Pauli
\begin{align}
    \sigma_z &= \begin{pmatrix} 1 & 0 \\ 0 & -1 \end{pmatrix} & \sigma_x &= \begin{pmatrix} 0 & 1 \\ 1 & 0 \end{pmatrix}  & \sigma_y &= \begin{pmatrix} 0 & -i \\ i & 0 \end{pmatrix} 
\end{align}

Une rotation d'angle $\alpha$ de direction $\vec{n}$ est represent\'ee par l'exponentielle $e^{i\alpha \frac{\vec{n} \cdot \vec{\sigma}}{2}}$. Cette application $(\vec{n}, \alpha) \to \exp\left( i\alpha\frac{n \cdot \vec{\sigma}}{2} \right)$ est encore pas une application dans $\mathbf{SO}(3)$, \`a cause de $\alpha \to \alpha + 2\pi$ mais c'est qu'une repr\'esentation projective de $\mathbf{SO}(2)$. Et donc $\mathbf{SO}(2)$ est une rel\^evement/recouvrement universelle de $\mathbf{SO}(3)$ mais pas de l'autre direction.

\section{Dynamiques \`a cause des symmetries, lois de conservation}

Les dynamiques d'une syst\`emes suivent l'hamiltonien $H$ par l'\'equation de Schrodinger $\ket{\psi(t)} = e^{iHt}\ket{\psi(t_0)}$. Supposons qu'on a unegroupe $G$ qui est une repr\'esentation sur $H$ comme $U(g) \in \mathbf{End}(H)$. On appelle $G$ une sym\'etrie si son action sur $H$ commute avec le flot engendr\'e par $H$, ou $\left[ U(g), H \right] = 0$. Egalement on peut demander que $[H,U(X)] = 0, X \in \mathbf{Lie}(G)$.

Par exemple, si $\expvalue{U(g)}(t) = \bra{\psi_0}e^{iHt}U(g)e^{-iHt}\ket{\psi_0} = \bra{\psi_0}U(g)\ket{\psi_0}$ est ind\'ependant du temps on a une loi de conservation pour $G$. 

On veut maintenant d\'ecomposer $H$ et l'action de $G$. Supposons que $H$ est totalement r\'eductible en somme des repr\'esentations irr\'eductibles, donc $H = R_1 \oplus R_{1'} \oplus R_2 +\dots$ avec peut-\^etre des repr\'esentations \'equivalentes, comme $R_1, R_{1'}$. Ca veut dire que on peut \'ecrire l'hamiltonien en bloques diagonale avec les $R_i$ commes bases. On va appeler les bloques $D_i(g)$ pour chaque bases. Chaque bloques est une repr\'esentation de $G$, et avec des repr\'esentations \'equivalentes on a une base qui lie les deux repr\'esentations comme $D_{1'}(g) = U^\dagger D_1(g)U$ avec $U$ unitaire.

On peut aussi \'ecrire $H = \bigoplus_\lambda \bigoplus_{j=1}^{m_\lambda} R_\lambda^{(j)}$ avec $m_\lambda$ la multiplicit\'e de $\lambda$ dans la d\'ecomposition. Ca produit la decomposition canonique aussi $H = \bigoplus_\lambda W_\lambda \otimes R_\lambda$ avec $\dim W_\lambda = m_\lambda$. Donc pour $H_\lambda = W_\lambda \otimes R_\lambda$ on voit que $U(g)$ ne s'agit pas sur $W_\lambda$ seulement sur $D_\lambda(g)$ et donc $U(g) = \mathbf{I} \otimes D_\lambda(g)$. 

On se rappele la Lemme de Schur: soit deux repr\'esentations irr\'eductibles $D_1, D_2$ d'une groupe $G$ tel qu'il existe un entrelaceur $S$ entre $D_1,D_2$ qui s'agit comme $SD_1(g) = D_2(g)S$, alors $S=0$ ou $S$ est invertibles et $D_1 \simeq D_2$. Donc, $[H, U(g)] = 0$ dit que $H$ est bloque diagonal dans la base de $\lambda$. 

On propose donc que 
\begin{align}
    H &= \bigoplus h_\lambda \otimes \mathbf{I}\label{07.10.H}\\
    U(g) = \mathbf{I} \otimes D_\lambda(g)
\end{align}
J'ai pas compris la d\'emonstration.

On parle maintenant de l'invariance par rotation, une particule de masse $m$ dans un potential central. On sait que $H = \bigoplus_{l=0}^\infty L^2\left( \mathbb{R}_+ \right)\otimes R_l$; la premi\`ere terme c'est $f(r)$ le potentiel et la dernni\`ere terme c'est la repr\'esentation de $\mathbf{SO}(3)$. Et donc l'hamiltonien qui s'agit sur $\mathbb{R}_+$ est donn\'e par
\begin{align}
    H &= -\frac{\hbar^2}{2m}\left( \rd{}{r} \right)^2 + \frac{l(l+1)\hbar^2}{2mr^2} + V(r)
\end{align}

On propose donc quelques consequences
\begin{itemize}
    \item R\`egle de selection --- Les dynamiques laisse stable les scalaires $H_\lambda$.
    \item La dynamique de $H_\lambda$ ne transition pas entre les valeurs differenttes.
    \item Les nombres $\lambda$ sont conserv\'ees, et donc la loi de conservation
    \item Si $H$ est invariant par une groupe, on voit dans la d\'ecomposition dans l'\'equation \eqref{07.10.H} qu'il y a plusieurs espace avec $h_\lambda$. Donc les degeneracies correspond aux symmetries.
\end{itemize}

\section{Groupes de Lorenttz et de Poincar\'e}

La groupe de Poincar\'e est la groupe d'invariance de l'espace sur $\mathbb{R}^4$ munie de la metrique Minkowskienne. La groupe de Lorentz s'agit sur $\mathbb{R}^4, \eta\left[ X,Y \right] = X_0Y_0 - X_1Y_1 - X_2Y_2 - X_3Y_3$, il est la groupe $\mathbf{O}(1,3)$ qui satisfait $\eta(\Lambda X, \Lambda Y) = \eta(X,Y)$. On \'ecrit aussi que
\begin{align}
    \eta(X,Y) &= X^\mu\eta_{\mu \nu}Y^\nu
\end{align}

On commence de parler des rotations, qui prend la forme $\begin{pmatrix} 1 & 0 & 0 & 0\\0 & \cos \alpha & \sin\alpha & 0 \\ 0 & -\sin\alpha & \cos\alpha & 0 \\ 0 & 0 & 0 & 1 \end{pmatrix} $. Note que $\mathbf{O}(3) \subset \mathbf{O}(1,3)$. On parle aussi des boost, par exemple de l'axe $z$: $\begin{pmatrix} \cosh\alpha & \sinh\alpha & 0 & 0 \\ \sinh \alpha & \cosh\alpha & 0 & 0 \\ 0 & 0 & 1 & 0\\0 & 0 & 0 & 1 \end{pmatrix}$. 

On propose que $\det \Lambda = \pm 1$, et de \c{c}a on trouve que $\abs{\Lambda_0^0} \geq 1$. On obtient donc quatre cas, $\det \Lambda = \pm 1, \Lambda_0^0 \gtrless 0$.

L'algebre de Lie de cette groupe on \'etudie les petites transformations $\Lambda_\nu^\mu = \delta_\nu^\mu + \epsilon \omega_\nu^\mu +\dots$ tel qu'on satisfait
\begin{align}
    \eta_{\mu \nu} &= \eta_{\rho\sigma}\Lambda_\mu^\sigma\Lambda_\nu^\rho\\
    &= \eta_{\mu \nu} + \epsilon\left( \eta_{\rho\mu}\omega_{\nu}^\rho + \eta_{\nu\sigma}\omega_\mu^\sigma \right)\\
    0 &= \eta_{\rho\mu}\omega_{\nu}^\rho + \eta_{\nu\sigma}\omega_\mu^\sigma
\end{align}
l'\'equation d'invariance. Si o pose donc $\omega_{\mu\nu} = \eta_{\rho\mu}\omega_{\nu}^\rho$ on trouve $\omega_{\mu\nu} + \omega_{\nu\mu} = 0$, antisymmetrique. Donc la dimension d'algebre de Lie est $6$, la dimension des matrices $4 \times 4$ antisymmetriques. On peut trouver une base pour $\mathfrak{so}(1,3)$ simplement par les matricies $(1,-1)$ pour toutes les entr\'ees antisymmetriques. Une autre base est compos\'ee pa les generateur de rotation $\mathfrak{so}(3)$ autour les $3$ axes et les boosts selon les $3$ axes; on appele les rotation $J_i$ et les boost $K_i$. Les matricies sont comme par exemples
\begin{align}
    J_z &= i\begin{pmatrix} 0 & 0 & 0 & 0\\0 & 0 & 1 & 0\\0 & -1 & 0 & 0\\0 & 0 & 0 & 0 \end{pmatrix} &K_x &= i\begin{pmatrix} 0 & 1 & 0 & 0\\1 & 0 & 0 & 0\\0 & 0 & 0 & 0\\0 & 0 & 0 & 0 \end{pmatrix} 
\end{align}

On trouve la structure de Lie, c'est $\left[ J_x, J_y \right] = iJ_z, \left[ J_x K_y \right] = iK_z, \left[ K_x, K_y = -iJ_z \right]$. Comme \c{c}a on a trouv\'e une repr\'esentation de l'algebre de Lie de $\mathfrak{so}(1,3)$.

\chapter{14/10/14 --- Plus de Groupes de Lorentz et Poincar\`e}

\section{$\mathfrak{so}(1,3)$}

La groupe de Lorentz est d\'efini $\mathbf{O}(1,3) \subset \mathbf{SO}(1,3)$. On a $\Lambda \in \mathbb{R}^4$ une op\'erateur quelconque tel que $\eta(X,Y) = \eta(\Lambda x, \Lambda y)$ pour la m\'etrique $\eta$. On munit l'espace vectoriel d'un produit Minkowskian $X\cdot Y = \Lambda X \cdot \Lambda Y$ qui est invariant par $\Lambda$ avec la signiature $(+,-,-,-)$.

On commence en calculer l'alg\`ebre de Lie $\mathfrak{so}(1,3)$. On fait une expansion $\Lambda = I + \epsilon\omega +\dots$ et donc $\Lambda^{\mu}_\nu = \delta^\mu_\nu + \epsilon \omega^\mu_\nu +\dots$ avec $\omega^\mu_\nu$ une matrice $4\times4$. On voit (c'est une consequence de demander l'invariance de Lorentz) que $\omega_{\mu\nu} + \omega_{\nu\mu} = 0$ et donc $\omega$ est antisymmetric, apr\`es lequel on trouve que $\dim_{\mathbb{R}} \mathfrak{so}(1,3) = 6$. 

On choisit une base de tenseur antisymmetrique alors, comme $\left( J^{\rho\sigma} \right)_{\mu\nu} = \delta^\rho_\mu\delta^\sigma_\nu - \delta^\sigma_\mu\delta^\rho_\nu$ avec $\rho,\sigma$ la labelle. Donc, apr\`es la m\'etrique on trouve que
\begin{align}
    \left(J^{\rho\sigma}\right)^\mu_\nu &= \eta^{\rho\mu}\delta^\sigma_\nu - \eta^{\sigma\mu}\delta^{\rho}_\nu
\end{align}
qui est \'evidement antisymmetrique. Alors on a trouv\'e une base de $\mathfrak{so}(1,3)$, et on peut \'ecrire $\omega^\mu_\nu$ en somme de $J^{\rho\sigma}$ comme $\omega = \frac{1}{2}\omega_{\rho\sigma}J^{\rho\sigma}$ (par exemple $\omega_{\mu\nu} = \frac{1}{2}\omega_{\rho\sigma}\left(J^{\rho\sigma}\right)_{\mu\nu} = \omega_{\mu\nu}$).

On examine maintenant la relation de commutation, la crochet de Lie de $\mathfrak{so}(1,3)$ comme 
\begin{align}
    \pm\left[ J^{\rho\sigma}, J^{\theta\phi} \right] = \eta^{\rho\theta}J^{\sigma\xi} - \eta^{\sigma\theta}J^{\rho\xi} - \eta^{\rho\xi}J^{\sigma\theta} + \eta^{\sigma\xi}J^{\rho\theta}
\end{align}
ou le prof a oubli\'e la signe exacte. En physique, on met des $i$ par l'expansion $\Lambda = I - i\epsilon\omega +\dots$, et \c{c}a change l\'eg\`erement ce qui proc\`ede ci-dessus mais pas beaucoup.

Alors on cherche maintenant des d\'ecompositions des op\'erateurs de rotation. $\vec{J}$ est d\'efinit par deux indices (la direction de la rotation dans l'espace) et donc $J^j = \frac{1}{2}\epsilon^{jkl}J^{kl}$. Par contre, $\vec{K}$ c'est la rotation en temps suivant un axe et donc $K^j = J^{0j}$. Les relations de commutations pour eux sont
\begin{align}
    \left[ J^j, J^k \right] &= i\epsilon^{jkl}J^l\\
    \left[ J^j, K^k \right] &= i\epsilon^{jkl}K^l\\
    \left[ K_j, K_l \right] &= i\epsilon^{jkl}J_l
\end{align}

\section{Action sur les fonctions}

On examine donc l'action sur les fonctionnes (les champs scalaires). On d\'efinit l'actioon comme une repr\'esentation $\Lambda: \phi \to \Lambda \phi$. Rappelons que $\Lambda \phi(X) = \phi(\Lambda^{-1}X)$. Quelle est alors la repr\'esentaiton d'alg\`ebre de Lie associ\'ee \`a cette action? On \'ecrit $\Lambda = I + \epsilon \omega, \Lambda^{-1} = I - \epsilon \omega$. On prend maintenant
\begin{align}
    \phi(\Lambda^{-1}X) &= \phi\left[ X^\mu - \epsilon(\omega X)^\mu \right]\\
    &= \phi(X) - \epsilon\left( \omega X \right)^\mu\pd{\phi}{x_\mu}+\dots\\
    &= \phi(X) - i\left( J^{\rho\sigma}X \right)^\mu\partial_\mu \phi(x)\\
    &= \phi(X) - i\left( J^{\rho\sigma} \right)_{\mu\nu}X^\mu\partial^\mu \phi(x)\\
    &= \phi(X) + i\left( X^\rho\partial^\sigma - X^\sigma\partial^\rho \right)\phi(x)
\end{align}

Donc sur l'espace des fonctions $J^{\rho\sigma}$ est represent\'ee par $J^{\rho\sigma} = \pm i\left( X^\rho\partial^\sigma - X^\sigma\partial^\rho \right)$, ou on a encore oubli\'e la signe.

\section{Repr\'esentations de dimension finie}

On veut construire sp\'ecifiquement $\mathfrak{so}(1,3)$. Notons que $\mathfrak{so}_{\mathbb{C}}(1,3) \simeq \mathfrak{sl}(2,\mathbb{C}) \oplus \mathfrak{sl}(2,\mathbb{C})$ ou aussi $\mathfrak{so}_\mathbb{C}(1,3) \simeq \mathfrak{so}(1,3) \otimes \mathbb{C}$, les combinaisons lin\'eaires \`a coefficients complexes.

On d\'efinit donc une base de $\mathfrak{so}(1,3)_{\mathbb{C}}$ comme $N^j = \frac{J^j + iK^j}{2}, M_j = \frac{J_j - iK_j}{2}$. L'avantage de faire \c{c}a c'est \`a cause des relations de commutations
\begin{align}
    \left[ N_j, N_k \right] &= i\epsilon^{jkl}N^l & \left[ M_j, M_k \right] &= i\epsilon^{jkl}M^l & \left[ N_j, M_k \right] &= 0
\end{align}

On voit dans les relations deux copies de $\mathfrak{sl}(2,\mathbb{C})$ dans les deux premi\`eres relations, ce qui est en d'accord avec ce qu'on proposait ci-dessus. 

Pour construire une repr\'esentation de $\mathfrak{sl}(2,\mathbb{C}) \oplus \mathfrak{sl}(2,\mathbb{C})$:
\begin{enumerate}[(i)]
    \item On prend \emph{deux} repr\'esentations de $\mathfrak{sl}(2,\mathbb{C})$. On les appelle $V,W$ avec des dimensions $n+1, m+1$ respectivement.

    \item On consid\`ere le produit tensoriel $V \otimes W$ avec dimension $(n+1)(m+1)$.

    \item On d\'efini la repr\'esentation en faisant agir $\vec{N}$ sur $V$ et $\vec{M}$ sur $W$. Ce produit agit comme
        \begin{align}
            \rho_{V \otimes W}\left( \vec{N} \in \mathfrak{so}_{\mathbb{C}}(1,3) \right) = \rho_V(\mathfrak{sl}(2,\mathbb{C})) \otimes I_W
        \end{align}
        et pareil pour $\vec{M}$. 

    \item Pour reconstruire l'action de $\vec{J}, \vec{K}$ on fait simplement les calculs et on obtient
        \begin{align}
            \rho(J) &= \rho_V(N)\otimes I + I \otimes \rho_W(M) &
            \rho(K) &= \frac{1}{i}\left(\rho_V(N)\otimes I - I \otimes \rho_W(M)\right)\label{14.10.truc}
        \end{align}
\end{enumerate}

\section{Exemples de repr\'esentations}

Les repr\'esentations de $\mathfrak{sl}(2,\mathbb{C})$ sont index\'ees par le spin $j$ entier ou demi-entier comme $\dim = 2j + 1$. Donc
\begin{itemize}
    \item $(j_N = j_M = 0)$, donc $\dim = 1$, $\rho(J) = \rho(K) = 0$ et on a une repr\'esentation scalaire.
    \item $\left(j_N = \frac{1}{2}, j_M = 0\right)$, donc $\dim = 2$. On note que l'action de $\vec{N}$ est donn\'ee par $\vec{\sigma}$ les matrices de Pauli, et donc apr\`es les \'equations \eqref{14.10.truc} on trouve
        \begin{align}
            \rho\left( \vec{J} \right) &= \frac{\vec{\sigma}}{2} & \rho\left( \vec{K} \right) &= -i\frac{\vec{\sigma}}{2}
        \end{align}

        On appelle \c{c}a la spineur droite $\chi_R$.
    \item $\left(j_N = 0, j_M = \frac{1}{2}\right)$ --- C'est la m\^eme que ce qui proc\`ede, sauf une signe parce que l'action est maintenant dans l'espace de $\vec{M}$:
        \begin{align}
            \rho\left( \vec{J} \right) &= \frac{\vec{\sigma}}{2} & \rho\left( \vec{K} \right) &= +i\frac{\vec{\sigma}}{2}
        \end{align}

        On appelle \c{c}a la spineur gauche $\chi_L$.
    \item $\left( j_N = j_M = \frac{1}{2} \right)$, donc $\dim = 2 \times 2 = 4$, et c'est donc sur $\mathbb{C}^{4}$; il faut utiliser les quadri-vecteurs $A^\mu$. On a
        \begin{align}
            \rho(\vec{J}) &= \frac{1}{2}\left( \vec{\sigma} \otimes I + I\otimes \vec{\sigma} \right) &
            \rho(\vec{K}) &= -\frac{i}{2}\left( \vec{\sigma} \otimes I - I\otimes \vec{\sigma} \right) 
        \end{align}

        On peut montrer que cette repr\'esentation est \'equivalent \`a la repr\'sentation avec les $A^\mu$. 

    \item $\left( j_N = 0, j_M = 1 \right) \oplus \left( j_N = 1, j_M = 0 \right)$, donc $\dim = 6$. On voit la dimension et fait la connection avec les tenseurs $4 \times 4$ antisymmetriques! Ce sont les repr\'esentations \'equivalentes (? ou il a peut-\^etre dit adjointe, je ne l'ai pas \'ecout\'e)
\end{itemize}

On se demande maintenant aussi des matrices symmetriques $G_{\mu\nu}$ qui ont la dimension $\frac{4 \times 5}{2} = 10$. Mais on voit que les matrices avec la m\^eme trace sont invariantes (contracter $G_{\mu\nu}$ comme $G = \eta^{\mu\nu}G_{\mu\nu}$ produit un scalaire invariant par $\eta^{\mu\nu}$) et donc \c{c}a fait un sous-espace stable, et l'espace des matrices symmetriques est de dimension $9$, comme $j_N = j_M = 1$. Donc on voit que $\mathrm{Sym}(\mathbb{M}_{4\times 4} = (0,0) \oplus (1,1)$. 

Aussi on peut \'examiner la parit\'e $\left( x^0, \vec{x} \right) \to (x^0, -\vec{x})$. On sait que cette transformation $P \in \mathbf{SO}(1,3) \in \mathbf{O}(1,3)$ et donc son action sur $\mathfrak{so}(1,3)$ est donn\'ee par $P\vec{J}P = \vec{J}, P \vec{K}P = -\vec{K}$. En effet \c{c}a change $\vec{N} \leftrightarrow \vec{M}$ dans l'\'equation \eqref{14.10.truc}. Plus exactement, $P$ se represente sur la somme directe $\left( j_N = j_i, j_M = j_2 \right) \oplus \left( j_N = j_2, j_M = j_1 \right)$. Pour les spineurs en particulier on trouve qu'il se represente sur l'espace $\left( \frac{1}{2},0 \right) \oplus \left( 0, \frac{1}{2} \right)$, de dimension $4$. On l'appelle le spineur de Dirac $\psi(x) = \begin{pmatrix} \chi_R\\ \chi_L \end{pmatrix} $. 

\section{Equation de Dirac}

Dirac est arriv\'e \`a son \'equation c\'el\`ebre par l'\'equation de Schrodinger par $E^2 = p^2 + m^2$ et par prendre la racine carr\'ee mais on va le faire via la th\'eorie de groupe.

On a une particule avec l'impulsion $\mathbf{P} = \left( P^0, \vec{P} \right)$ et masse $m$ tel que $\mathbf{P}^2 = m^2$. 
\begin{enumerate}[i)]
    \item Dans le centre de masse, $\mathbf{P} = \left( m, \vec{0} \right)$. Aussi par l'action de la parit\'e on \'ecrit $\chi_R^* = \chi_L^*$ dans le referentiel du centre de masse.
    \item Dans le referentiel quelconque $\chi_R, \chi_L$ sont obtenus par la transformation de Lorentz. On appelle la courbe $\mathbf{P}^2 = m^2, P_0 > 0$ la ``couche de masse.'' On se bouge sur cette couche par la transformation de Lorentz. On trouve alors que $P^0 = m\cosh \alpha, \vec{P} = m\hat{n}\sinh\alpha$ apr\`es un boost dans la direction $\hat{n}$ avec une rapidit\'e $\alpha$ du referentiel de centre de masse. Si on appelle ce boost $\Lambda(\hat{n},\alpha)$, on agit sur les spineurs par l'action donn\'ee par la representation comme
        \begin{align}
            \chi_R(\vec{p}) &= \rho_{\left( \frac{1}{2},0 \right)}\left( \Lambda(\hat{n},\alpha) \right) \cdot \chi^{*}&
            \chi_L(\vec{p}) &= \rho_{\left( 0, \frac{1}{2} \right)}\left( \Lambda(\hat{n},\alpha) \right) \cdot \chi^{*}
        \end{align}

    \item On prend simplement l'\'exponentiel de l'alg\`ebre de Lie $\mathfrak{so}(1,3)$ et on obtient
        \begin{align}
            \chi_R(\vec{p}) &= e^{-\alpha\frac{\hat{n}\cdot\vec{\sigma}}{2}} \chi^{*}&
            \chi_L(\vec{p}) &= e^{-\alpha\frac{\hat{n}\cdot\vec{\sigma}}{2}} \chi^{*}
        \end{align}

        Si on develope l'\'exponentiel on trouve que $e^{\alpha \hat{n} \cdot \vec{\sigma}} = \mathbf{I}\cosh \alpha + \left( \hat{n} \cdot \vec{\sigma} \right) \sinh \alpha$ et finalement
        \begin{align}
            \begin{pmatrix} 0 & e^{-\alpha(\hat{n} \cdot \vec{\sigma})}\\e^{\alpha(\hat{n} \cdot \vec{\sigma})} & 0 \end{pmatrix} \begin{pmatrix} \chi_R \\ \chi_L \end{pmatrix} &= \begin{pmatrix} \chi_R \\ \chi_L \end{pmatrix} 
        \end{align}

        On identifie $me^{\pm \alpha\left( \hat{n} \cdot \vec{\sigma} \right)} = E \pm \vec{p} \cdot \vec{\sigma}$ et finalement on obtient pour $\psi(\vec{p}) = \begin{pmatrix} \chi_R(\vec{p}) \\ \chi_L(\vec{p}) \end{pmatrix} $
        \begin{align}
            \begin{pmatrix} 0 & E-\vec{p}\cdot \vec{\sigma}\\E + \vec{p} \cdot \vec{\sigma} & 0 \end{pmatrix} \psi &= m\psi
        \end{align}

        On peut identifier $\gamma_0 = \begin{pmatrix} 0 & \mathbf{I} \\ \mathbf{I} & 0 \end{pmatrix} , \gamma_j = \begin{pmatrix} 0 & -\sigma^j\\ \sigma^j & 0 \end{pmatrix} $ et on obtient la forme $\left(\mathbf{P}^\mu\gamma_\mu - m\right)\psi(\vec{p}) = 0$. Et si on prend la transform de Fourier $\psi(p) \to \psi(x), P^\mu \to i\partial^\mu$ on obtient l'\'equation de Dirac qu'on connait.
\end{enumerate}

\section{Groupe de Poincar\'e}

Le groupe de Poincar\'e est form\'e par le produit des espaces de transformation de Lorentz et de translation, ca veut dire $\mathbf{O}(1,3) \times \mathbb{R}^4 = \left( \Lambda, T_a \right)$. On sait que $\Lambda: X \to \Lambda X$ et $T_a: X \to X + a$ et on sait les structures des deux groupes. On voit que $\Lambda, T_a$ commutent. Examinons donc les relations de commutation des alg\`ebres de Lie des deux groupes
\begin{align}
    \left[ J,J \right] &= \text{d\'ej\`a vu}\\\left[ P^\sigma, P^\rho \right] &= 0 \\ \left[ J^{\rho\sigma}, P^\nu \right] &= \pm \left( \eta^{\rho\nu}P^\sigma - \eta^{\sigma\nu}P^\rho\right)
\end{align}

On cherche pour ce groupe de Poincar\'e une repr\'esentation unitaire. On commence en essayant de trouver une repr\'esentation irr\'eductible du groupe de Lorentz. Si on peut trouver une op\'erateur, qui s'appelle un op\'erateur de Casimir, qui commute avec les g\'en\'erateurs de la repr\'esentation irr\'eductible on sait apr\`es le lemme de Schur que cet op\'erateur a une valeur propre pour tout l'espace.

Pour notre groupe de Poincar\'e on peut trouver quelques Casimirs
\begin{itemize}
    \item $C_1 = \mathbf{P}^2$ qui a la valeur propre $m^2 > 0$.
    \item $C_2 = W^2, W^\mu = \frac{1}{2}\epsilon^{\mu \nu \rho \sigma}J_{\nu \rho}P_\sigma$. On montre simplement qu'il commute avec $J, \mathbf{P}$. $C_2$ a des valeurs propres $\sim$ le spin. 
\end{itemize}

La repr\'esentation est different soit $m=0$ soit $m \neq 0$. On \'examine d'abord $m \neq 0$, o\`u on a des sous-espaces propres $(P^0, \vec{P})$ qui s'\'echangent par la transformation de Lorentz. On peut donc se placer dans le referentiel du centre de masse ou $P^0 = m, \vec{P} = 0$. Dans ce referentiel, $W^0 = 0, \vec{W} = m\vec{J}$. On voit donc que $C_2 = m^2J^2 = m^2s(s+1)$. Donc le groupe des rotations dans le centre de masse ets represent\'e par la repr\'esentation de spin $s$. Finalement, les \'etats de la repr\'esentation sont $\ket{\mathbf{P}; s, m}$ avec $\mathbf{P}^2 = m^2, \ket{s,m} \in \mathfrak{so}(3)$ du centre de masse. 

La repr\'esentation de masse nulle est un peu different parce que $P^2 = 0$. On peut quand m\^eme toujours trouver un referentiel ou $P^0 = k, \vec{P} = (0,0,k)$. Et alors $W^0 = \frac{1}{2}\epsilon^{0jk3}J_{jk}k = kJ^z, W^3 = -kJ^z$, $W^2 = 0$ dans les repr\'esentations utilis\'ees. 

Car $J^z \to \mathbf{SO}(2)$ le groupe des rotations autour $\vec{P}$, on se donne une repr\'esentation de $\mathbf{SO}(2)$ de dimension $1$ qui est caracteris\'e par un nombre $\lambda$ nomm\'e \emph{h\'elicit\'e}. Et alors $W^\mu = \lambda P^\mu$ car $J_z$ agit comme $\lambda$ sur les \'etats $(k,0,0,k)$. 

\chapter{04/11/14 --- Espaces de Fock et leur lien avec les groupes $SU(D), SO(D), Sp(2D)$}

Les trois groupes ci-dessus sons les groupes unitaires, orthogonales et symplectiques (preservent le crochet de Poisson). 

\section{Espace de Fock bosonique}

\subsection{L'oscillateur Harmonique}

On prend comme exemple l'oscillateur harmonique. Le crochet ici fait $[p,q] = -i$, par exemple $p = \frac{1}{i}\pd{}{q}$ et l'hamiltonien est $h = \frac{1}{2}\left( p^2 + \omega^2q^2 \right)$ qui a des valeurs propres $\epsilon_n = \omega (n + 1/2)$. On rappele les operateurs cr\'eation et annihilation:
\begin{align}
    q &= \frac{1}{\sqrt{2\omega}}\left( a + a^\dagger \right) & p &= \frac{\sqrt{\omega}}{i\sqrt{2}}\left( a - a^\dagger \right)
\end{align}

On calcule que $\left[ a, a^\dagger \right] = 1$ et que $h = \omega\left( a^\dagger a + \frac{1}{2} \right)$. Pour trouver les \'etats propres on construit un \'etat $\ket{0}$ tel que $a\ket{0} = 0$. On verifie simplement que $\left[ h, a^\dagger \right] = \omega a^\dagger$ et donc $\ket{n} \propto \left( a^\dagger \right)^n\ket{0}$. Ca d\'emonstre que l'espace Hilbert est l'espace vectoriel avec la base $\left\{ \left( a^\dagger \right)^n\ket{0} \right\}$, qui est aussi l'espace de Fock sur $\mathbb{C}$. 

\subsection{Espace de Fock sur $\mathbb{C}^D$}

On se donne $D$ paires des op\'erateurs annihilation et cr\'eation $a_i, a_i^\dagger$ tel que $\left[ a_i, a^\dagger_j \right] = \delta_{ij}$. On se donne un vecteur, dit le vide, tel que $a_i \ket{0} = 0$ pour tout $i$. Alors l'espace de Fock sur $\mathbb{C}^D$ est donn\'e par la base compos\'ee des actions des $a_i^\dagger$ sur $\ket{0}$.

On peut trouver une d\'ecomposition suivant le nombre des ``particules,'' comme $\mathcal{F} = \oplus \mathcal{F}^{(n)}$ tel que $\mathcal{F}^{(0)}$ est le vide, $\mathcal{F}^{(1)} = \left\{ a_j^\dagger\ket{0} \right\}$ pour tout $j$, etc. 

Les op\'erateurs $a_i,a_i^\dagger$ sont donc represent\'es sur cet espace de Fock par sa relation de commutation $\left[ a_i, a_j^\dagger \right] = \delta_{ij}$. On voit que $a_j^\dagger: \mathcal{F}^{(n)} \to F^{(n+1)}$. $a$ agit par les relations de commutation, par exemple
\begin{align}
    a_j a_{j_1}^\dagger \dots a_{j_n}^\dagger \ket{0} &= \delta_{jj_1}a_{j_2}\dots a_{j_D}^\dagger + a_{j_1}^\dagger a_ja_{j_2}^\dagger\dots a_{j_D}^\dagger\ket{0}\\
    &= \dots = \sum\limits_{k}^{}\delta_{j, j_k} a_{j_1}^\dagger \dots a_{j_k}a_{j_k}^\dagger \dots a_{j_D}^\dagger \ket{0}
\end{align}
et on trouve finalement que $a_j: \mathcal{F}^{(n)} \to F^{(n-1)}$

Une autre \'ecriture de tout ce qui passait peut \^etre achev\'ee via les polyn\^omes. On consid\`ere $D$ variables $x_i$, et alors l'espace de Fock sur $\mathbb{C}^D$ est donn\'e par les polyn\^omes sur $\left\{ x_i \right\}$. Et alors $\ket{j_1j_2} = x_1x_2$, et le nombre des particules correspond au degr\'e du polyn\^ome. Alors $a_j^\dagger \leftrightarrow x_j$ et $a_j \leftrightarrow \pd{}{x_j}$ qui suivent les relations de commutation. 

\subsection{Fonction de partition, fonction g\'en\'eratrice}

Rappelons de la m\'ecanique statistique $\mathcal{Z}\left( \beta = \frac{1}{k_BT} \right) = \mathrm{Tr}_{\mathcal{H}}e^{-\beta \hat{H}}$ o\`u $\hat{H}$ est l'hamiltonien. C'est alors aussi la somme sur tout $e^{-\beta\epsilon_n}$ fois la multiplicit\'e de l'\'energie $\epsilon_n$. On consid\'ere $\hat{H} = \sum\limits_{}^{}\omega_j N_j$ avec $N_j = a_j^\dagger a_j$ qui compte le nombre des particules dans l'\'etat $j$. Alors on va calculer
\begin{align}
    \mathcal{Z} &= \mathrm{Tr} \left( e^{-\beta \sum\limits_{}^{}\omega_j N_j} \right)\\
    &= \mathrm{Tr} \left( \prod p_q^{N_j} \right)
\end{align}
avec $q_j = e^{-\beta \omega_j}$ et le produit est pris sur toute $D$ dimensions. Et alors si on a un seul type de particule, alors $N = a^\dagger a$ et on est dans l'espace Fock sur $\mathbb{C}$. Et alors $\mathcal{Z} = \mathrm{Tr}(q^N)$. On chosit la base des nombres des particules $\ket{0}, \ket{1}\dots$. Dans cette bas $N$ est diagonal et $Z = 1 + q + q^2 +\dots = \frac{1}{1-q}$. Alors on voit que l'espace de Fock sur $\mathbb{C}^D$ est egale a $(\mathbb{Fock}(\mathbb{C}))^{\otimes D}$, et que
\begin{align}
    \mathcal{Z} &= \prod \frac{1}{1 - q_i}
\end{align}

\subsection{Lien avec les groupes classiques}

On \'etudie maintenant les op\'erateurs quadratiques de $a, a^\dagger$. On a d\'ej\`a examin\'e $N = a_i^\dagger a_i$ mais on peut considerer $T_{ij} = a_i^\dagger a_j$. On peut facilement voir que $\left[T_{ij}, T_{kl}\right]$ est une combinaison lin\'eaire des $T$. Ca d\'emonstre que les $T$ forment une alg\`ebre de Lie. On trouve a laquelle elle correspond en regardant la dimension. Le nombre des $T_{ij}$ est $D^2$ (parce que $i \in [1,D]$) et on rappele que $\dim SU(D) = D^2 - 1, \dim U(1) = 1$ et donc $T_{ij} = SU(D) \oplus U(1)$. Le $U(1)$ correpsond \`a $\sum N_j$. Alors dans l'espace de Fock on a trouv\'e $SU(D)$. 

Notons qu'on a seulement consider\'e $T_{ij}$ ou il y a le m\^eme nombre des $a, a^\dagger$. On peut aussi consider l'espace complet $a_ia_j, a_j^\dagger a_i, a_i^\dagger a_j^\dagger$. Sans trop des calculs on voit que c'est $Sp(2D)$ les transformations symplectiques sur l'espace de phase de dimension 2D. On ne parle plus de ce groupe mais c'est plus gros $U(D) \in Sp(2D)$. 

\section{Espace de Fock fermionique}

\subsection{Op\'erateur cr\'eation-annihilation}

On note que ci-dessus les trucs (je ne sais pas\dots comme les polyn\^omes) sont symm\'etrique par l'\'echange, On \'etudie maintenant les espaces qui sont anti-symmetrique par l'\'echange. Alors, nos op\'erateurs de cr\'eation et d'annihilation obeissent les lois d'anti-commutation comme $\left\{ b_j, b_i^\dagger \right\} = \delta_{ij}$ et $\left\{ b_j, b_i \right\} = \left\{ b_j^\dagger, b_i^\dagger \right\} = 0$. On veut trouver une repr\'esentation sur un espace de Hilbert. 

On commence en essayant de trouver l'\'etat du vide tel que $b_j\ket{0} = 0$ pour tout $b_j$. On consid\`ere les espaces engendr\'e par les \'etats \`a $n$ particules, les $b_j^\dagger$ agissent sur $\ket{0}$. Par exemple, avec une particule les \'etats possibles sont $b_j\ket{0}$. Mais avec deux particules, $b_i^\dagger b_j^\dagger \ket{0}$, $i$ est forc\'ement pas \'egale \`a $j$ parce que $b_ib_j + b_jb_i  = \delta_{ij}$! C'est exactement la principe d'\'exclusion de Pauli. 

Il y a plusieurs differences des espaces bosoniques. La repr\'esentation qu'on a montr\'e plus t\^ot avec des polyn\^omes n'est plus possible parce que les polyn\^omes n'anti-commutent pas. Les espaces bosoniques sont de dimension infinis, mais les espaces fermioniques sont de dimension finis $2^D$. 

\subsection{Fonction de Partition, fonction g\'en\'eratrice}

Comme avant, l'op\'erateur $N = b_j^\dagger b_j$ compte le nombre de particule. Par exemple, avec un seul \'etat les \'etats possibles sont $\ket{0}, b^\dagger\ket{0}$ et les valeurs propres sous $N$ sont respectivement $0,1$. Alors on a encore l'hamiltonien $H = \omega_j N_j$ et la fonction de partition
\begin{align}
    \mathcal{Z} &= \mathrm{Tr}\left( e^{-\beta}\omega_jD_j \right) = \mathrm{Tr}\left( \prod e^{-\beta \omega_j D_j} \right)
\end{align}

Encore avec un seul \'etat on voit que $\mathcal{Z} = 1 + q$, et avec $D$ \'etats on a $\mathcal{Z} = \prod^D \left( 1 + q_j \right)$. Si on d\'eveloppe ce produit on trouve 
\begin{align}
    \mathcal{Z} &= 1 + \left( q_1 +\dots + q_D \right) + \dots
\end{align}
et c'est encore la base du nombre d'occupation.

\subsection{Le Groupe}

On commence encore en consid\`erant $M_{ij} = b_i^\dagger b_j$ (en particulier $J_{ii} = b_i^\dagger b_i$). Alors $\left[ M_{ij}, M_{kl} \right]$ est encore forc\'ement une combinaison lin\'eaire des $M$, et c'est encore une alg\`ebre de Lie. On voit la dimension est encore $D^2$ et qu'il est isomorphe \`a (je sais pas si c'est la fa\c{c}on correcte de le dire\dots) $U(D) = SU(D) + U(1)$. C'est naturel qu'il est la m\^eme alg\`ebre qu'avant parce qu'il satisfie le m\^eme crochet.

On peut aussi considerer $b_ib_j, b_i^\dagger b_j, b_i^\dagger b_j^\dagger$ qui forment aussi une alg\`ebre de Lie. Le premier est de dimension $\binom{D}{2}$ parce qu'il faut \^etre anti-symmetrique, et alors la dimension totale est $\frac{D(D-1)}{2} + D^2 + \frac{D(D - 1)}{2} = 2D^2 - D = \binom{2D}{2}$. Ca correspond donc \`a l'alg\`ebre $SO(2D)$.

\section{Exemple de quantification d'un champ bosonique}

On va commencer avec un champ classique satisfiant l'\'equation Klein-Gordon $\phi(t,x)$. Le plus simple cas c'est $(\partial_t^2 - \partial_x^2)\phi(t,x) = 0$. Pour simplifier, on se met dans une bo\^ite et impose des conditions limites en $x$ comme $\phi(t,x=0) = \phi(t,x=L) = 0$. Suivant ces conditions on peut d\'ecomposer en s\'erie Fourier $\phi(t,x) = Q_n(t)\sin(nx)$. On le met dans l'\'equation KG et obtient $\left[\partial^2_t Q_n(t) + n^2Q_n(t)\right]\sin (nx) = 0$ ou $\partial_t^2 Q_n(t) + n^2Q_n(t) = 0$ pour tout $n$ et toutes les modes, et on obtient de l'\'equation d'onde l'\'equation des oscillateurs harmoniques.

Si on quantifie chacun de ces oscillateurs avec $Q_n, P_n$ les coordon\'ees pour toutes $n$ obeissant $H_n = \frac{1}{2}\left( P_n^2 + n^2Q_n^2 \right)$, on trouve les op\'erateurs de cr\'eation et d'annihilation $a_n, a_n^\dagger$ et on retrouve les espaces de Fock. L'\'evolution temperelle des $Q_n(t)$ est donn\'ee par $Q_n(t) = e^{itH}Q_ne^{-itH} = \frac{1}{\sqrt{2\omega}}\left( a_ne^{i\omega_n t} + a_n^\dagger e^{-i\omega_nt} \right)$, et
\begin{align}
    \phi(t,x) = \sum\limits_{n}^{}\frac{1}{2\omega_n} \left( a_ne^{i\omega_nt} + a_n^\dagger e^{-i\omega_n t} \right)\sin (nx)
\end{align}

La fonction de partition est alors donn\'ee comme
\begin{align}
    \mathcal{Z} &= \mathrm{Tr} \left( e^{-\beta H} \right)\\
    &= \prod_n \left( \frac{1}{1 - e^{-\beta \omega_n}} \right)
\end{align}

\end{document}
